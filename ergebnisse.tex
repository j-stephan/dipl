\chapter{Ergebnisse}\label{ergebnisse}

\section{Nutzbarkeit der SYCL-Implementierungen}
\label{ergebnisse:nutzbarkeit}

Im Folgenden wird die Nutzbarkeit von drei öffentlich verfügbaren
SYCL"=Implementierungen dargelegt. Dabei handelt es sich um ComputeCpp sowie
die Implementierungen der Firmen Intel und Xilinx. Die im Kapitel~\ref{sycl}
erwähnten Implementierungen hipSYCL und sycl-gtx wurden während dieser Arbeit
nicht in Betracht gezogen, da ihnen zu viele kritische Features fehlen und sie
daher nicht für eine Nutzung von Alpaka geeignet sind.

\subsection{ComputeCpp}

Die folgenden Ausführungen zu ComputeCpp beziehen sich auf die frei verfügbare
\textit{Community Edition} und alle Versionen bis einschließlich Version 1.1.5.

Von den derzeit der Öffentlichkeit zugänglichen SYCL"=Implementierungen ist
ComputeCpp die einzige, die nicht quelloffen ist. Sie wird vom schottischen
Unternehmen Codeplay entwickelt, welches ebenfalls federführend an der
Entwicklung des SYCL"=Standards beteiligt ist. ComputeCpp unterstützt mehr
Hardware"=Plattformen als die anderen SYCL"=Implementierungen

Im Laufe der Implementierung des SYCL"=Backends stellte sich schnell heraus,
dass ComputeCpp nicht in der Lage sein würde, das SYCL"=Alpaka"=Backend zu
verwenden. Dafür gibt es zwei Gründe, die nachstehend weiter ausgeführt werden.

\subsubsection{Zeiger}

Zusätzlich zu der in Kapitel~\ref{implementierung} beschriebenen Problematik
mit Zeigern kommt es zu weiteren Schwierigkeiten, wenn Zeiger in Verbindung mit
ComputeCpp genutzt werden sollen.

ComputeCpp versucht die Information, zu welchem Adressraum ein Zeiger gehört,
durch spezielle Zeiger"=Attribute nachzuliefern. Ein Zeiger des Typs
\texttt{int*}, der auf den globalen Adressraum zeigt, wird vom
ComputeCpp"=Compiler zu \texttt{\_\_global int*} transformiert. Das Ergebnis der
Transformation wird vom selben Compiler jedoch als ein eigener Typ betrachtet,
der nichts mehr \texttt{int*} zu tun hat. Dadurch kommt es zu Problemen mit den
\textit{type traits} der C++"=Standardbibliothek, die einen Zeiger vom Typ
\texttt{\_\_global int*} nicht mehr als \texttt{int}"=Zeiger erkennen.

Da es dem Programmierer ebenfalls verboten ist, diese Attribute selbst zu
verwenden (ComputeCpp generiert einen Syntax"=Fehler), kann eine manuelle
Auswahl entsprechender Code"=Pfade auch nicht vorgenommen werden.

Darüber hinaus sind diese Zeigertypen nicht mit den SYCL"=Klassen kompatibel.
So führt die Umwandlung eines entsprechenden Zeigers in einen
SYCL"=\texttt{multi\_ptr} dazu, dass letzterer nicht mehr mit den atomaren
Funktionen des SYCL"=Standards verwendet werden kann. In diesem Fall meldet
ComputeCpp den Fehler, dass die Verwendung von Adressraum"=Attributen in
Verbindung mit atomaren Funktionen verboten ist.

\subsubsection{Fehlerhafte Instruktionen}

Auf NVIDIA"=GPUs generiert ComputeCpp mitunter Instruktionen, die von NVIDIAs
OpenCL"= oder CUDA"=Umgebung nicht verstanden werden. Dies fällt erst bei der
Ausführung des Kompilats auf, das entsprechende Fehlermeldungen der
NVIDIA"=Laufzeitumgebung meldet. Darüber hinaus fehlen ComputeCpp noch einige
wichtige Instruktionen, die für ein Funktionieren mit Alpaka nötig sind,
darunter viele mathematische Funktionen.

\subsection{Intel}

Eine weitere wichtige Implementierung des SYCL"=Standards wird seit Anfang des
Jahres 2019 von der Firma Intel herausgegeben. Diese quelloffene Variante ist
auf die Nutzung der Intel"=OpenCL"=Umgebungen für CPUs und GPUs ausgelegt. Es
ist aufgrund der parallel zu dieser Arbeit verlaufenen Weiterentwicklung
absehbar, dass kurz"= bis mittelfristig auch die FPGAs dieses Herstellers
unterstützt werden sollen.

Intels Compiler ist die einzige SYCL"=Implementierung, die Alpaka"=Quelltexte
mit aktiviertem SYCL"=Backend kompilieren konnte und kommt deshalb zur
Verfizierung desselben zum Einsatz.

\subsection{Xilinx}

Die SYCL"=Implementierung der Firma Xilinx hängt eng mit der Entwicklung des
Intel"=SYCL"=Compilers zusammen. Dabei wird in unregelmäßigen Abständen die
Code"=Basis des Intel"=Compilers übernommen und die Xilinx"=eigenen Codepfade
darin integriert. Das hat zur Folge, dass Fehlerkorrekturen des Intel"=Compilers
erst mit einiger Verzögerung in die Xilinx"=Implementierung Eingang finden.

Darüber hinaus erwies sich die Schnittstelle des SYCL"=Compilers zu Xilinx'
SDAccel"=Plattform, welche die eigentliche Synthese durchführt, sowie zu Xilinx'
OpenCL"=Treibern im Laufe dieser Arbeit als fehleranfällig oder unvollständig.
Aufgrund dieser Probleme war eine Nutzung des Alpaka"=SYCL"=Backends für
Xilinx"=FPGAs nicht möglich.

\subsubsection{Mathematische Funktionen}

Der Compiler generiert aus einigen mathematischen SYCL"=Funktionen
Instruktionen, die in Xilinx' OpenCL"=Treiber nicht vorhanden sind. Dies ist
zwar auf ein falsches Benennungsschema innerhalb der OpenCL"=Implementierung
zurückzuführen, steht einer Nutzbarkeit im Zusammenhang mit Alpaka aber trotzdem
im Wege. Andere mathematische Funktionen führen unter ungünstigen Umständen
durch ihre Nutzung dazu, dass der Compiler selbst abstürzt.  

\subsubsection{Strukturen}

Zu einem gravierenden Problem kommt es bei der Nutzung von benutzerdefinierten
Strukturen. Sofern diese außerhalb eines Kernels definiert und dann innerhalb
eines Kernels verwendet werden, kommt es zu einem Absturz des Compilers. Der
Grund dafür liegt in einem Fehler des im Hintergrund verwendeten
Xilinx"=OpenCL"=Compilers \texttt{xocc}, welcher die Synthese steuert. Nach
Aussage der an der SYCL"=Implementierung beteiligten Xilinx"=Mitarbeiter genießt
die Behebung dieses Fehlers niedrige Priorität, weshalb in nächster Zeit nicht
mit Besserung zu rechnen ist. Dieser Fehler ist der hauptsächliche Grund, warum
das aus vielen Strukturen bestehende Alpaka nicht mit Xilinx'
SYCL"=Implementierung verwendet werden kann.

\subsubsection{OpenCL-Treiber}

Xilinx' OpenCL"=Treiber, auf dem die SYCL"=Implementierung aufsetzt, erwies sich
im Zusammenspiel mit SYCL ebenfalls als äußerst instabil. Aufgrund seiner
internen Struktur ist er nicht in der Lage, einmal reservierten Speicher wieder
freizugeben, wenn das reservierende Programm abstürzt. Der Speicher bleibt so
lange unzugänglich, bis er neugestartet wird. Da dies nur mit
Administrationsrechten funktioniert, ist dies de facto ein Ausschlusskriterium
für den Einsatz in Rechenzentren oder Hochleistungs"=Clustern.

Sofern der Treiber nicht das oben beschriebene Verhalten zeigt, kann ein
fehlerhaftes Programm auch zum Komplettabsturz des Gesamtsystems führen. In
diesem Fall ist das System per Fernzugriff nicht mehr erreichbar und muss
vom Administrator (oder physisch per Reset"=Taste) neugestartet werden. Für den
Einsatz in Rechenzentren und vergleichbaren Einrichtungen ist dieses
Fehlerverhalten denkbar ungeeignet.

\subsection{Zusammenfassung}
 
Von den beschriebenen SYCL"=Implementierungen konnte nur der Intel"=Compiler im
Zusammenhang mit Alpaka genutzt werden, was als Ziel"=Hardware nur Intel"=CPUs
und -GPUs zulässt. Insbesondere FPGAs können zum aktuellen Zeitpunkt nicht vom
Alpaka"=SYCL"=Backend verwendet werden.

\section{Vergleich zwischen Alpaka und SYCL}

Im direkten Vergleich erwies sich SYCL gegenüber Alpaka als die modernere,
intuitivere und angenehmer zu benutzende Schnittstelle.

Dazu tragen SYCLs Orientierung an modernen C++"=Standards (alle untersuchten
Implementierungen unterstützen den C++17"=Standard, Intel und Xilinx den in
Entwicklung befindlichen C++20"=Standard) sowie die stilistische Nähe zur
C++"=Standardbibliothek bei. Dem gegenüber stehen Alpakas Stil"=Konventionen,
die vom in der C++"=Standardbibliothek verwendeten \textit{snake case}
(\texttt{eine\_kleine\_funktion()}) zugunsten des
\textit{lower camel case} (\texttt{eineKleineFunktion()}) abweichen. Darüber
hinaus finden sich in Alpaka stilistische Eigentümlichkeiten, die in den meisten
C++"=Projekten unüblich sind, z.B. die Schreibweise als \texttt{char const *}
anstelle des weiter verbreiteten \texttt{const char *}. Damit steht das
C++"=Projekt Alpaka auch im Gegensatz zu den vom C++"=Standardisierungskomitee
veröffentlichen Stilrichtlinien, den \textit{C++ Core Guidelines}.

Die Modellierung von Aufgabengraphen bzw.\ der Abhängigkeiten zwischen Kerneln
ist in SYCL deutlich einfacher als in Alpaka. Während dies in SYCL automatisch
von der Laufzeitumgebung übernommen wird, muss der Programmierer in Alpaka
selbst tätig werden -- der Aufwand ist in Alpaka also höher.

Die meisten Konzepte sind in SYCL und Alpaka jedoch recht ähnlich, sodass
hinsichtlich der Mächtigkeit keine großen Unterschiede bestehen. Darüber hinaus
hat Alpaka gegenüber SYCL den faktischen Vorteil der Hardware"=Unterstützung.
Während SYCL zur Zeit nur mit Intel"=CPUs und -GPUs zufriedenstellend
funktioniert (und möglicherweise nicht getesteter Automotive"= und
Embedded"=Hardware), ist Alpaka auf NVIDIA"= und AMD"=GPUs sowie über OpenMP auf
allen CPUs lauffähig. Daher ist Alpaka bereits in einigen produktiven
Anwendungen im Einsatz, während sich SYCLs Ökosystem bislang auf die diversen
Implementierungen sowie einige von der Firma Codeplay entwickelte Bibliotheken
für Mathematik und neuronale Netzwerke beschränkt. 

\section{Messergebnisse}
\label{ergebnisse:messergebnisse}

\subsection{Methoden}
\label{ergebnisse:messergebnisse:methoden}

\subsubsection{Verwendete Hard- und Software}
\label{ergebnisse:messergebnisse:methoden:stack}

\subsubsection{Beispielalgorithmus}
\label{ergebnisse:messergebnisse:methoden:algorithmus}

\cite{benner2018}
