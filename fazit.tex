\chapter{Fazit}
\label{fazit}

Es konnte gezeigt werden, dass SYCL sich als Backend für die Alpaka"=Bibliothek
grundsätzlich eignet. Da zur Zeit noch gravierende Inkompatibilitäten zwischen
SYCL und Alpaka bestehen, ist eine Entwicklung über den Prototypen"=Status
im Moment nicht durchführbar. Diesbezüglich sind noch einige konzeptionelle
Änderungen in Alpaka und/oder SYCL notwendig.

Aufgrund des Entwicklungsstands der verfügbaren SYCL"=Implementierungen ist eine
Ausführung von Alpaka"=Programmen auf den meisten Hardware"=Plattformen derzeit
nicht möglich. Lediglich Intel"=CPUs und "=GPUs sind gegenwärtig nutzbar, sofern
man die Beschränkungen des Backend"=Prototypen in Kauf zu nehmen bereit ist.

Gegenüber Alpaka bietet SYCL das modernere, intuitivere und standardnähere
Programmier"=Interface. SYCLs fragmentiertes Ökosystem sowie der eingeschränkte
Hardware"=Support sprechen im Moment allerdings noch recht deutlich gegen die
Nutzung dieses insgesamt vielversprechenden Ansatzes zur parallelen Programmierung. Es
bleibt deswegen abzuwarten, ob SYCL in den nächsten Monaten und Jahren weitere
Verbreitung erfährt oder in der momentanen relativen Bedeutungslosigkeit
verharrt. Wichtig wäre hier eine baldige vollständige Unterstützung von
NVIDIA"=GPUs, die den Bereich des HPC mit weitem Abstand dominieren. Auch für
moderne AMD"=GPUs existiert bislang keine SYCL"=Implementierung, die auf dem
HPC"=Sektor vorbehaltlos einzusetzen ist. Dadurch kann SYCL in der
GPGPU"=Programmierung keine Rolle spielen, sofern man von den vergleichsweise
leistungsschwachen Intel"=Laptop"=GPUs absieht. Diesbezüglich wird interessant
sein, ob und wie gut Intels zukünftige dedizierte GPU"=Plattform mit SYCL
genutzt werden kann.

Für die Zukunft der Alpaka"=Bibliothek wäre eine weniger starke Anlehnung an
CUDA wünschenswert. Diese ist zwar durch die historische Entwicklung der
Alpaka"=Bibliothek begründet, erschwert aber die Integration konkurrierender
Plattformen, wie die Probleme bei der Entwicklung des SYCL"=Backends zeigten.
Beispielhaft sei hier die Unterstützung unterschiedlicher Beschleunigertypen
und/oder Hersteller genannt, die erst zur Laufzeit des Programms bekannt sind.
Dieses Problem betrifft nicht nur SYCL, sondern auch neuere Fassungen des
OpenMP"=Standards. Nicht nur an dieser Stelle ist daher eine Diskussion über das
zukünftige Design der Bibliothek innerhalb der Alpaka"=Entwicklergemeinde
notwendig. Trotz alledem ist Alpaka für den Einsatz in der Praxis gut geeignet,
wie bereits produktiv genutzte Programme auf Alpaka"=Basis beweisen.

Die Programmierbarkeit von FPGAs sowohl durch das Alpaka"=Backend als auch durch
SYCL selbst ist langfristig eine attraktive Option. Hier dürfte zunächst die
Intel"=Implementierung des SYCL"=Standards von Bedeutung sein, da eine
Unterstützung der hauseigenen FPGAs mittelfristig wahrscheinlich ist und die
bisherige Implementierung für CPUs und GPUs recht ausgereift wirkt.

Dem gegenüber steht die auf der Intel"=Implementierung aufsetzende
Xilinx"=Implementierung, die sich noch in einem sehr frühen Entwicklungsstadium
befindet und unter einer sehr kleinen Entwicklerzahl leidet. Es ist dennoch
lohnenswert, das Projekt der FPGA"=Unterstützung für beide Hersteller in Alpaka
weiter zu verfolgen: Einerseits hätte Alpaka damit ein wichtiges
Alleinstellungsmerkmal gegenüber vergleichbaren Projekten wie Kokkos oder
HPX.Compute. Andererseits sind die Hardware"= und Parallelisierungseigenschaften
der FPGAs für viele Algorithmen -- die zum Teil schon in einer Alpaka"=Fassung
vorliegen -- sehr interessant. Insbesondere Programme aus dem Umfeld der
Laser"=Teilchenbeschleunigung könnten von einer physischen Schaltung
profitieren, die auf der Basis eines bildverarbeitenden Algorithmus 
synthetisiert und an einen mit hoher Messfrequenz betriebenen Detektor
angeschlossen wurde.
