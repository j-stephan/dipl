\chapter{Fazit}
\label{fazit}

Es konnte gezeigt werden, dass SYCL sich als Backend für die Alpaka"=Bibliothek
grundsätzlich eignet. Da jedoch gravierende Inkompatibilitäten zwischen SYCL und
Alpaka bestehen, ist eine Entwicklung über den Prototypen"=Status hinaus derzeit
nicht umzusetzen. Diesbezüglich sind noch einige konzeptionelle Änderungen in
Alpaka und/oder SYCL notwendig.

Aufgrund des Zustands der verfügbaren SYCL"=Implementierungen ist eine
Ausführung von Alpaka"=Programmen auf den meisten Hardware"=Plattformen derzeit
nicht möglich. Lediglich Intel"=CPUs und "=GPUs sind zur Zeit nutzbar, sofern
man die Beschränkungen des Backend"=Prototypen in Kauf zu nehmen bereit ist.
Dagegen ist die Nutzbarkeit von FPGAs sowohl durch das Alpaka"=Backend als auch
durch SYCL selbst in nächster Zeit fraglich. Hier dürfte vor allem die
Beobachtung der Intel"=Implementierung des SYCL"=Standards von Bedeutung sein,
da eine Unterstützung der hauseigenen FPGAs mittelfristig wahrscheinlich ist und
die bisherige Implementierung recht ausgereift wirkt. Dem gegenüber steht die
auf der Intel"=Implementierung aufsetzende Xilinx"=Implementierung, die
insgesamt sehr instabil ist und unter einer sehr kleinen Entwicklerzahl leidet.
Es ist dennoch lohnenswert, das Projekt der FPGA"=Unterstützung für beide
Hersteller in Alpaka weiter zu verfolgen, weil Alpaka so ein wichtiges
Alleinstellungsmerkmal gegenüber vergleichbaren Projekten wie Kokkos oder
HPX.Compute hätte.

Gegenüber Alpaka bietet SYCL das modernere, intuitivere und standardnähere
Programmier"=Interface. Das fragmentierte Ökosystem sowie der eingeschränkte
Hardware"=Support sprechen im Moment aber recht deutlich gegen die Nutzung
dieses insgesamt vielversprechenden Ansatzes zur parallelen Programmierung. Es
bleibt daher abzuwarten, ob SYCL in den nächsten Monaten und Jahren weitere
Verbreitung erfährt oder in der Bedeutungslosigkeit versinkt. Wichtig wäre hier
eine kurzfristige und umfangreiche Unterstützung von NVIDIA"=GPUs, die
den Bereich des HPC mit weitem Abstand dominieren. Auch für moderne AMD"=GPUs
existiert bislang keine SYCL"=Implementierung, die auf dem HPC"=Sektor
vorbehaltlos einzusetzen ist. Dadurch kann SYCL in der GPGPU"=Programmierung
keine Rolle spielen, sofern man von den vergleichsweise leistungsschwachen
Intel"=Laptop"=GPUs absieht. Diesbezüglich wird interessant sein, ob und wie gut
Intels zukünftige dedizierte GPU"=Plattform mit SYCL genutzt werden kann.
