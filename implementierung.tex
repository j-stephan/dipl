\chapter{Implementierung des SYCL-Backends der Alpaka-Bibliothek}\label{implementierung}

\section{Allgemeiner Aufbau}\label{implementierung:aufbau}

\section{Besonderheiten des SYCL-Backends}\label{implementierung:besonderheiten}

\section{Probleme}\label{implementierung:probleme}

\subsection{Event-System}

\begin{itemize}
    \item Alpaka: Übernimmt CUDA-Eventsystem. Erfordert manuelle Erzeugung des
          Events und die Einsortierung in den SYCL-Stream.
    \item SYCL: Events dienen eigentlich nur dem Profiling. Abhängigkeiten
          werden von der SYCL-Laufzeitumgebung synchronisiert und basieren nicht
          auf Events, sondern der Verfügbarkeit des Speichers.
    \item Lösung: Leere Funktion für das SYCL-Backend. \texttt{alpaka::wait()}
\end{itemize}

\subsection{Block-Synchronisierung}

\begin{itemize}
    \item Alpaka will überall einen Dimensions-Parameter, nur nicht bei der
          Block-Synchronisierung.
    \item SYCL braucht den Dimensionsparameter bei der Block-Synchronisierung.
    \item Lösung: Erfolgt, funktioniert über Templates.
\end{itemize}

\subsection{Shared Memory}

\begin{itemize}
    \item SYCL kann keinen shared memory innerhalb eines Kernels anlegen, Alpaka
          schon.
    \item Lösung: Keine möglich, Begründung des SYCL-Kommittees einfügen.
\end{itemize}

\subsection{Atomics}

\begin{itemize}
    \item SYCL kann anhand eines rohen Zeigers nicht ableiten, welcher
          \texttt{multi\_ptr}-Typ verwendet werden soll.
    \item \texttt{multi\_ptr} wird für SYCL-Atomics benötigt.
    \item Alpaka gibt uns nur rohe Zeiger.
    \item Lösung: Keine. Im Moment funktionieren Atomics nur für den globalen
          Addressraum. Begründung SYCL-Kommittee einfügen.
\end{itemize}

\subsection{Zeiger}

\begin{itemize}
    \item Alpaka: Will überall reine Zeiger (Kernel-Code)
    \item SYCL: Will überall \texttt{accessor}, alternativ \texttt{multi\_ptr}
    \item Lösung: \texttt{multi\_ptr} lässt sich implizit zu reinem Zeiger casten
    \item Hinweis: SYCL-Spezifikation gibt fälschlicherweise an, dass
          \texttt{accessor} implizit castbar ist.
\end{itemize}
