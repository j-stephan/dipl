\makeglossaries

\newacronym{api}{API}{\textit{application programming interface}}
\newacronym{alu}{ALU}{\textit{arithmetic logic unit}}
\newacronym{apu}{APU}{\textit{accelerated processing unit}}
\newacronym{asic}{ASIC}{\textit{ap\-pli\-ca\-tion-spe\-cif\-ic integrated circuit}}
\newacronym{clb}{CLB}{\textit{configurable logic block}}
\newacronym{cmt}{CMT}{\textit{clock management tile}}
\newacronym{cpu}{CPU}{\textit{central processing unit}}
\newacronym{cu}{CU}{\textit{compute unit}}
\newacronym{dram}{DRAM}{\textit{dynamic random-access memory}}
\newacronym{dsp}{DSP}{\textit{digital signal processor}}
\newacronym{fifo}{FIFO}{\textit{first in, first out}}
\newacronym{fpga}{FPGA}{\textit{field programmable gate array}}
\newacronym{gpgpu}{GPGPU}{\textit{general-purpose computing on graphics
                                  processing units}}
\newacronym{gpu}{GPU}{\textit{graphics processing unit}}
\newacronym{hls}{HLS}{\textit{High-Level-Synthese}}
\newacronym{hpc}{HPC}{\textit{high-performance computing}}
\newacronym{ic}{IC}{\textit{integrated circuit}}
\newacronym{ii}{II}{\textit{initiation interval}}
\newacronym{io}{I/O}{\textit{input/output}}
\newacronym{iob}{IOB}{\textit{input/output block}}
\newacronym{lut}{LUT}{Lookup-Tabelle}
\newacronym{pe}{PE}{\textit{processing element}}
\newacronym{ram}{RAM}{\textit{random-access memory}}
\newacronym{simd}{SIMD}{\textit{single instruction, multiple data}}
\newacronym{slr}{SLR}{\textit{super logic region}}
\newacronym{vhsic}{VHSIC}{\textit{very high speed integrated circuit}}

\newglossaryentry{alpaka}{name = Alpaka,
                          description = {C++-Abstraktionsbibliothek für die
                                         Beschleunigerprogrammierung}}
\newglossaryentry{architecture}{name = \textit{Architecture},
                                description = {Verhaltensbeschreibung einer
                                               VHDL-Komponente}}
\newglossaryentry{blas}{name = BLAS,
                        description = {\textit{Basic Linear Algebra
                                       Subprograms}, Fortran-Bibliothek für
                                       Operationen der linearen Algebra}}
\newglossaryentry{cuda}{name = CUDA,
                        description = {GPGPU-Sprache der Firma NVIDIA}}
\newglossaryentry{device}{name = Device,
                          description = {Alpaka- und SYCL-Bezeichnung für einen
                                         Beschleuniger},
                          plural = Devices}
\newglossaryentry{entity}{name = \textit{Entity},
                          description = {Schnittstellenbeschreibung einer
                                         VHDL-Komponente}}
\newglossaryentry{flipflop}{name = Flip-Flop,
                            description = {Elektronische Schaltung mit zwei
                                           Zuständen des Ausgangssignals, die
                                           ein Bit beliebig lange speichern
                                           kann. Die Zustandsänderung erfolgt
                                           durch Ereignisse, nicht durch die
                                           Zeit.}}
\newglossaryentry{hip}{name = HIP,
                       description = {\textit{Heterogeneous-compute Interface
                                      for Portability}, von CUDA abgeleitete
                                      GPGPU-Sprache der Firma AMD}}
\newglossaryentry{host}{name = Host,
                        description = {Alpaka- und SYCL-Bezeichnung für den Teil
                                       des Rechners, der den Kontrollfluss
                                       steuert}}
\newglossaryentry{iso}{name = ISO,
                       description = {\textit{International Organization for
                                      Standardization}. Kurzname von griechisch
                                      \textgreek{ίσος} (\glqq gleich\grqq).
                                      Internationale Vereinigung von
                                      Normungsorganisationen, umfasst unter
                                      anderem das Deutsche Institut für Normung
                                      (DIN)}}
\newglossaryentry{kernel}{name = Kernel,
                          description = {Programm, das auf einem Beschleuniger
                                         ausgeführt wird.},
                          plural = Kernel}
\newglossaryentry{latch}{name = Latch,
                         description = {Sonderform des Flip-Flops. Kann den
                                        Zustand von der Anfangs- bis zur
                                        Endflanke des Taktschrittes ändern.}}
\newglossaryentry{opencl}{name = OpenCL,
                          description = {\textit{Open Computing Language},
                                         standardisierte C-Schnittstelle des
                                         Industriekonsortiums \textit{Khronos}
                                         für die Programmierung von
                                         Beschleunigern}}
\newglossaryentry{openmp}{name = OpenMP,
                          description = {\textit{Open Multi-Processing},
                                         standardisierte C, C++- und
                                         Fortran-Schnittstelle des
                                         Industriekonsortiums
                                         \textit{OpenMP Architecture Review
                                         Board} für die parallele Programmierung
                                         auf Thread-Ebene}}
\newglossaryentry{spir}{name = SPIR,
                        description = {\textit{Standard Portable Intermediate
                                       Language}, standardisierte
                                       Zwischencode-Sprache des
                                       Industriekonsortiums \textit{Khronos}}}
\newglossaryentry{sycl}{name = SYCL,
                        description = {standardisierte C++-Schnittstelle des
                                       Industriekonsortiums \textit{Khronos} für
                                       die Programmierung von Beschleunigern}}
\newglossaryentry{tbb}{name = TBB,
                       description = {\textit{Threading Building Blocks}, von
                                      der Firma Intel entwickelte Bibliothek für
                                      die parallele Programmierung von
                                      Mehrkern-CPUs}}
\newglossaryentry{verilog}{name = Verilog,
                           description = {Hardware-Beschreibungssprache}}
\newglossaryentry{vhdl}{name = VHDL,
                        description = {\textit{VHSIC Hardware Description
                                       Language},
                                       Hardware-Beschreibungssprache}}
