\makeglossaries

\newacronym{api}{API}{\textit{application programming interface}}
\newacronym{alu}{ALU}{\textit{arithmetic logic unit}}
\newacronym{apu}{APU}{\textit{accelerated processing unit}}
\newacronym{blas}{BLAS}{\textit{Basic Linear Algebra Subprograms}}
\newacronym{clb}{CLB}{\textit{configurable logic block}}
\newacronym{cmt}{CMT}{\textit{clock management tile}}
\newacronym{cpu}{CPU}{\textit{central processing unit}}
\newacronym{dsp}{DSP}{\textit{digital signal processor}}
\newacronym{fpga}{FPGA}{\textit{field programmable gate array}}
\newacronym{gpgpu}{GPGPU}{\textit{general-purpose computing on graphics processing units}}
\newacronym{gpu}{GPU}{\textit{graphics processing unit}}
\newacronym{hpc}{HPC}{\textit{high"=performance computing}}
\newacronym{ii}{II}{\textit{initiation interval}}
\newacronym{iob}{IOB}{\textit{input/output block}}
\newacronym{lut}{LUT}{Lookup-Tabelle}
\newacronym{slr}{SLR}{\textit{super logic region}}

\newglossaryentry{cuda}{name = CUDA,
                        description = {GPGPU-Sprache der Firma NVIDIA}}
\newglossaryentry{device}{name = Device,
                          description = {Alpaka- und SYCL-Bezeichnung für einen Beschleuniger},
                          plural = Devices}
\newglossaryentry{kernel}{name = Kernel,
                          description = {Programm, das auf einem Beschleuniger
                                         ausgeführt wird.},
                          plural = Kernel}
\newglossaryentry{hip}{name = HIP,
                       description = {\textit{Heterogeneous-compute Interface for Portability}, von CUDA abgeleitete GPGPU-Sprache der Firma AMD}}
\newglossaryentry{host}{name = Host,
                        description = {Alpaka- und SYCL-Bezeichnung für den Teil des Rechners, der den Kontrollfluss steuert}}
\newglossaryentry{iigls}{name = Initiation Interval,
                         description = {Anzahl der Zyklen zwischen dem Ausführungsbeginn aufeinanderfolgender Schleifeniterationen}}
\newglossaryentry{opencl}{name = OpenCL,
                          description = {\textit{Open Computing Language}, offener Standard für die Programmierung von Beschleunigern}}
\newglossaryentry{openmp}{name = OpenMP,
                          description = {\textit{Open Multi-Processing}, offener Standard für die Shared-Memory-Programmierung}}
\newglossaryentry{spir}{name = SPIR,
                        description = {\textit{Standard Portable Intermediate Language}, offene Zwischencode-Sprache}}
\newglossaryentry{sycl}{name = SYCL,
                        description = {moderne C++-Abstraktionsschicht über OpenCL}}
\newglossaryentry{tbb}{name = TBB,
                       description = {\textit{Threading Building Blocks}, von der Firma Intel entwickelte Bibliothek für die parallele Programmierung von Mehrkern-CPUs}}
