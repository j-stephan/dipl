\chapter{Einleitung}\label{einleitung}

\section{Motivation}\label{einleitung:motivation}

\gls{fpga}

\gls{gpgpu}

\gls{kernel}

\section{Forschungsstand}\label{einleitung:forschung}

\cite{howes2006} - Vergleich zwischen GPUs, FPGAs und Playstation-2-Vektoreinheit durch einheitlichen Quellcode (A Stream Compiler / ASC) \\
\cite{gaster2013} - SYCL-Vorläufer OpenCL C++ \\
\cite{wong2016} - Wechselwirkung zwischen SYCL und C++ (inkl. Problemen) \\
\cite{fifield2016} - OpenCL-Optimierung auf Xilinx-FPGAs \\
\cite{copik2017} - SYCL-Backend für HPX.Compute\\
\cite{doumoulakis2017} - SYCL-OpenCL-Interoperabilität auf Xilinx-FPGAs \\
\cite{krebs2019} - SYCL-Backend für RVC-CAL-Datenflussprogramme \\
\cite{burns2019} - SYCL-DNN (plattformunabhängig) vs. cuDNN, MIOpen (plattformabhängig) \\
\cite{sycl2019} - SYCL-Spec.\\
\cite{rodriguez-gutiez2019} - portables BLAS

Hervorzuheben ist außerdem Peter Žužeks Masterarbeit aus dem Jahre 2016, in
deren Rahmen eine eigene \gls{sycl}"=Implementierung entwickelt wurde. \cite{zuzek2016}

\section{Ziel der Arbeit}\label{einleitung:ziel}
