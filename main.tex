% \RequirePackage[ngerman=ngerman-x-latest]{hyphsubst}
\documentclass[english,greek,ngerman,
               abstract=section,abstract=nottotoc,abstract=all,abstract=vfill,
               declaration=true,declaration=section,declaration=totoc]{tudscrbook}

\usepackage[utf8]{inputenc}
\usepackage[T1]{fontenc}

\usepackage{anyfontsize}
\usepackage{babel}
\usepackage[backend=biber,style=alphabetic]{biblatex}
\usepackage[math]{blindtext}
\usepackage{booktabs}
\usepackage{caption}
\usepackage[outline]{contour}
\usepackage{csquotes}
\usepackage{dialogue}
\usepackage[acronyms,nonumberlist,nopostdot,toc,savewrites,xindy={glsnumbers=false}]{glossaries}
\usepackage[ngerman=ngerman-x-latest]{hyphsubst}
\usepackage{isodate}
\usepackage{makecell}
\usepackage[newfloat]{minted}
\usepackage{multirow}
\usepackage{nth}
\usepackage{pdfpages}
\usepackage{pgfplots}
\usepackage{selinput}
\usepackage[binary-units]{siunitx}
\usepackage{tablefootnote}
\usepackage{tabulary}
\usepackage{tcolorbox}
\usepackage{textcomp}
\usepackage{tikz}

\usetikzlibrary{arrows,
                arrows.meta,
                decorations,
                decorations.markings,
                decorations.pathreplacing,
                decorations.text}

\addbibresource{bibliography.bib}

\makeglossaries

\newacronym{fpga}{FPGA}{\textit{field programmable gate array}}

\newglossaryentry{kernel}{name = Kernel,
                          description = {Programm, das auf einem Beschleuniger,
                                         wie etwa einem FPGA, ausgeführt wird.},
                          plural = Kernel}


% Germanisiere LaTeX
\deftranslation[to=German]{Acronyms}{Abkürzungsverzeichnis}
\deftranslation[to=German]{Glossary}{Glossar}
\renewcommand*{\listlistingname}{Quelltextverzeichnis}
\sisetup{locale = DE}

% Einstellungen für minted
\newenvironment{code}{\captionsetup{type=listing}}{}
\SetupFloatingEnvironment{listing}{name=Quelltext}
\definecolor{keyword-green}{RGB}{0, 128, 0}
\BeforeBeginEnvironment{minted}{\begin{tcolorbox}}
\AfterEndEnvironment{minted}{\end{tcolorbox}}
\newcommand{\lstfont}[1]{\color{#1}\small\ttfamily}

\begin{document}

\faculty{Fakultät Informatik}
\department{}
\institute{Institut für Technische Informatik}
\chair{Seniorprofessor Dr.-Ing. habil. Rainer G. Spallek}

\date{28.11.2019}
\title{Entwicklung eines SYCL-Backends für die Alpaka-Bibliothek und dessen
       Evaluation mit Schwerpunkt auf FPGAs}
\subject{diploma}
\graduation[Dipl.-Inf.]{Diplom-Informatiker}

\author{%
        Jan Stephan%
        \matriculationnumber{3755136}%
        \dateofbirth{8.5.1991}%
        \placeofbirth{Wilhelmshaven}}
\matriculationyear{2012}
\course{Informatik}

\supervisor{Dr.-Ing. Oliver Knodel \and
            Matthias Werner, M.Sc.}
\referee{Prof. Dr.-Ing. habil. Rainer G. Spallek \and
         Prof. Dr. rer. nat. Ulrich Schramm}

\dedication{Für hilfreiche Antworten und Anregungen während der Implementierung
            des Alpaka"=SYCL"=Backends bedanke ich mich herzlichst bei Herrn
            René Widera. Großer Dank gebührt darüber hinaus Herrn Jonas Schenke,
            der mir während der Arbeit mit dem \textit{jungfrau"=photoncounter}
            noch in den tiefsten Abendstunden mit zahlreichen Ratschlägen zur
            Seite stand.}

%%%%%%%%%%%%%%%%%%%%%%%%%%%%%%%%%%%%%%%%%%%%%%%%%%%%%%%%%%%%%%%%%%%%%%%%%%%%%%%%
% Vorspann
%%%%%%%%%%%%%%%%%%%%%%%%%%%%%%%%%%%%%%%%%%%%%%%%%%%%%%%%%%%%%%%%%%%%%%%%%%%%%%%%

\maketitle

\includepdf[pages=-]{aufgabenstellung.pdf}
\begin{abstract}[ngerman]
Eine deutsche Zusammenfassung.
\nextabstract[english]
An English summary.
\end{abstract}



\tableofcontents

\glsaddall
\printglossary
\printglossary[type=\acronymtype]

%%%%%%%%%%%%%%%%%%%%%%%%%%%%%%%%%%%%%%%%%%%%%%%%%%%%%%%%%%%%%%%%%%%%%%%%%%%%%%%%
% Hauptteil
%%%%%%%%%%%%%%%%%%%%%%%%%%%%%%%%%%%%%%%%%%%%%%%%%%%%%%%%%%%%%%%%%%%%%%%%%%%%%%%%

\chapter{Einleitung}\label{einleitung}

Die Hardware"=Ausstattungen moderner Computersysteme werden zunehmend heterogen.
Mobiltelefone und PCs verfügen nicht nur über mehrkernige, leistungsstarke
Prozessoren (\gls{cpu}), sondern auch über dedizierte Chips zur
Grafikdarstellung (\gls{gpu}). Währenddessen verschwimmt bei Spielekonsolen und
Laptops die Unterscheidung zwischen Haupt- und Grafikprozessoren dank
integrierter Chips, die beide Funktionalitäten performant abdecken können
(\gls{apu}). Im Bereich des \gls{hpc} sind verschiedene Beschleunigertypen seit
jeher im Einsatz, seien es CPUs mit besonders vielen Kernen, von den GPUs
abstammende Beschleuniger ohne grafische Ausgabe wie NVIDIAs Tesla"=Reihe, oder
Hybride wie Intels kurzlebige Xeon"=Phi"=Produkte.
% Begriff des Beschleunigers verdient Erklärung
% Kapitel 2. heißt FPGAs als Beschleuniger, daher spätestens dort erörtern, ggf. in Hinblick auf Einsatzfelder und Beschleunigerhierarchien

Allen gemeinsam ist die Idee der Beschleunigung eines Algorithmus durch dessen
Parallelisierung, also in der Regel die parallele Ausführung einzelner Schleifeniterationen
auf den in den Beschleunigern vorhandenen Kernen, Multiprozessoren usw. Wie eine
CPU sind die Beschleuniger allerdings darauf ausgelegt, von verschiedenen
Algorithmen genutzt werden zu können, und entsprechend allgemein aufgebaut. Der
Programmierer findet also möglicherweise eine Hardware"=Plattform vor, die für
seinen Algorithmus nicht ideal geeignet ist.

Chips, die für einen speziellen Algorithmus entworfen werden, erfordern durch
die Produktionskosten einen hohen finanziellen Aufwand und sind daher erst ab
hohen Stückzahlen interessant. Darüber hinaus bergen sie das Risiko, geänderten
Einsatzzwecken -- z.B. durch beim Schaltungsentwurf nicht bedachte Aspekte oder
später geänderte Anforderungen -- nicht mehr zu genügen, was wiederum eine
aufwendige Neuentwicklung und -produktion nötig macht. Im schlimmsten Fall ist
der zu tauschende Chip für einen Menschen physisch nicht mehr zu erreichen, wie
etwa in der Raumfahrt.
% ASICs?

Durch dynamisch rekonfigurierbare bzw. programmierbare Hardware lassen sich die
beschriebenen Herausforderungen besser meistern -- wenn auch mit dem Nachteil,
gegenüber spezialisierten Chips deutlich langsamer zu sein. Auf diesem Gebiet
ist vor allem der Hardware"=Typ \gls{fpga} zu nennen. Gegenüber den oben
genannten Beschleunigern und spezialisierten Chips haben FPGAs den Vorteil, dass
die einzelnen Chip"=Bestandteile -- also Logikfunktionen, On"=Chip"=Speicher
usw. -- weitestgehend frei verschaltbar sind. 
Lange Zeit stand dem jedoch die vergleichsweise schwierige Verwendung der
FPGAs entgegen, da diese einen Schaltungsentwurf auf der Register"=Ebene (im
Gegensatz zur Programmierung auf der algorithmischen Ebene) benötigten. Durch
das Aufkommen automatischer Synthese"=Werkzeuge, die Algorithmen in Schaltungen
umwandeln können, wurde der Entwicklungsprozess in jüngerer Zeit
aber deutlich vereinfacht.

Dadurch sind FPGAs auch für Anwendungen abseits des klassischen
Schaltungsentwurfs interessant. Insbesondere latenzkritische Anwendungen können
von FPGAs profitieren, da der Zeitpunkt der Ergebnisausgabe taktgenau
vorhersagbar ist.
% Mit der Vorhersage allein ist es nicht getan. Die FPGAs sind viel flexibler, was die Hardware-Schnittstellen angeht, was wiederum Kommunikationsstrecken zur Peripherie verkürzt.
% Das Hardware-Kompilat des Algorithmus hat idealerweise auch deutlich weniger Overhead auf FPGAs als bei den GPGPU Karten.
% FPGAs sind ggf. auch weniger störanfällig, was sie auch für Teilchenphysikexperimente interessant macht. Sie lassen sich auch besser in Experimentierhardware integrieren als GPUs.
% (Ggf. einen eigenen Absatz für GPUs vs FPGAs aus Anwendersicht)
Die Inferenz neuronaler Netzwerke oder die Verarbeitung großer
Datenströme im \gls{hpc}"=Bereich sind Beispiele für \textit{stream}"=artige
Probleme, die vom Einsatz eines FPGAs profitieren können.
% 2x profitieren

Da die Synthese einer Schaltung je nach Komplexität des Algorithmus einige
Stunden bis Tage benötigen kann, ist für das \textit{Prototyping} ratsam,
schnellere Methoden für die inkrementelle Entwicklung zu wählen -- also z.B.
Beschleuniger. Wünschenswert ist hier vor allem Portabilität zwischen den
verschiedenen Hardware"=Plattformen.

Das ist durchaus nicht unproblematisch: so waren in den Anfangsjahren der
GPU"=Programmierung nur die Produkte des Herstellers NVIDIA ohne Umwege über
Grafikschnittstellen programmierbar, erforderten dafür aber die Nutzung der
NVIDIA"=eigenen CUDA"=Schnittstelle. Diese war (und ist) jedoch nicht mit
Konkurrenzprodukten kompatibel. Für CPUs etablierte sich schnell die
OpenMP"=Schnittstelle als Mittel der Wahl für die Nutzung mehrerer Kerne.
Die Vektorisierung erfordert dagegen entweder befehlssatzspezifische
Erweiterungen, die auch innerhalb derselben Befehlssatzfamilie mitunter
inkompatibel sind, oder sehr gut optimierende Compiler. Die Synthese"=Werkzeuge
der FPGA"=Hersteller ermöglichen zwar die Nutzung der Hochsprachen, reichern
diese aber mit zahlreichen Erweiterungen an, was die Übertragung zwischen den
Plattformen erschwert.
% anfangs ggf. GPGPU Begriff einbringen
% Jahreszahlen?

Um die Entwicklung portabler Programme zu ermöglichen, wurden früh verschiedene
Ansätze entwickelt. Das Khronos"=Industriekonsortium gab wenige Jahre nach der
ersten CUDA"=Veröffentlichung die Spezifikation der OpenCL"=Schnittstelle
heraus.
% Jahresangaben
Dahinter verbarg sich die Idee, eine einheitliche Schnittstelle für den
Programmierer zu schaffen, die im Hintergrund von jedem Hardware"=Hersteller für
die eigenen Produkte implementiert und optimiert wird. Die OpenCL"=Entwicklung
wurde anfangs von zahlreichen namhaften Hard- und Software"=Herstellern
unterstützt (z.B. Apple, AMD, NVIDIA, Intel und Xilinx), flachte jedoch nach
wenigen Jahren wieder ab und konnte sich in der GPU"=Welt nicht gegen CUDA
und auf CPUs nicht gegen OpenMP durchsetzen.

Ein anderer Ansatz liegt in der Entwicklung eines abstrakten Interfaces, das im
Hintergrund auf die herstellereigenen Schnittstellen abgebildet wird. Ein
solches Interface transformiert einen vom Programmierer entwickelten Quelltext
z.B. in einen äquivalenten CUDA"=Quelltext, ohne dass der Programmierer selbst
weitere Anstrengungen in dieser Richtung unternehmen muss. So genügt ein
einfacher Austausch der Zielplattform und eine erneute Kompilierung für die
Nutzung eines anderen Hardware"=Typs. Dieses Prinzip wird von mehreren
parallelen Projekten verfolgt, wie etwa der vom Helmholtz"=Zentrum
Dresden"=Rossendorf entwickelten Alpaka"=Bibliothek oder der Kokkos"=Bibliothek,
die von einer Forschungseinrichtung des US"=amerikanischen Energieministeriums
stammt. Bisher bieten jedoch weder Alpaka noch Kokkos ein Backend für FPGAs an.

Seit einigen Jahren versucht das Khronos"=Konsortium erneut, einen Standard für
die parallele Programmierung zu etablieren. Dieser SYCL genannte Ansatz basiert
auf dem einige Jahre älteren OpenCL, bietet aber eine deutlich modernere und
fortschrittlichere C++"=Programmierschnittstelle. Der neue Standard wird unter
anderem von den FPGA"=Herstellern Xilinx und Intel vorangetrieben und wäre damit
eine interessante Backend"=Variante für die oben genannten
Abstraktionsbibliotheken.

\section{Forschungsstand}\label{einleitung:forschung}

In den letzten Jahren befassten sich mehrere Forschungsgruppen mit der
automatischen \mbox{FPGA}"=Synthese für hochparallele Programme.

Schon 2009 veröffentlichten Papakonstantinou \textit{et al.} einen Ansatz, der
die Synthese einer Schaltung auf Basis der eigentlich für NVIDIA"=GPUs gedachten
CUDA"=Schnittstelle ermöglichte. \cite[vgl.][]{papakonstantinou2009} 

Diese Arbeit konnte sich jedoch nicht langfristig durchsetzen. Seit der ersten
Veröffentlichung einer OpenCL"=Implementierung für FPGAs durch den Hersteller
Altera (heute Intel) im Jahr 2013 verlagerte sich das Interesse der Forschung
auf diese Plattform. 

Eine der ersten Arbeiten in diesem Bereich wurde 2013 von Settle, einem
damaligen Altera"=Mitarbeiter, veröffentlicht. In ihr zeigte Settle den
-- gegenüber der Entwicklung auf Registerebene -- durch eine Hochsprache wie
OpenCL zu erreichenden Produktivitätsgewinn bei gleichzeitiger Beibehaltung der
erreichten Leistung. \cite[vgl.][]{settle2013}

Fifield \textit{et al.} demonstrierten 2016 die Optimierung von
OpenCL"=Programmen für die im Vorjahr veröffentlichte
Xilinx"=OpenCL"=Implementierung. \cite[vgl.][]{fifield2016}

Duarte \textit{et al.} stellten 2018 das \textit{hls4ml}"=Projekt vor. Dabei
handelt es sich um Implementierungen neuronaler Netzwerke, die durch automatische
Synthese in Schaltungen für Xilinx"=FPGAs umgewandelt werden. In diesem Projekt
kommt allerdings nicht OpenCL zum Einsatz, sondern die Xilinx"=Erweiterungen für
die Programmiersprache C++. \cite[vgl.][]{duarte2018}
% überall: neural => neuronal

Die SYCL"=Spezifikation wurde in der Literatur in den ersten Jahren ihres
Bestehens vornehmlich als einfacher Überbau für OpenCL betrachtet. Erst in
jüngerer Zeit kam es zu eigenständigen Untersuchungen SYCLs.

Eine der frühen Arbeiten, die sich von dieser unscharfen Betrachtungsweise
abhebt, ist Žužeks Masterarbeit aus dem Jahre 2016, in deren Rahmen eine
eigenständige \gls{sycl}"=Implementierung entwickelt wurde.
\cite[vgl.][]{zuzek2016}

Wong \textit{et al.}, Mitarbeiter der Firma Codeplay -- einer der führenden
Firmen im SYCL"=Umfeld --, befassten sich ebenfalls 2016 mit den
Wechselwirkungen zwischen dem C++"=Standard und der auf diesem Standard
aufbauenden SYCL"=Spezifikation. Aufmerksamkeit wurde insbesondere den bei der
Codeplay"=SYCL"=Implementierung aufgetretenen Problemen sowie Unzulänglichkeiten
des C++"=Standards zuteil.
\cite[vgl.][]{wong2016}

Aliaga \textit{et al.} veröffentlichten 2017 eine in C++ und SYCL geschriebene
Implementierung der BLAS"=Schnittstelle -- ein Quasistandard für
Rechenoperationen der linearen Algebra --, die sie \textit{SYCL"=BLAS} nannten.
\cite[vgl.][]{aliaga2017}

Burns \textit{et al.} stellten 2019 das \textit{SYCL"=DNN}"=Projekt vor, eine in
C++ und SYCL geschriebene Bibliothek für die Beschleunigung von Operationen, die
typischerweise in neuralen Netzwerken verwendet werden. Teil der Arbeit war auch
ein Vergleich mit den konkurrierenden Bibliotheken cuDNN (NVIDIA) und
MIOpen (AMD). Im Gegensatz zu diesen herstellerspezifischen Bibliotheken soll
SYCL"=DNN auf einer Vielzahl OpenCL"=fähiger Beschleuniger lauffähig sein.
\cite[vgl.][]{burns2019}

Hinsichtlich der Verwendung von SYCL auf FPGAs ist in der Literatur -- neben
einigen untereinander recht ähnlichen Vorträgen des Xilinx"=Mitarbeiters Ronan
Keryell -- bisher nur der 2017 von Doumoulakis \textit{et al.} veröffentlichte
Artikel zu finden, der sich mit der Interoperabilität von SYCL und OpenCL auf
Xilinx"=FPGAs befasst.
\cite[vgl.][]{doumoulakis2017}

Als Backend für Abstraktionsbibliotheken wie Alpaka oder Kokkos fand SYCL bisher
keine Verwendung. Zwar veröffentlichten Copik \textit{et al.} 2017 einen Artikel
über die experimentelle Implementierung eines solchen Backends für die Kokkos
und Alpaka sehr ähnliche Bibliothek \textit{HPX.Compute}, bis heute ist dieser
Entwicklungszweig aber nicht in das Hauptprojekt aufgenommen worden.
\cite[vgl.][]{copik2017}

Im Zusammenhang mit Kokkos findet SYCL bisher nur in Form einer von Hammond
\textit{et al.} 2019 durchgeführten Studie Erwähnung. Dabei handelt es sich
jedoch um einen Vergleich der Programmiermodelle von Kokkos und SYCL und nicht
um eine Implementierung eines Kokkos"=Backends.
\cite[vgl.][]{hammond2019}

\section{Ziel der Arbeit}\label{einleitung:ziel}

In dieser Arbeit wird der SYCL"=Standard hinsichtlich der verfügbaren
Implementierungen und deren Nutzbarkeit untersucht. Dies geschieht vor allem im
Hinblick auf FPGAs, die durch ihre frei veränderbare Hardware"=Konfiguration in
Verbindung mit einer modernen C++"=Programmierschnittstelle eine
vielversprechende Hardware"=Plattform darstellen. Dabei wird ein
experimentelles SYCL"=Backend für die Alpaka-Bibliothek implementiert. Die
während des Entwicklungsprozesses aufgetretenen Schwierigkeiten und
Unzulänglichkeiten werden analysiert, sowie erste Einschätzungen der
Leistungsfähigkeit vorgenommen.

\chapter{FPGAs als Beschleuniger}

Konzeption und Aufbau der \gls{fpga}s sowie der zugehörige Entwicklungsprozess
werden in diesem Kapitel geschildert. Abschließend werden die auf \gls{fpga}s
und \gls{gpu}s zu findenden Parallelitätskonzepte miteinander verglichen.

\section{Überblick}

Für das Verständnis der Funktionsweise eines \gls{fpga}s ist es notwendig, die
zugrunde liegenden Konzepte in Abgrenzung zu herkömmlicher Hardware
darzustellen. Dieser Abschnitt definiert zunächst den \gls{fpga}-Begriff und
erläutert im Anschluss daran den Aufbau moderner \gls{fpga}-Architekturen sowie
traditionelle und neuartige Nutzungsmöglichkeiten dieses Hardware-Typus.

\subsection{Definition}

\textit{Field-programmable gate arrays} sind, wie der Name andeutet,
konzeptionell mit den \textit{gate arrays} verwandt.

Die klassischen \textit{gate arrays} sind eine Untergruppe der integrierten
Schaltkreise (engl. \textit{integrated circuits}, IC) und gehören zur Gattung
der anwendungsspezifischen ICs (engl. \textit{application specific IC}, ASIC).
Unter ASICs versteht man jene Chips, die bereits bei der Herstellung mit einer
kundenspezifischen Schaltung versehen werden. Innerhalb dieser Kategorie gehören
\textit{gate arrays} zu den teil-vorgefertigten ASICs (engl.
\textit{semi-custom ASIC}). Diese werden zunächst in großer Menge mit demselben
technischen Grundgerüst produziert und erst in einem späteren
Herstellungsschritt in kleineren Mengen mit kundenspezifischen Schaltungen
versehen. Im Gegensatz zu ASICs, die von Anfang an nach Kundenwunsch hergestellt
wurden (engl. \textit{full-custom ASIC}), lässt sich so eine Reduktion der
Produktionskosten erreichen. \cite[vgl.][123]{kesel2013}

Allerdings haben \textit{gate arrays} den Nachteil, dass sie nur vom Hersteller
programmiert werden können. Eine Anpassung der Schaltung im Feld (engl.
\textit{field-programmable}) ist damit nicht möglich. Mit \gls{fpga}s wurde
dieses Problem in den 1980er Jahren gelöst, indem man aus Gattern
(engl. \textit{gates}) bestehende Logikzellen von geringer Komplexität in einer
regelmäßigen Feldstruktur (engl. \textit{array}) anordnete und über
programmierbare Verdrahtungen miteinander verband.
\cite[vgl.][208]{kesel2013} 

Mittlerweile gibt es viele verschiedene \gls{fpga}-Varianten, die jedoch einige
Gemeinsamkeiten aufweisen. \gls{fpga}s bestehen stets aus einem Feld aus
Blockzellen, die so konfiguriert werden, dass sie eine bestimmte Funktion
ausführen. Diese Blockzellen integrieren durch ein dichtes Verbindungsnetz
Logikgatter und Speicher über ein dichtes Verbindungsnetz. Dabei lassen sich
vier zentrale Strukturen unterscheiden:
\begin{itemize}
    \item konfigurierbare Logikblöcke,
    \item programmierbare Verbindungen,
    \item Puffer für die Ein- und Ausgabe (E/A) und
    \item weitere Elemente (Speicher, arithmetische Einheiten, Taktnetzwerke,
          usw.).
\end{itemize}
In Abbildung~\ref{fpga:definition:aufbau} ist eine abstrakte \gls{fpga}-Struktur
dargestellt, die aus Logikblöcken, Verbindungen, E/A-Puffern und speziellen
Speicher- und Multiplizierer-Blöcken aufgebaut ist.
\cite[vgl.][10-13f.]{hawkins2010}

\begin{figure}[htb]
    \centering
    \begin{tikzpicture}
        % (0,0) ist unten links

        % untere E/A-Reihe
        \draw [fill = HKS41!60]
                (0.0, -1.875) rectangle (1.5, -1.125)
                node[pos = 0.5, text = white, align = center] {\small E/A};
        \draw [fill = HKS41!60]
                (2.625, -1.875) rectangle (4.125, -1.125)
                node[pos = 0.5, text = white, align = center] {\small E/A};
        \draw [fill = HKS41!60]
                (5.25, -1.875) rectangle (6.75, -1.125)
                node[pos = 0.5, text = white, align = center] {\small E/A};
        \draw [fill = HKS41!60]
                (7.875, -1.875) rectangle (9.375, -1.125)
                node[pos = 0.5, text = white, align = center] {\small E/A};

        % Reihe 1
        \draw [fill = HKS41!60]
                (-1.875, 0.0) rectangle (-1.125, 1.5)
                node[pos = 0.5, text = white, align = center] {\small E/A};
        \draw [fill = HKS41!80]
                (0.0, 0.0) rectangle (1.5, 1.5)
                node[pos = 0.5, text = white, align = center] {\small Logik};
        \draw [fill = HKS41!40]
                (2.625, 0.0) rectangle (4.125, 1.5)
                node[pos = 0.5, text = white, align = center] {\small Speicher};
        \draw [fill = HKS41!80]
                (5.25, 0.0) rectangle (6.75, 1.5)
                node[pos = 0.5, text = white, align = center] {\small Logik};
        \draw [fill = HKS41]
                (7.875, 0.0) rectangle (9.375, 1.5)
                node[pos = 0.5, text = white, align = center] {\small Multipli- \\ zierer};
        \draw [fill = HKS41!60]
                (10.5, 0.0) rectangle (11.25, 1.5)
                node[pos = 0.5, text = white, align = center] {\small E/A};

        % Reihe 2
        \draw [fill = HKS41!60]
                (-1.875, 2.625) rectangle (-1.125, 4.125)
                node[pos = 0.5, text = white, align = center] {\small E/A};
        \draw [fill = HKS41!80]
                (0.0, 2.625) rectangle (1.5, 4.125)
                node[pos = 0.5, text = white, align = center] {\small Logik};
        \draw [fill = HKS41!40]
                (2.625, 2.625) rectangle (4.125, 4.125)
                node[pos = 0.5, text = white, align = center] {\small Speicher};
        \draw [fill = HKS41!80]
                (5.25, 2.625) rectangle (6.75, 4.125)
                node[pos = 0.5, text = white, align = center] {\small Logik};
        \draw [fill = HKS41]
                (7.875, 2.625) rectangle (9.375, 4.125)
                node[pos = 0.5, text = white, align = center] {\small Multipli- \\ zierer};
        \draw [fill = HKS41!60]
                (10.5, 2.625) rectangle (11.25, 4.125)
                node[pos = 0.5, text = white, align = center] {\small E/A};

        % Reihe 3
        \draw [fill = HKS41!60]
                (-1.875, 5.25) rectangle (-1.125, 6.75)
                node[pos = 0.5, text = white, align = center] {\small E/A};
        \draw [fill = HKS41!80]
                (0.0, 5.25) rectangle (1.5, 6.75)
                node[pos = 0.5, text = white, align = center] {\small Logik};
        \draw [fill = HKS41!40]
                (2.625, 5.25) rectangle (4.125, 6.75)
                node[pos = 0.5, text = white, align = center] {\small Speicher};
        \draw [fill = HKS41!80]
                (5.25, 5.25) rectangle (6.75, 6.75)
                node[pos = 0.5, text = white, align = center] {\small Logik};
        \draw [fill = HKS41]
                (7.875, 5.25) rectangle (9.375, 6.75)
                node[pos = 0.5, text = white, align = center] {\small Multipli- \\ zierer};
        \draw [fill = HKS41!60]
                (10.5, 5.25) rectangle (11.25, 6.75)
                node[pos = 0.5, text = white, align = center] {\small E/A};

        % Reihe 3
        \draw [fill = HKS41!60]
                (-1.875, 7.875) rectangle (-1.125, 9.375)
                node[pos = 0.5, text = white, align = center] {\small E/A};
        \draw [fill = HKS41!80]
                (0.0, 7.875) rectangle (1.5, 9.375)
                node[pos = 0.5, text = white, align = center] {\small Logik};
        \draw [fill = HKS41!40]
                (2.625, 7.875) rectangle (4.125, 9.375)
                node[pos = 0.5, text = white, align = center] {\small Speicher};
        \draw [fill = HKS41!80]
                (5.25, 7.875) rectangle (6.75, 9.375)
                node[pos = 0.5, text = white, align = center] {\small Logik};
        \draw [fill = HKS41]
                (7.875, 7.875) rectangle (9.375, 9.375)
                node[pos = 0.5, text = white, align = center] {\small Multipli- \\ zierer};
        \draw [fill = HKS41!60]
                (10.5, 7.875) rectangle (11.25, 9.375)
                node[pos = 0.5, text = white, align = center] {\small E/A};

        % obere E/A-Reihe
        \draw [fill = HKS41!60]
                (0.0, 10.5) rectangle (1.5, 11.25)
                node[pos = 0.5, text = white, align = center] {\small E/A};
        \draw [fill = HKS41!60]
                (2.625, 10.5) rectangle (4.125, 11.25)
                node[pos = 0.5, text = white, align = center] {\small E/A};
        \draw [fill = HKS41!60]
                (5.25, 10.5) rectangle (6.75, 11.25)
                node[pos = 0.5, text = white, align = center] {\small E/A};
        \draw [fill = HKS41!60]
                (7.875, 10.5) rectangle (9.375, 11.25)
                node[pos = 0.5, text = white, align = center] {\small E/A};

        % Verbindungen - Spalte 1
        \draw [color = HKS92!90, line width = 5mm]
                (0.75, -1.125) -- (0.75, 0.0);
        \draw [color = HKS92!90, line width = 5mm]
                (0.75, 1.5) -- (0.75, 2.625);
        \draw [color = HKS92!90, line width = 5mm]
                (0.75, 4.125) -- (0.75, 5.25);
        \draw [color = HKS92!90, line width = 5mm]
                (0.75, 6.75) -- (0.75, 7.875);
        \draw [color = HKS92!90, line width = 5mm]
                (0.75, 9.375) -- (0.75, 10.5);

        % Verbindungen - Spalte 2
        \draw [color = HKS92!90, line width = 5mm]
                (3.375, -1.125) -- (3.375, 0.0);
        \draw [color = HKS92!90, line width = 5mm]
                (3.375, 1.5) -- (3.375, 2.625);
        \draw [color = HKS92!90, line width = 5mm]
                (3.375, 4.125) -- (3.375, 5.25);
        \draw [color = HKS92!90, line width = 5mm]
                (3.375, 6.75) -- (3.375, 7.875);
        \draw [color = HKS92!90, line width = 5mm]
                (3.375, 9.375) -- (3.375, 10.5);

        % Verbindungen - Spalte 3
        \draw [color = HKS92!90, line width = 5mm]
                (6.0, -1.125) -- (6.0, 0.0);
        \draw [color = HKS92!90, line width = 5mm]
                (6.0, 1.5) -- (6.0, 2.625);
        \draw [color = HKS92!90, line width = 5mm]
                (6.0, 4.125) -- (6.0, 5.25);
        \draw [color = HKS92!90, line width = 5mm]
                (6.0, 6.75) -- (6.0, 7.875);
        \draw [color = HKS92!90, line width = 5mm]
                (6.0, 9.375) -- (6.0, 10.5);

        % Verbindungen - Spalte 4
        \draw [color = HKS92!90, line width = 5mm]
                (8.625, -1.125) -- (8.625, 0.0);
        \draw [color = HKS92!90, line width = 5mm]
                (8.625, 1.5) -- (8.625, 2.625);
        \draw [color = HKS92!90, line width = 5mm]
                (8.625, 4.125) -- (8.625, 5.25);
        \draw [color = HKS92!90, line width = 5mm]
                (8.625, 6.75) -- (8.625, 7.875);
        \draw [color = HKS92!90, line width = 5mm]
                (8.625, 9.375) -- (8.625, 10.5);

        % Verbindungen - Reihe 1
        \draw [color = HKS92!90, line width = 5mm]
                (-1.125, 0.75) -- (0.0, 0.75);
        \draw [color = HKS92!90, line width = 5mm]
                (1.5, 0.75) -- (2.625, 0.75);
        \draw [color = HKS92!90, line width = 5mm]
                (4.125, 0.75) -- (5.25, 0.75);
        \draw [color = HKS92!90, line width = 5mm]
                (6.75, 0.75) -- (7.875, 0.75);
        \draw [color = HKS92!90, line width = 5mm]
                (9.375, 0.75) -- (10.5, 0.75);

        % Verbindungen - Reihe 2
        \draw [color = HKS92!90, line width = 5mm]
                (-1.125, 3.375) -- (0.0, 3.375);
        \draw [color = HKS92!90, line width = 5mm]
                (1.5, 3.375) -- (2.625, 3.375);
        \draw [color = HKS92!90, line width = 5mm]
                (4.125, 3.375) -- (5.25, 3.375);
        \draw [color = HKS92!90, line width = 5mm]
                (6.75, 3.375) -- (7.875, 3.375);
        \draw [color = HKS92!90, line width = 5mm]
                (9.375, 3.375) -- (10.5, 3.375);

        % Verbindungen - Reihe 3
        \draw [color = HKS92!90, line width = 5mm]
                (-1.125, 6.0) -- (0.0, 6.0);
        \draw [color = HKS92!90, line width = 5mm]
                (1.5, 6.0) -- (2.625, 6.0);
        \draw [color = HKS92!90, line width = 5mm]
                (4.125, 6.0) -- (5.25, 6.0);
        \draw [color = HKS92!90, line width = 5mm]
                (6.75, 6.0) -- (7.875, 6.0);
        \draw [color = HKS92!90, line width = 5mm]
                (9.375, 6.0) -- (10.5, 6.0);

        % Verbindungen - Reihe 4
        \draw [color = HKS92!90, line width = 5mm]
                (-1.125, 8.625) -- (0.0, 8.625);
        \draw [color = HKS92!90, line width = 5mm]
                (1.5, 8.625) -- (2.625, 8.625);
        \draw [color = HKS92!90, line width = 5mm]
                (4.125, 8.625) -- (5.25, 8.625);
        \draw [color = HKS92!90, line width = 5mm]
                (6.75, 8.625) -- (7.875, 8.625);
        \draw [color = HKS92!90, line width = 5mm]
                (9.375, 8.625) -- (10.5, 8.625);

        % Verbindungen - Zwischenspalten
        \draw [color = HKS92!90, line width = 5mm]
                (-0.5625, -0.8125) -- (-0.5625, 10.1875);
        \draw [color = HKS92!90, line width = 5mm]
                (2.0625, -0.5625) -- (2.0625, 9.9375);
        \draw [color = HKS92!90, line width = 5mm]
                (4.6875, -0.5625) -- (4.6875, 9.9375);
        \draw [color = HKS92!90, line width = 5mm]
                (7.3125, -0.5625) -- (7.3125, 9.9375);
        \draw [color = HKS92!90, line width = 5mm]
                (9.9375, -0.8125) -- (9.9375, 10.1875);

        % Verbindungen - Zwischenzeilen
        \draw [color = HKS92!90, line width = 5mm]
                (-0.5625, -0.5625) -- (9.9375, -0.5625);
        \draw [color = HKS92!90, line width = 5mm]
                (-0.5625, 2.0625) -- (9.9375, 2.0625);
        \draw [color = HKS92!90, line width = 5mm]
                (-0.5625, 4.6875) -- (9.9375, 4.6875);
        \draw [color = HKS92!90, line width = 5mm]
                (-0.5625, 7.3125) -- (9.9375, 7.3125);
        \draw [color = HKS92!90, line width = 5mm]
                (-0.5625, 9.9375) -- (9.9375, 9.9375);
    \end{tikzpicture}
    \caption{abstrakter FPGA-Aufbau \cite[nach][S. 10-14]{hawkins2010}}
    \label{fpga:definition:aufbau}
\end{figure}

\subsection{Aufbau moderner FPGAs}

Am Beispiel der Virtex-UltraScale+-Architektur der Firma Xilinxrs soll der
Aufbau eines modernen \gls{fpga} verdeutlicht werden. \gls{fpga}s dieser
Architektur bestehen aus sieben fundamentalen programmierbaren Elementen:
\begin{itemize}
    \item Konfigurierbare Logikblöcke (engl. \textit{configurable logic blocks},
          CLB) bestehen aus Lookup-Tabellen (LUT), die zur Generierung von
          Logikfunktionen verwendet werden können, sowie Flipflops, die als
          Speicher dienen.
    \item I/O
    \item Block RAM
    \item UltraRAM
    \item DRAM
    \item DSPs
    \item Transceiver
\end{itemize}

Das \gls{fpga} mit der Modellbezeichnung \textit{XCU200} ist neben diversen
Anschlüssen (zum Beispiel USB, PCI Express und Ethernet) und dem
Beschleuniger-Hauptspeicher auf der Platine verbaut \cite[vgl.][2]{alveo2019}.
Es ist in drei Abschnitte unterteilt, die von Xilinx als \gls{slr} bezeichnet
werden. Gemeinsam bilden die \gls{slr}s drei dynamische Regionen sowie eine
statische Region, welche mit dem Hauptspeicher (DDR) des Beschleunigers
verbunden sind (siehe Abbildung~\ref{fpga:aufbau:alveoslr}). Die dynamischen
Regionen lassen sich vom Benutzer programmieren, während die statische Region
der Laufzeitumgebung des \gls{fpga} vorbehalten ist \cite[vgl.][4]{alveo2019}.
Die \gls{slr}s verfügen über verschiedene Ressourcen in unterschiedlicher
Anzahl (siehe Tabelle~\ref{fpga:aufbau:ressourcen}), die in den folgenden
Abschnitten erläutert werden.

\begin{figure}[htb]
    \centering
    \begin{tikzpicture}
        % unten
        \draw (0.0, 0.0) rectangle (7.5, 2.5)
            node [pos = 0.5] {dynamische Region}
            node [pos = 0.1] {SLR0};

        % Mitte
        \draw [draw = none] (0.0, 2.5) rectangle (7.5, 5)
            node [pos = 0.1] {SLR1};
        \draw (0.0, 2.5) rectangle (3.75, 5)
            node [pos = 0.5] {dynamische Region};
        \draw [fill = lightgray] (3.75, 2.5) rectangle (7.5, 5)
            node [pos = 0.5] {statische Region};

        % oben
        \draw (0.0, 5) rectangle (7.5, 7.5)
            node [pos = 0.5] {dynamische Region}
            node [pos = 0.1] {SLR2};
    \end{tikzpicture}
    \caption{Aufbau eines XCU200-FPGAs \cite[nach][5]{alveo2019}}
    \label{fpga:aufbau:alveoslr}
\end{figure}

\begin{table}[htb]
    \centering
    \begin{tabulary}{\textwidth}{@{}LCCCC@{}}
        \toprule
        \textbf{Ressource} & \textbf{Gesamt} & \textbf{SLR0} & \textbf{SLR1}
            & \textbf{SLR2} \tabularnewline\midrule
        \textit{Configurable logic blocks (CLBs)}
            \tablefootnote{Xilinx macht
            in seinen Datenblättern keine Angaben zur Anzahl der CLBs auf
            dem XCU200-FPGA. Die Zahl der CLBs wurde daher rechnerisch aus
            der Zahl der \textit{look-up tables} (LUT) bestimmt, die in den
            Datenblättern angegeben wird. Die UltraScale-Architektur bildet
            aus jeweils acht LUTs einen CLB
            \cite[siehe][38]{ultrascale2019}.} & \num{111500} &
            \num{45625} & \num{20250} & \num{45625}\tabularnewline
        Register & \num{1831000} & \num{746000} & \num{339000} & 
            \num{746000}\tabularnewline
        Block-RAM (\SI{36}{\kibi\byte}) & \num{1766} & \num{695} & 
            \num{376} & \num{695}\tabularnewline
        UltraRAM (\SI{288}{\kibi\byte}) & \num{800} & \num{320} &
            \num{160} & \num{320}\tabularnewline
        DSP-Schichten & \num{5867} & \num{2275} & \num{1317} &
            \num{2275}\tabularnewline\bottomrule
    \end{tabulary}
    \caption{Ressourcen der dynamischen Regionen eines XCU200-FPGAs
             \cite[siehe][5]{alveo2019}}
    \label{fpga:aufbau:ressourcen}
\end{table}


\cite{ultrascale2019}

Alveo U200
\begin{itemize}
    \item drei \textit{super logic regions} (SLRs)
    \item entspricht drei dynamischen Regionen (vom Benutzer programmierbar) und
          einer statischen Region (deployment shell für Konfiguration)
    \item SLR0, SLR2: 365K LUTs, 746K Register, 695 36KB block RAM, 320 288Kb UltraRAM, 2275 DSPs
    \item SLR1: 162K LUTs, 339K Register, 376KB block RAM, 160 288Kb UltraRAM, 1317 DSPs
\end{itemize}

LUT
\begin{itemize}
    \item 45625 Configurable Logic Blocks (CLBs) à 8 LUTs und 16 flip-flops (SLR0, SLR2) bzw. 20250 CLBs (SLR1)
    \item demnach 730000 flip-flops (SLR0, SLR2) bzw. 324000 flip-flops (SLR1)
    \item CLB enthält Logik für arithmetischen Übertrag und Multiplexer, dadurch komplexere Logikfunktionen möglich, sowie Kontrollsignale
    \item CLB sind über verschiedene Verbindungspfade verbunden, die 1, 2, 4, 5, 12 oder 16 CLBs miteinander verbinden
\end{itemize}

Register -- TODO

block RAM
\begin{itemize}
    \item jeder 36Kb Block hat zwei unabhängige Ports, die sich die Daten im Block teilen
    \item kann als 1 36Kb Block oder als 2 18Kb Blöcke konfiguriert werden
    \item Ports haben programmierbare Datenbreite: 32K x 1, 16K x 2, 8K x 4, 4K x 9 (oder 8), 2K x 18 (oder 16), 1K x 36 (oder 32), 512 x 72 (oder 64)
    \item unterschiedliche Datenbreiten für die Ports möglich
    \item ECC unterstützt
    \item als FIFO konfigurierbar
    \item vertikal nebeneinander liegende Blocks lassen sich verbinden, wodurch größere Speicherbereiche entstehen
\end{itemize}

UltraRAM
\begin{itemize}
    \item jeder 288Kb UltraRAM hat zwei unabhängige Ports
    \item feste Datenbreite: 4K x 72
    \item UltraRAM-Spalten lassen sich verbinden, so sind bis zu 100MB möglich
\end{itemize}

DSP
\begin{itemize}
    \item spezielle Funktionseinheiten, die für Signalverarbeitung optimiert sind
    \item enthalten 27 x 18 bit Multiplizierern (Zweierkomplement) und einen 48-bit Akkumulator
    \item Multiplizierer kann überbrückt werden
    \item zwei 48-bit Eingaben können eine SIMD-Einheit füttern (2x 24-bit Addition / Subtraktion / Akkumulation oder 4x 12-bit Addition / ...) oder aber eine Logik-Einheit, die bis zu zehn verschiedene Logikfunktionen für beide Operanden generieren kann
    \item enthält Voraddierer (symmetrische Filter)
    \item 96-bit XOR-Funktion, programmierbar auf 12, 24, 48 oder 96 bit
    \item 48-bit Muster-Detektor, kann mit der Logikeinheit verbunden werden, um Logikfunktionen mit 96-bit Breite zu generieren
    \item eignet sich für Pipelining und Anwendungen außerhalb der DSP
\end{itemize}

\subsection{Anwendungsfälle}

Gegenüber ASICs bieten \gls{fpga}s einige Vorteile. Da sich Schaltungen ohne
einen Produktionsprozess schneller in Hardware abbilden lassen, eignen sich
\gls{fpga}s für die Entwicklung neuer Schaltungen durch die Methode des
\textit{rapid prototyping} und damit für eine schnellere Markteinführung. Durch
die einfache Neuprogrammierung lassen sich Fehler außerdem während des
Entwicklungsprozesses sowie während des Lebenszyklus des Produkts deutlich
einfacher beheben, als dies bei ASICs der Fall wäre.
\cite[vgl.][10-1]{hawkins2010}

Dadurch eignen sich \gls{fpga}s sehr gut für den Einsatz als Schaltkreise, die
in kleiner bis mittlerer Menge produziert werden sollen, weil die finanzielle
Einstiegshürde deutlich geringer als bei ASICs ist. Umgekehrt sind ASICs bei
hohen Produktionsvolumen überlegen, da die Kosten pro Chip geringer sind.
\cite[vgl.][10-2]{hawkins2010}

In jüngerer Zeit wurden \gls{fpga}s auch außerhalb des klassischen
Schaltkreisentwurfs eingesetzt. So setzt die Firma Microsoft beispielsweise
\gls{fpga}s des Herstellers Intel für die Inferenz tiefer neuraler Netzwerke
\cite[vgl.][]{fowers2018, chung2018} sowie als besonders schnelle
Netzwerkkarten ein \cite[vgl.][]{firestone2018}.

\section{Entwicklungsprozess}

\subsection{Hardware-Beschreibungssprachen}

\subsection{High-Level-Synthese}

\section{Parallelität}

\chapter{Der SYCL-Standard}\label{sycl}

\section{Einführung}\label{sycl:intro}

\section{Grundlagen}\label{sycl:grundlagen}

\section{Weiterführende Konzepte}\label{sycl:konzepte}

\section{Implementierungen}\label{sycl:implementierungen}

\chapter{Die Alpaka-Bibliothek}\label{alpaka}

Dieses Kapitel führt in die Alpaka"=Bibliothek ein. Wie im vorherigen
SYCL"=Kapitel wird der grundlegende Aufbau eines Alpaka"=Programms anhand des
AXPY"=Beispiels dargestellt. Ein weiterer Abschnitt ist den darauf aufbauenden,
erweiterten Konzepten, wie etwa der Hardware"=Abstraktion, gewidmet.

\section{Überblick}\label{alpaka:ueberblick}

Alpaka (Eigenschreibweise: \textit{alpaka}) steht für
\textit{Abstraction Library for Parallel Kernel Acceleration} und wurde
ursprünglich von Benjamin Worpitz im Rahmen seiner Masterarbeit entwickelt
\cite[vgl.][]{worpitz2015}. Mittlerweile wird die Entwicklung durch
die Gruppe \textit{Computergestützte Strahlenphysik} des
\textit{Instituts für Strahlenphysik} am
\textit{Helmholtz"=Zentrum Dresden"=Rossendorf} fortgeführt.

Die Alpaka"=Bibliothek definiert eine abstrakte C++"=Schnittstelle, mit deren
Hilfe parallele Programme geschrieben werden können. Im Hintergrund wird Alpaka
auf hersteller- oder hardware"=spezifische Schnittstellen, wie CUDA oder OpenMP,
-- im Folgenden als \textit{Backend} bezeichnet -- abgebildet. Alpaka ist somit
ein einheitliches Paket, das die abstrakte Schnittstelle nach außen und die
konkrete Implementierung vereinigt. Damit unterscheidet sich die Bibliothek von
ähnlichen Ansätzen wie OpenCL oder SYCL, die ebenfalls eine abstrakte
Schnittstelle definieren, die Implementierung jedoch den Hardware- und
Software"=Herstellern überlassen.

Wie bei SYCL sind die Quelltexte für \textit{Host} und \textit{Device} nicht
voneinander getrennt. Die Abbildung auf ein oder mehrere Backends erfolgt zur
Compile"=Zeit durch Template"=Metaprogrammierung, wodurch ein
Abstraktions"=Overhead zur Laufzeit vermieden wird.

\subsection{AXPY und Alpaka}\label{alpaka:ueberblick:axpy}

\subsubsection{Beschleunigerwahl und Befehlswarteschlange}
\label{alpaka:ueberblick:axpy:queue}

\subsubsection{Speicherreservierung und -initialisierung}
\label{alpaka:ueberblick:axpy:buffer}

\subsubsection{Kerneldefinition und -ausführung}
\label{alpaka:ueberblick:axpy:kernel}

\subsubsection{Synchronisierung}
\label{alpaka:ueberblick:axpy:sync}

\subsubsection{Zusammenfassung}
\label{alpaka:ueberblick:axpy:zusammenfassung}

\section{Weiterführende Konzepte}\label{alpaka:konzepte}

\subsection{Hardware"=Abstraktion}

\subsection{Abhängigkeiten zwischen Kerneln}
\label{alpaka:konzepte:abhaengigkeiten}

\subsection{Fehlerbehandlung}

\subsection{Profiling}

\chapter{Implementierung des SYCL-Backends der Alpaka-Bibliothek}
\label{implementierung}

Mit Ausnahme der in den folgenden Abschnitten aufgeführten Besonderheiten
(Abschnitt~\ref{implementierung:besonderheiten}) bzw.\ Probleme
(Abschnitt~\ref{implementierung:probleme}) konnte die Implementierung des
SYCL-Backends für Alpaka recht einfach durchgeführt werden. Als Vorlage für
dessen Aufbau dienten die bereits vorhandenen Backends, wobei hier besonders
das CUDA-Backend hervorzuheben ist.

\section{Besonderheiten des SYCL-Backends}
\label{implementierung:besonderheiten}

\subsection{Beschleuniger-Auswahl}

TODO

\subsection{Zeiger und \textit{accessors}}
\label{implementierung:besonderheiten:zeiger}

Die Übergabe von Datenfeldern an \gls{device}- und \gls{kernel}-Funktionen
erfolgt in Alpaka über die von C, älterem C++ und CUDA bekannten Zeiger. Pro
Datenfeld erhält die Funktion üblicherweise einen Zeiger, z.B. vom Typ
\texttt{float*}, und einen ganzzahligen Parameter (meist vom Typ
\texttt{size\_t}), der die Länge des Speicherbereichs angibt.

\begin{itemize}
    \item Alpaka: Will überall reine Zeiger (Kernel-Code)
    \item SYCL: Will überall \texttt{accessor}, alternativ \texttt{multi\_ptr}
    \item Lösung: \texttt{multi\_ptr} lässt sich implizit zu reinem Zeiger casten
    \item Hinweis: SYCL-Spezifikation gibt fälschlicherweise an, dass
          \texttt{accessor} implizit castbar ist.
\end{itemize}

\subsection{Block-Synchronisierung}
\label{implementierung:besonderheiten:synchronisierung}

\begin{itemize}
    \item Alpaka will überall einen Dimensions-Parameter, nur nicht bei der
          Block-Synchronisierung.
    \item SYCL braucht den Dimensionsparameter bei der Block-Synchronisierung.
    \item Lösung: Erfolgt, funktioniert über Templates.
\end{itemize}

\section{Probleme}\label{implementierung:probleme}

Während der Implementierung traten einige Probleme auf, die sich vornehmlich
auf gravierende konzeptionelle Unterschiede zwischen Alpaka und SYCL
zurückführen lassen. Diese werden in den folgenden Abschnitten näher
beschrieben.

\subsection{Event-System}\label{implementierung:probleme:events}

Alpaka übernimmt viele Konzepte des CUDA-API, darunter auch das Event-System.
Durch CUDA-\textit{events} wird dem Programmierer eine weitere
Synchronisationsmöglichkeit eröffnet. Diese können in einem \textit{stream} vor
oder nach asynchronen Operationen -- wie etwa Kopiervorgängen oder dem Starten
eines Kernels -- einsortiert werden. Der Programmierer kann dann später oder
parallel in einem anderen \textit{stream} abfragen, ob das jeweilige
\textit{event} bereits erreicht wurde und gegebenenfalls darauf warten. Darüber
hinaus ermöglichen \textit{events} ein simples Profiling, da z.B. die Zeitspanne
zwischen verschiedenen \textit{events} gemessen werden kann.

SYCL kennt ebenfalls ein \textit{event}-Konzept, das sich jedoch von CUDAs bzw.
Alpakas System unterscheidet. Auch SYCL-\textit{events} lassen sich für
einfaches Profiling nutzen, ermöglichen dem Programmierer jedoch nicht, eine
weitere SYCL-\textit{queue} (dem Gegenstück zu CUDA-\textit{streams}) auf ein
bestimmtes \textit{event} zu warten.

Dies ist für die Implementierung der Alpaka-\textit{events} ein Problem, da hier
CUDAs Verhalten simuliert wird. Zur Zeit ist der Befehl
\texttt{alpaka::wait::waiterWaitFor()} für die \textit{event}-basierte
Synchronisation in Alpaka nicht implementiert, wenn der \textit{waiter} eine
Alpaka-\textit{queue} ist.

\subsection{Geteilter Speicher}\label{implementierung:probleme:shared}

Einige Plattformen der parallelen Programmierung, wie etwa CUDA und OpenCL,
kennen das Konzept eines Speichers auf Multiprozessor-Ebene, der ungefähr
einem programmierbaren L1-Cache entspricht. Dieser Speicher nennt sich im
CUDA-Umfeld \textit{shared memory}, während er bei OpenCL (und SYCL)
\textit{local memory} heißt. Alpaka übernimmt für dieses Speicherkonzept die
CUDA-Terminologie. \textit{Shared memory} steht allen \textit{threads} auf
der \textit{block}-Ebene zur Verfügung und bietet deutlich schnellere
Zugriffszeiten als der globale Speicher.

Es gibt zwei mögliche Arten, Speicher dieses Typs zu reservieren:
\textit{dynamisch}, das heißt außerhalb des Kernel-Codes und zur Laufzeit, sowie
\textit{statisch}, das heißt innerhalb des Kernel-Codes und mit einer zur
Compile-Zeit feststehenden Größe.

Alpaka stellt für beide Varianten eine Schnittstelle bereit. Für den dynamischen
Fall muss der Programmierer das \textit{type trait}
\texttt{alpaka::kernel::traits::BlockSharedMemDynSizeBytes} auf der Host-Seite
für seine Anwendung implementieren. Innerhalb des Kernels kann er dann über die
Alpaka-Funktion \texttt{alpaka::block::shared::dyn::getMem()} auf den Zeiger
zum so reservierten geteilten Speicher zugreifen. Dieser Fall lässt sich auch
für SYCL implementieren, indem man die in
Abschnitt~\ref{implementierung:besonderheiten:zeiger} vorgestellten
Umwandlungen von \textit{accessor}-Typen in \textit{Zeiger} anwendet.

Für den statischen Fall existiert die Funktion
\texttt{alpaka::block::shared::st::allocVar()}, die innerhalb des Kernels
aufgerufen wird und eine beliebige Variable im geteilten Speicher ablegt. Diese
Funktion kann für SYCL nicht implementiert werden, da die SYCL-Spezifikation
dies (im Gegensatz zu OpenCL) bis auf einen bestimmten Sonderfall
\cite[siehe][Abschnitt 4.8.5.3]{sycl2019} nicht vorsieht.

Diese Einschränkung erklärt sich dadurch, dass SYCL mit dem Anspruch entworfen
wurde, von jedem beliebigen modernen C++-Compiler übersetzt werden zu können,
auch wenn dieser keine Unterstützung für OpenCL- und/oder SYCL-Konzepte mit sich
bringt. In diesem Fall generiert der Compiler wie bei jedem anderen C++-Programm
normalen CPU-Maschinencode. Der Umfang des statischen geteilten Speichers steht
zwar bereits zur Compile-Zeit fest und kann daher schon vor dem Aufruf des
Kernels alloziert werden. Der Zeiger auf diesen Speicherbereich kann durch einen
C++-Compiler ohne SYCL-Unterstützung dem betreffenden Kernel vor dessen
Ausführung jedoch gar nicht zugeordnet werden, da für diese Funktion noch kein
Stapelrahmen existiert. (vgl. die GitHub"=Diskussion mit Mitgliedern des
SYCL"=Spezifikationskomitees im
Anhang~\ref{anhang:diskussionen:syclspec:staticshared})

\subsection{Atomare Funktionen}\label{implementierung:probleme:atomics}

\begin{itemize}
    \item SYCL kann anhand eines rohen Zeigers nicht ableiten, welcher
          \texttt{multi\_ptr}-Typ verwendet werden soll.
    \item \texttt{multi\_ptr} wird für SYCL-Atomics benötigt.
    \item Alpaka gibt uns nur rohe Zeiger.
    \item Lösung: Keine. Im Moment funktionieren Atomics nur für den globalen
          Addressraum. Begründung SYCL-Kommittee einfügen.
\end{itemize}

\subsection{FPGA-Erweiterungen}\label{implementierung:probleme:fpga}

Wie SYCL-FPGA-Erweiterungen nutzen?

\chapter{Ergebnisse}\label{ergebnisse}

\section{Nutzbarkeit der SYCL-Implementierungen}
\label{ergebnisse:nutzbarkeit}

Im Folgenden wird die Nutzbarkeit von drei öffentlich verfügbaren
SYCL"=Implementierungen dargelegt. Dabei handelt es sich um ComputeCpp sowie
die Implementierungen der Firmen Intel und Xilinx. Die im Kapitel~\ref{sycl}
erwähnten Implementierungen hipSYCL und sycl-gtx wurden während dieser Arbeit
nicht in Betracht gezogen, da ihnen zu viele kritische Features fehlen und sie
daher nicht für eine Nutzung von Alpaka geeignet sind.

\subsection{ComputeCpp}

Die folgenden Ausführungen zu ComputeCpp beziehen sich auf die frei verfügbare
\textit{Community Edition} und alle Versionen bis einschließlich Version 1.1.5.

Von den derzeit der Öffentlichkeit zugänglichen SYCL"=Implementierungen ist
ComputeCpp die einzige, die nicht quelloffen ist. Sie wird vom schottischen
Unternehmen Codeplay entwickelt, welches ebenfalls federführend an der
Entwicklung des SYCL"=Standards beteiligt ist. ComputeCpp unterstützt mehr
Hardware"=Plattformen als die anderen SYCL"=Implementierungen

Im Laufe der Implementierung des SYCL"=Backends stellte sich schnell heraus,
dass ComputeCpp nicht in der Lage sein würde, das SYCL"=Alpaka"=Backend zu
verwenden. Dafür gibt es zwei Gründe, die nachstehend weiter ausgeführt werden.

\subsubsection{Zeiger}

Zusätzlich zu der in Kapitel~\ref{implementierung} beschriebenen Problematik
mit Zeigern kommt es zu weiteren Schwierigkeiten, wenn Zeiger in Verbindung mit
ComputeCpp genutzt werden sollen.

ComputeCpp versucht die Information, zu welchem Adressraum ein Zeiger gehört,
durch spezielle Zeiger"=Attribute nachzuliefern. Ein Zeiger des Typs
\texttt{int*}, der auf den globalen Adressraum zeigt, wird vom
ComputeCpp"=Compiler zu \texttt{\_\_global int*} transformiert. Das Ergebnis der
Transformation wird vom selben Compiler jedoch als ein eigener Typ betrachtet,
der nichts mehr \texttt{int*} zu tun hat. Dadurch kommt es zu Problemen mit den
\textit{type traits} der C++"=Standardbibliothek, die einen Zeiger vom Typ
\texttt{\_\_global int*} nicht mehr als \texttt{int}"=Zeiger erkennen.

Da es dem Programmierer ebenfalls verboten ist, diese Attribute selbst zu
verwenden (ComputeCpp generiert einen Syntax"=Fehler), kann eine manuelle
Auswahl entsprechender Code"=Pfade auch nicht vorgenommen werden.

Darüber hinaus sind diese Zeigertypen nicht mit den SYCL"=Klassen kompatibel.
So führt die Umwandlung eines entsprechenden Zeigers in einen
SYCL"=\texttt{multi\_ptr} dazu, dass letzterer nicht mehr mit den atomaren
Funktionen des SYCL"=Standards verwendet werden kann. In diesem Fall meldet
ComputeCpp den Fehler, dass die Verwendung von Adressraum"=Attributen in
Verbindung mit atomaren Funktionen verboten ist.

\subsubsection{Fehlerhafte Instruktionen}

Auf NVIDIA"=GPUs generiert ComputeCpp mitunter Instruktionen, die von NVIDIAs
OpenCL"= oder CUDA"=Umgebung nicht verstanden werden. Dies fällt erst bei der
Ausführung des Kompilats auf, das entsprechende Fehlermeldungen der
NVIDIA"=Laufzeitumgebung meldet. Darüber hinaus fehlen ComputeCpp noch einige
wichtige Instruktionen, die für ein Funktionieren mit Alpaka nötig sind,
darunter viele mathematische Funktionen.

\subsection{Intel}

Eine weitere wichtige Implementierung des SYCL"=Standards wird seit Anfang des
Jahres 2019 von der Firma Intel herausgegeben. Diese quelloffene Variante ist
auf die Nutzung der Intel"=OpenCL"=Umgebungen für CPUs und GPUs ausgelegt. Es
ist aufgrund der parallel zu dieser Arbeit verlaufenen Weiterentwicklung
absehbar, dass kurz"= bis mittelfristig auch die FPGAs dieses Herstellers
unterstützt werden sollen.

Intels Compiler ist die einzige SYCL"=Implementierung, die Alpaka"=Quelltexte
mit aktiviertem SYCL"=Backend kompilieren konnte und kam deshalb zur
Verfizierung zum Einsatz.

\subsection{Xilinx}
\label{ergebnisse:nutzbarkeit:xilinx}

Während der Niederschrift dieser Arbeit wurde der Entwicklungszweig der
Xilinx"=SYCL"=Implementierung verwendet, der dem Hauptzweig einige Wochen voraus
und näher an der zugrunde liegenden Intel"=Implementierung ist. Die folgenden
Ausführungen beziehen sich auf:

\begin{itemize}
    \item den Commit \#\texttt{dfb95af} des Zweiges \texttt{sycl/unified/next}
          der Xilinx"=Implementierung,
    \item die Entwicklungsumgebung SDAccel 2019.1,
    \item die OpenCL"=Umgebung XRT 2.2 und
    \item die Deployment"=Plattform \texttt{xilinx-u200-xdma} für den
          Beschleuniger Alveo U200 in der Version \texttt{201830.2-2580015} für
          Ubuntu 18.04 und
    \item die zugehörigen Entwicklungs"=Plattform \texttt{xilinx-u200-xdma-dev}
          in der Version \texttt{201830.2-2580015} für Ubuntu 18.04.
\end{itemize}

Die SYCL"=Implementierung der Firma Xilinx hängt eng mit der Entwicklung des
Intel"=SYCL"=Compilers zusammen. Dabei wird in unregelmäßigen Abständen die
Code"=Basis des Intel"=Compilers übernommen und die Xilinx"=eigenen Codepfade
darin integriert. Das hat zur Folge, dass Fehlerkorrekturen des Intel"=Compilers
erst mit einiger Verzögerung in die Xilinx"=Implementierung Eingang finden.

Darüber hinaus erwies sich die Schnittstelle des SYCL"=Compilers zu Xilinx'
SDAccel"=Plattform, welche die eigentliche Synthese durchführt, sowie zu Xilinx'
OpenCL"=Treibern im Laufe dieser Arbeit als fehleranfällig oder unvollständig.
Aufgrund dieser Probleme war eine Nutzung des Alpaka"=SYCL"=Backends für
Xilinx"=FPGAs nicht möglich.

\subsubsection{Mathematische Funktionen}

Der Compiler generiert aus einigen mathematischen SYCL"=Funktionen
Instruktionen, die in Xilinx' OpenCL"=Treiber nicht vorhanden sind. Dies ist
zwar auf ein falsches Benennungsschema innerhalb der OpenCL"=Implementierung
zurückzuführen, steht einer Nutzbarkeit im Zusammenhang mit Alpaka aber trotzdem
im Wege. Andere mathematische Funktionen führen unter ungünstigen Umständen
durch ihre Nutzung dazu, dass der Compiler selbst abstürzt.  

\subsubsection{Strukturen}

Zu einem gravierenden Problem kommt es bei der Nutzung von benutzerdefinierten
Strukturen. Sofern diese außerhalb eines Kernels definiert und dann innerhalb
eines Kernels verwendet werden, kommt es zu einem Absturz des Compilers. Der
Grund dafür liegt in einem Fehler des im Hintergrund verwendeten
Xilinx"=OpenCL"=Compilers \texttt{xocc}, welcher die Synthese steuert. Nach
Aussage der an der SYCL"=Implementierung beteiligten Xilinx"=Mitarbeiter genießt
die Behebung dieses Fehlers niedrige Priorität, weshalb in nächster Zeit nicht
mit Besserung zu rechnen ist. Dieser Fehler ist der hauptsächliche Grund, warum
das aus vielen Strukturen bestehende Alpaka nicht mit Xilinx'
SYCL"=Implementierung verwendet werden kann.

\subsection{Block RAM und Pipelining}

In Xilinx' OpenCL"=Implementierung wird \textit{local memory} in Block RAM
synthetisiert, um den Logikblöcken möglichst schnelle Speicherzugriffe bieten zu
können. Xilinx' SYCL"=Implementierung verfügt ebenfalls über die notwendigen
Klassen und Strukturen, um \textit{local memory} innerhalb eines
\textit{Kernels} verwenden zu können. Aufgrund eines Compiler"=Fehlers werden
diese jedoch nicht als solche erkannt, wodurch Felder im \textit{local memory}
tatsächlich im \textit{global memory} angelegt werden.

Alternativ ließe sich BlockRAM über die in
Abschnitt~\ref{sycl:erweiterungen:xilinx:partitioning} erwähnten
SYCL"=Erweiterungen für die Feldpartitionierung verwenden. Diese sind jedoch
ebenfalls vom oben genannten Problem mit \textit{kernel}"=fremden Strukturen
betroffen und führen zu einem Compiler"=Absturz.

Zusammengefasst lässt sich Block RAM mit keiner der in SYCL dafür vorgesehenen
Funktionalität nutzen. Dies wirkt sich auch auf die Nutzung der Erweiterung
für Pipelining aus. Ein häufiges Speicherzugriffsmuster bei FPGAs sind die
\textit{burst reads} genannten Speicherzugriffe. In einer Schleife werden
aufeinander folgende Daten -- z.B. eine Pixelzeile eines Bildes -- vom
\textit{global memory} in den \textit{local memory} kopiert. Durch die Anwendung
des Pipelining"=Prinzips auf diese Schleife lässt sich die zur Verfügung
stehende Bandbreite besser ausnutzen, als wenn erst im eigentlichen Algorithmus
auf die Daten des \textit{global memory} zugegriffen würde.

Da der \textit{local memory} nicht zur Verfügung steht, erfolgen Lese"= und
Schreibzugriffe ausschließlich auf den \textit{global memory}. Die Zahl der
zugehörigen Lese"= und Schreib"=Ports ist jedoch begrenzt, wodurch die Schleife
nicht dem Pipelining"=Prinzip unterworfen werden kann.

\subsubsection{Kompatibilität mit der SDAccel"=Umgebung}

Zur Generierung von Profiling"=Informationen während der Hardware"=Emulation ist
es notwendig, die ausführbare Datei mit Debug"=Symbolen zu generieren
(Compiler"=Flag \texttt{-g}). Durch einen Compiler"=Fehler werden allerdings im
\textit{Kernel}"=Kompilat inkompatible Debug"=Symbole generiert, die von Xilinx'
OpenCL"=Umgebung nicht verarbeitet werden können. Das macht die Nutzung des
mitgelieferten visuellen Profilers \texttt{sdx} bzw. die Visualisierung der
Profiling"=Ergebnisse in Form von Timelines unmöglich.

\subsubsection{OpenCL-Treiber}

Xilinx' OpenCL"=Treiber, auf dem die SYCL"=Implementierung aufsetzt, erwies sich
im Zusammenspiel mit SYCL als äußerst instabil. Aufgrund seiner internen
Struktur ist er nicht in der Lage, einmal reservierten Speicher wieder
freizugeben, wenn das reservierende Programm abstürzt. Der Speicher bleibt so
lange unzugänglich, bis er neugestartet wird. Da dies nur mit
Administrationsrechten funktioniert, ist dies de facto ein Ausschlusskriterium
für den Einsatz in Rechenzentren oder Hochleistungs"=Clustern.

Sofern der Treiber nicht das oben beschriebene Verhalten zeigt, kann ein
fehlerhaftes Programm auch zum Komplettabsturz des Gesamtsystems führen. In
diesem Fall ist das System per Fernzugriff nicht mehr erreichbar und muss
vom Administrator (oder physisch per Reset"=Taste) neugestartet werden. Für den
Einsatz in Rechenzentren und vergleichbaren Einrichtungen ist dieses
Fehlerverhalten denkbar ungeeignet.

\subsection{Zusammenfassung}
 
Von den beschriebenen SYCL"=Implementierungen konnte nur der Intel"=Compiler im
Zusammenhang mit Alpaka genutzt werden, was als Ziel"=Hardware für das
SYCL"=Backend nur Intel"=CPUs und -GPUs zulässt. Insbesondere Xilinx"=FPGAs
können zum aktuellen Zeitpunkt aufgrund zahlreicher Probleme der
SYCL"=Implementierung derzeit nicht vom Alpaka"=SYCL"=Backend verwendet werden.

\section{Vergleich zwischen Alpaka und SYCL}

Im direkten Vergleich erwies sich SYCL gegenüber Alpaka als die modernere,
intuitivere und angenehmer zu benutzende Schnittstelle.

Dazu tragen SYCLs Orientierung an modernen C++"=Standards (alle untersuchten
Implementierungen unterstützen den C++17"=Standard, Intel und Xilinx den in
Entwicklung befindlichen C++20"=Standard) sowie die stilistische Nähe zur
C++"=Standardbibliothek bei. Dem gegenüber stehen Alpakas Stil"=Konventionen,
die vom in der C++"=Standardbibliothek verwendeten \textit{snake case}
(\texttt{eine\_kleine\_funktion()}) zugunsten des
\textit{lower camel case} (\texttt{eineKleineFunktion()}) abweichen. Darüber
hinaus finden sich in Alpaka stilistische Eigentümlichkeiten, die in den meisten
C++"=Projekten unüblich sind, z.B. die Schreibweise als \texttt{char const *}
anstelle des weiter verbreiteten \texttt{const char *}. Damit steht das
C++"=Projekt Alpaka auch im Gegensatz zu den vom C++"=Standardisierungskomitee
veröffentlichen Stilrichtlinien, den \textit{C++ Core Guidelines}.

Die Modellierung von Aufgabengraphen bzw.\ der Abhängigkeiten zwischen Kerneln
ist in SYCL deutlich einfacher als in Alpaka. Während dies in SYCL automatisch
von der Laufzeitumgebung übernommen wird, muss der Programmierer in Alpaka
selbst tätig werden -- der Aufwand ist in Alpaka also höher.

Die meisten Konzepte sind in SYCL und Alpaka jedoch recht ähnlich, sodass
hinsichtlich der Mächtigkeit keine großen Unterschiede bestehen. Darüber hinaus
hat Alpaka gegenüber SYCL den faktischen Vorteil der Hardware"=Unterstützung.
Während SYCL zur Zeit nur mit Intel"=CPUs und -GPUs zufriedenstellend
funktioniert (und möglicherweise nicht getesteter Automotive"= und
Embedded"=Hardware), ist Alpaka auf NVIDIA"= und AMD"=GPUs sowie über OpenMP auf
allen CPUs lauffähig. Daher ist Alpaka bereits in einigen produktiven
Anwendungen im Einsatz, während sich SYCLs Ökosystem bislang auf die diversen
Implementierungen sowie einige von der Firma Codeplay entwickelte Bibliotheken
für Mathematik und neuronale Netzwerke beschränkt. 

\section{Verifizierung des SYCL-Alpaka-Backends}
\label{ergebnisse:verifizierung}

Als Integrationstest für das SYCL"=Backend diente das in Alpaka geschriebene
Programm \textit{jungfrau"=photoncounter}, das von der Gruppe
\textit{Computergestützte Strahlenphysik} des \textit{Instituts für
Strahlenphysik} am \textit{Helmholtz"=Zentrum Dresden"=Rossendorf} in
Zusammenarbeit mit dem schweizerischen
\textit{Paul Scherrer Institut}\footnote{Sic! Das PSI schreibt sich ohne
Bindestriche.} (PSI) entwickelt wird und 2018 in Sebastian Benners
Bachelorarbeit beschrieben wurde \cite[vgl.][]{benner2018}.

\subsection{Der \textit{jungfrau-photoncounter}}
\label{ergebnisse:verifizierung:jungfrau}

Der \textit{jungfrau"=photoncounter} wertet Daten des Photonendetektor"=Typs
JUNGFRAU (\textit{ad\textbf{ju}sti\textbf{n}g \textbf{g}ain detector
\textbf{f}o\textbf{r} the \textbf{A}ramis \textbf{u}ser station}) aus und zählt
die von diesem registrierten Photonen. Der JUNGFRAU"=Detektortyp wird für den am
PSI befindlichen Freie"=Elektronen"=Laser \textit{SwissFEL} entwickelt. Ein
JUNGFRAU"=Detektor verfügt über eine Auflösung von 16 Megapixeln mit 2 Byte pro
Pixel und produziert Messergebnisse mit einer Frequenz von derzeit
\SI{100}{\hertz} (umgerechnet \SI{3.2}{\giga\byte\per\second}). Der Detektortyp
befindet sich nach wie vor in Entwicklung und soll langfristig Daten mit einer
Frequenz von \SI{2.2}{\kilo\hertz} (umgerechnet \SI{74}{\giga\byte\per\second})
generieren. Das macht eine dementsprechend schnelle Weiterverarbeitung der Daten
durch den \textit{jungfrau"=photoncounter} notwendig.

In Rahmen dieser Arbeit wurde nur der Teil des gesamten Funktionsumfangs des
\textit{jungfrau"=photoncounters} betrachtet, der sich mit dem Zählen der
Photonen pro Detektor"=Pixel befasst. Der Algorithmus berechnet für jedes
Detektorpixel die Formel

\begin{equation}\label{ergebnisse:verifizierung:jungfrau:formel}
    N_\gamma = \frac{\text{ADC} - \text{Sockel}}{\text{Verstärkung} \cdot E_\gamma}
\end{equation}
\noindent
Dabei bezeichnet $N_\gamma$ die Zahl der erkannten Photonen, \textit{ADC} das
vom Analogen ins Digitale konvertierte Messergebnis des Pixels, \textit{Sockel}
das von Mess- und Umweltbedingungen abhängige Grundrauschen des jeweiligen
Pixels (engl. \textit{pedestal}), \textit{Verstärkung} die Signalverstärkung
des Pixels (engl. \textit{gain}) und $E_\gamma$ die Konstante der
Photonenenergie.

Um -- im Vergleich zur Multiplikation -- langsame Divisionen bei der Berechnung
zu vermeiden, werden für \textit{Verstärkung} und $E_\gamma$ vor der Ausführung
des Algorithmus die Kehrwerte gebildet. Für $E_\gamma$ kann dies global
erfolgen. Für \textit{Verstärkung} erfolgt die Invertierung mittels eines
eigenen \textit{Kernels} (\texttt{GainmapInversionKernel}), der den Kehrwert für
jedes Pixel bildet.

Der \textit{Sockel}"=Wert wird ebenfalls für jedes Pixel bestimmt. Je nach
Messbedingungen kann er entweder im Abstand einiger Stunden gemessen (z.B. beim
Betrieb des Lasers bei Zimmertemperatur) oder kontinuierlich während der Messung
aktualisiert (z.B. beim Betrieb bei Temperaturen unter dem Gefrierpunkt, da hier
die Pixel empfindlicher sind) werden. Der zweite Fall wird durch einen weiteren
\textit{Kernel} (\texttt{CalibrationKernel}) abgedeckt, der anhand der
Standardabweichung und des Durchschnitts der zuletzt betrachteten Messergebnisse
und \textit{Sockel}"=Werte einen neuen \textit{Sockel}"=Wert berechnet.

Die Formel~\ref{ergebnisse:verifizierung:jungfrau:formel} wird in einem eigenen
\textit{Kernel} umgesetzt (\texttt{PhotonFinderKernel}), der auf jedes vom
Detektor produzierte Messergebniss angewendet wird.

\subsection{Verifizierung und Performanz}
\label{ergebnisse:verifizierung:performanz}

Das SYCL"=Alpaka"=Backend wurde mit dem Commit \#\texttt{78d9957} des
\texttt{sycl}"=Zweigs des Intel"=SYCL"=Compilers in Kombination mit der
\textit{Intel Graphics Compute Runtime for OpenCL} in der Version 19.46.14807
verifiziert. Dabei wurde eine integrierte GPU des Typs \textit{Intel HD Graphics
520} (auch als \textit{Skylake GT2} bezeichnet) als Hardware"=Plattform
verwendet, die über \SI{6}{\gibi\byte} Speicher verfügen kann. Diesen muss sie
jedoch mit dem \textit{Host}"=System teilen. Diese GPU erreicht eine maximale
Performanz von \num{100.8} GFLOPS bei doppelter Präzision\footnote{Die doppelte
Präzision wird innerhalb des Programms benötigt, da einige Zwischenergebnisse in
diesem Format gespeichert werden.}. Das \textit{Host}"=System selbst verfügt
über insgesamt \SI{8}{\gibi\byte} Speicher und ist mit einem Intel"=Prozessor
des Typs i7-6500U ausgestattet, der eine maximale Taktfrequenz von
\SI{3.1}{\giga\hertz} erreichen kann. Als Betriebssystem kam Ubuntu 19.10 zum
Einsatz.

Der Quelltext des \textit{jungfrau"=photoncounters} wurde dem
\texttt{master}"=Zweig des Projekts entnommen (Commit \#\texttt{d94f836}) und
geringfügig an das SYCL"=Backend angepasst. Dabei wurden keine Optimierungen der
\textit{Kernel} für Intel"=Hardware vorgenommen. Von den Entwicklern des
\textit{jungfrau"=photoncounters} wurde für die Messungen der Datensatz mit der
Bezeichung \texttt{px\_101016} zur Verfügung gestellt, der \num{10000}
Messungen mit jeweils $1024 \times 512$ Pixeln umfasst. Aufgrund der
Speicherlimitierung des Gesamtsystems wurde der Datensatz auf \num{1000}
Messergebnisse reduziert.

Die Abbildung~\ref{ergebnisse:verifizierung:performanz:photonen} zeigt das
Ergebnis des oben beschriebenen Algorithmus für den reduzierten Datensatz. Die
weißen Pixel zeigen ungültige Messergebnisse an. Diese werden im
\textit{jungfrau"=photoncounter} durch eine separate Maske eigentlich
abgeschaltet. Durch einen Fehler unbekannter Herkunft funktionierte dieser Teil
des Programms jedoch nicht mit dem Alpaka"=SYCL"=Backend. Stattdessen wurden
alle Pixel maskiert, was zu einem leeren Ergebnis führte. Aus diesem Grund wurde
auf den Einsatz der Maske verzichtet.

Nach mündlicher Aussage der Entwickler des \textit{jungfrau"=photoncounter} ist
es nicht möglich, die Ergebnisse des im vorigen Abschnitt beschriebenen
Algorithmus anhand synthetischer Daten zu verifizieren. Die Verifizierung realer
Daten erfolgt deswegen durch die Anwender des \textit{Paul Scherrer Instituts},
die die Ergebnisse auf ihre Plausibilität prüfen. Die
Abbildung~\ref{ergebnisse:verifizierung:performanz:photonen} wurde von einem der
Entwickler des \textit{jungfrau"=photoncounter}, Jonas Schenke, für plausibel
befunden.

\begin{figure}
    \centering
    \includegraphics{log/photon.png}
    \caption[Visualisierung eines Ergebnisses des
             \textit{jungfrau-photoncounters}]{Visualisierung eines Ergebnisses
             des \textit{jungfrau-photoncounters}. Weiße Pixel zeigen
             ungültige Ergebnisse an, die nicht maskiert werden konnten.}
    \label{ergebnisse:verifizierung:performanz:photonen}
\end{figure}

Das gezeigte Bild wurde anhand von \num{1000} Messergebnissen berechnet. Die
genannte GPU benötigte für die Verarbeitung der Eingangsdaten (ohne das Laden
der Daten von der Festplatte, Kalibrierung und Invertierung)
\SI{48.372}{\second}. Damit erreicht sie eine Verarbeitungsfrequenz von
\num{20.6731} Messungen pro Sekunde, was unter den derzeitigen \SI{100}{\hertz}
und deutlich unter den avisierten \SI{2.2}{\kilo\hertz} des JUNGFRAU"=Detektors
liegt. Da es sich bei der verwendeten GPU nicht um einen
Hochleistungsbeschleuniger handelt und ihr tatsächlicher Einsatz im Umfeld eines
JUNGFRAU"=Detektors nicht zu erwarten ist, reicht dieses Ergebnis als Beweis
für die grundsätzliche Funktionstüchtigkeit des Alpaka"=SYCL"=Backends aus. In
einem realen Anwendungsszenario wären Beschleuniger mit höherer Leistung
(insbesondere bei doppelter Präzision) zwingend erforderlich, um die angestrebte
Frequenz des JUNGFRAU"=Detektors erreichen zu können.

\subsection{Nutzbarkeit von FPGAs}

Da der Detektor mit fester Frequenz Daten produziert, würde sich ein FPGA für
den beschriebenen Verarbeitungsalgorithmus sehr gut eignen. Durch die in
Abschnitt~\ref{ergebnisse:nutzbarkeit:xilinx} geschilderten Schwierigkeiten
mit der derzeit zur Verfügung stehenden Xilinx"=SYCL"=Implementierung konnte 
das bestehende Alpaka"=Programm jedoch nicht auf Xilinx"=FPGAs zur Ausführung
gebracht werden.

\section{Box-Filter}
\label{ergebnisse:box}

Da sich die ursprünglich für den Einsatz von FPGAs vorgesehene Alpaka"=Anwendung
aufgrund des Entwicklungsstatus der Xilinx"=SYCL"=Implementierung nicht für
Messungen eignete, wurde ein Bildverarbeitungsprogramm in SYCL implementiert.
Dieses wendet einen einfachen Box"=Filter auf eine Menge von Bildern an.

\subsection{Algorithmus}

Die Grundlage des Programms bildet ein einfacher Box"=Filter für
zweidimensionale Bilder \cite[vgl.][]{nakamura2017}. Dieser berechnet für den
Pixelwert $p$ an der Position $(x, y)$ den gefilterten Wert $p'$ an gleicher
Stelle. $p'$ ist der Mittelwert des Werts $p$ sowie der direkt an der Position
$(x, y)$ horizontal, vertikal und diagonal angrenzenden Pixelwerte:

\begin{equation}
    p'(x, y) = \frac{1}{9} \cdot \sum_{j = {-1}}^{1} \sum_{i = {-1}}^{1} p(x + i, y + j)
\end{equation}

Sofern ein Pixel am Bildrand liegt, wird der Wert der Nachbarn, die außerhalb
des Bildes lägen, als $0$ angenommen.

Der dem Algorithmus entsprechende SYCL"=Kernel ist im
Anhang~\ref{anhang:source:cpp:boxkernel} zu finden.

\subsection{Messergebnisse}

Der obige Algorithmus wurde in einer Schleife nacheinander auf \num{512} Bilder
mit jeweils $512 \times 256$ Pixeln angewendet. Außerdem wurden separate
Messungen für die Pixel"=Datentypen \texttt{int} und \texttt{float}
vorgenommen, um die durch einen anderen Datentyp verursachten Veränderungen des
Ressourcenverbrauchs beobachten zu können.

Der SYCL"=Quelltext des Programms wurde mit dem SYCL"=Compiler der
Xilinx"=Implementierung (Entwicklungszweig \texttt{sycl/unified/next},
Commit \#\texttt{dfb95af}) übersetzt. Die darauf folgende High"=Level"=Synthese
wurde von den Werkzeugen der Entwicklungsumgebung SDAccel in der Version 2019.1
durchgeführt.

Die Ausführung der synthetisierten Schaltungen erfolgte auf dem FPGA"=Knoten
\texttt{h002} des am Helmholtz"=Zentrum Dresden"=Rossendorf befindlichen
HPC"=Systems \textit{Hemera}. Dieses verfügt über zwei
Xilinx"=FPGA"=Beschleuniger des Typs \textit{Alveo U200}, wovon einer für die
Messungen verwendet wurde. Softwareseitig kamen auf diesem Knoten die Module
\texttt{gcc/9.1.0} und \texttt{xilinx/2.2} zum Einsatz. Die
SYCL"=Implementierung steht nicht als Modul zu Verfügung und musste zunächst
in der oben genannten Version lokal im Benutzerverzeichnis auf dem Knoten
kompiliert werden.

\subsubsection{Compile"=Zeiten}

Schon während der Synthese der Schaltungen ergaben sich durch den Austausch der
Datentypen signifikante zeitliche Unterschiede bei den einzelnen
Syntheseschritten (siehe Abbildung~\ref{ergebnisse:box:messung:zeitvergleich}).
Während beide Datentypen während der Umwandlung des SYCL"=Kernels in ein
HLS"=taugliches Format noch ungefähr die gleiche Zeit benötigen, benötigt die
\texttt{int}"=Schaltung in allen folgenden Schritten deutlich messbar mehr Zeit
als das \texttt{float}"=Äquivalent. Anscheinend fällt es den
Synthesewerkzeugen leichter, optimierte Schaltungen für
\texttt{float}"=Operationen zu generieren.

\begin{figure}[htb]
    \centering
    \begin{tikzpicture}
        \begin{axis}[title = {Vergleich der für die HLS notwendigen Zeitaufwände},
                     axis line style = {HKS92},
                     %
                     xlabel = {Syntheseschritte},
                     xtick = data,
                     symbolic x coords = {Kernel,Synthese,Optimierung,Platzierung,Routing,Bitstream},
                     x tick label style = {align=center},
                     xlabel near ticks,
                     %
                     ylabel = {benötigte Zeit},
                     y filter/.code = {\pgfmathparse{#1/3600}},
                     yticklabel = { % Aufteilung in Stunden, Minuten und Sekunden
                        \pgfmathsetmacro\hours{floor(\tick)}%
                        \pgfmathsetmacro\minutes{(\tick - \hours) * 0.6}%
                        \pgfmathprintnumber{\hours}h\pgfmathprintnumber[fixed, fixed zerofill, skip 0.=true, dec sep = {}]{\minutes}m
                     },
                     ylabel near ticks,
                     %
                     ymajorgrids = true,
                     grid style = dashed,
                     legend cell align = left,
                     legend pos = north east,
                     legend style = {draw = HKS92},
                     no markers,
                     ybar,
                     width = \textwidth
                     ]

            \addplot[HKS92, fill=HKS44] table [x = step, y = int, col sep = semicolon] {data/compile.csv};
            \addlegendentry{\texttt{int}};

            \addplot[HKS92, fill=HKS33] table [x = step, y = float, col sep = semicolon] {data/compile.csv};
            \addlegendentry{\texttt{float}};
        \end{axis}
    \end{tikzpicture}
    \caption[Vergleich der für die HLS notwendigen Zeitaufwände bei
             verschiedenen Datentypen]{Vergleich der für die HLS notwendigen
             Zeitaufwände bei verschiedenen Datentypen. Die Syntheseschritte
             werden während der HLS von links nach rechts ausgeführt. Der
             Schritt \textit{Kernel} bezeichnet die Umwandlung des SYCL"=Kernels
             in ein für die HLS geeignetes Format. Der Schritt \textit{Synthese}
             meint die Block"=Level"=Synthese, \textit{Optimierung} die
             Logikoptimierung, \textit{Platzierung} die Logikplatzierung,
             \textit{Routing} das Routing und \textit{Bitstream} die
             Bitstream"=Generierung.}
    \label{ergebnisse:box:messung:zeitvergleich}
\end{figure}

\subsubsection{Ressourcenverbrauch}

Beim Blick auf die Auslastung der zur Verfügung stehenden Ressourcen (siehe
Abbildung~\ref{ergebnisse:box:messung:ressourcenvergleich}) fallen zwei Dinge
auf:

Zum einen benötigt die \texttt{int}"=Schaltung nicht nur mehr Zeit, sondern auch
mehr Ressourcen als ihr \texttt{float}"=Gegenstück. Die CLBs und ihre
Komponenten werden wesentlich stärker in Anspruch genommen als bei der
\texttt{float}"=Schaltung. Besonders deutlich wird der Unterschied bei der
Betrachtung der Netzlisten (siehe
Abbildung~\ref{ergebnisse:box:messung:netzliste}). Es ist klar zu sehen, dass
die \texttt{int}"=Schaltung weitaus mehr Fläche auf dem Chip belegt als ihr
\texttt{float}"=Gegenstück. Eine weitere Erkenntnis ist, dass die Synthese
zunächst den inneren Bereich des FPGAs zu füllen scheint und erst dann auf die
weiter außen liegenden Bereiche ausweicht.

Zum anderen ist das im vorigen Abschnitt beschriebene fehlerhafte Verhalten
bezüglich SYCLs \textit{local memory} gut sichtbar. Beide Schaltungen verwenden
keinen \textit{UltraRAM} und nur einen sehr kleinen Anteil des zur Verfügung
stehenden \textit{Block RAM}.

\begin{figure}[htb]
    \centering
    \begin{tikzpicture}
        \begin{axis}[title = {Vergleich der Ressourcennutzung},
                     axis line style = {HKS92},
                     %
                     xlabel = {Ressourcen},
                     xtick = data,
                     symbolic x coords = {CLB,Flip-Flops,LUTs,DSPs,Block RAM,UltraRAM},
                     x tick label style = {align=center},
                     xlabel near ticks,
                     %
                     ylabel = {Auslastung},
                     yticklabel = {\pgfmathprintnumber{\tick}\%},
                     ylabel near ticks,
                     ymin = 0,
                     ymax = 100,
                     %
                     ymajorgrids = true,
                     grid style = dashed,
                     legend cell align = left,
                     legend pos = north east,
                     legend style = {draw = HKS92},
                     no markers,
                     ybar,
                     %scale only axis
                     width = \textwidth
                     ]

            \addplot[HKS92, fill=HKS44] table [x = res, y = int, col sep = semicolon] {data/resources.csv};
            \addlegendentry{\texttt{int}};

            \addplot[HKS92, fill=HKS33] table [x = res, y = float, col sep = semicolon] {data/resources.csv};
            \addlegendentry{\texttt{float}};
        \end{axis}
    \end{tikzpicture}
    \caption[Vergleich der Ressourcennutzung bei verschiedenen Datentypen]
            {Vergleich der Ressourcennutzung bei verschiedenen Datentypen}
    \label{ergebnisse:box:messung:ressourcenvergleich}
\end{figure}

\begin{figure}[htb]
    \centering
    \pdfimageresolution=92
    \begin{minipage}{0.45\textwidth}
        \centering
        \includegraphics{box_res_int.png}
        \caption*{Netzliste für den Datentyp \texttt{int}}
    \end{minipage}\hfill
    \begin{minipage}{0.45\textwidth}
        \centering
        \includegraphics{box_res_float.png}
        \caption*{Netzliste für den Datentyp \texttt{float}}
    \end{minipage}
    \pdfimageresolution=72
    \caption[Netzliste des synthetisierten Box-Filter-Kernels mit verschiedenen
             Datentypen]{Netzliste des Box-Filter-Kernels mit verschiedenen
             Datentypen. Die unterschiedliche Auslastung der
             Hardware"=Ressourcen ist in dieser Ansicht deutlich sichtbar. Blaue
             Bereiche werden für die Schaltung verwendet, dunkelgraue sind nicht
             in Gebrauch. Der orange Bereich ist für die Laufzeitumgebung
             reserviert. Die kräftigen grauen und roten Balken repräsentieren
             die Zahl der Verbindungen zwischen den einzelnen
             FPGA"=Komponenten.}
    \label{ergebnisse:box:messung:netzliste}
\end{figure}

\subsubsection{Laufzeiten}

Über die in SYCL vorhandenen Profiling"=Funktionen lassen sich Erkenntnisse über
die Laufzeit gewinnen. Leider war trotz vorherige Verifizierung durch Software"=
und Hardware"=Emulation nur die \texttt{float}"=Schaltung auf dem FPGA"=Knoten
lauffähig. Diese benötigte für die Filterung von 512 Bildern etwa
\SI{22.26}{\second}, oder \SI{45}{\milli\second} pro Bild.

Die \texttt{int}"=Schaltung führte dagegen entweder zu einem Absturz des
gesamten Knotens oder wurde vom HPC"=System nach einer Stunde(!) beendet, ohne
ein Ergebnis zu produzieren. Daher ist ein Vergleich der Laufzeiten zwischen
den Schaltungen für verschiedene Datentypen nicht möglich.

\chapter{Fazit}
\label{fazit}

Es konnte gezeigt werden, dass SYCL sich als Backend für die Alpaka"=Bibliothek
grundsätzlich eignet. Da jedoch gravierende Inkompatibilitäten zwischen SYCL und
Alpaka bestehen, ist eine Entwicklung über den Prototypen"=Status hinaus derzeit
nicht umzusetzen. Diesbezüglich sind noch einige konzeptionelle Änderungen in
Alpaka und/oder SYCL notwendig.

Aufgrund des Zustands der verfügbaren SYCL"=Implementierungen ist eine
Ausführung von Alpaka"=Programmen auf den meisten Hardware"=Plattformen derzeit
nicht möglich. Lediglich Intel"=CPUs und "=GPUs sind zur Zeit nutzbar, sofern
man die Beschränkungen des Backend"=Prototypen in Kauf zu nehmen bereit ist.
Dagegen ist die Nutzbarkeit von FPGAs sowohl durch das Alpaka"=Backend als auch
durch SYCL selbst in nächster Zeit fraglich. Hier dürfte vor allem die
Beobachtung der Intel"=Implementierung des SYCL"=Standards von Bedeutung sein,
da eine Unterstützung der hauseigenen FPGAs mittelfristig wahrscheinlich ist und
die bisherige Implementierung recht ausgereift wirkt. Dem gegenüber steht die
auf der Intel"=Implementierung aufsetzende Xilinx"=Implementierung, die
insgesamt sehr instabil ist und unter einer sehr kleinen Entwicklerzahl leidet.
Es ist dennoch lohnenswert, das Projekt der FPGA"=Unterstützung für beide
Hersteller in Alpaka weiter zu verfolgen, weil Alpaka so ein wichtiges
Alleinstellungsmerkmal gegenüber vergleichbaren Projekten wie Kokkos oder
HPX.Compute hätte.

Gegenüber Alpaka bietet SYCL das modernere, intuitivere und standardnähere
Programmier"=Interface. Das fragmentierte Ökosystem sowie der eingeschränkte
Hardware"=Support sprechen im Moment aber recht deutlich gegen die Nutzung
dieses insgesamt vielversprechenden Ansatzes zur parallelen Programmierung. Es
bleibt daher abzuwarten, ob SYCL in den nächsten Monaten und Jahren weitere
Verbreitung erfährt oder in der Bedeutungslosigkeit versinkt. Wichtig wäre hier
eine kurzfristige und umfangreiche Unterstützung von NVIDIA"=GPUs, die
den Bereich des HPC mit weitem Abstand dominieren. Auch für moderne AMD"=GPUs
existiert bislang keine SYCL"=Implementierung, die auf dem HPC"=Sektor
vorbehaltlos einzusetzen ist. Dadurch kann SYCL in der GPGPU"=Programmierung
keine Rolle spielen, sofern man von den vergleichsweise leistungsschwachen
Intel"=Laptop"=GPUs absieht. Diesbezüglich wird interessant sein, ob und wie gut
Intels zukünftige dedizierte GPU"=Plattform mit SYCL genutzt werden kann.


%%%%%%%%%%%%%%%%%%%%%%%%%%%%%%%%%%%%%%%%%%%%%%%%%%%%%%%%%%%%%%%%%%%%%%%%%%%%%%%%
% Nachspann
%%%%%%%%%%%%%%%%%%%%%%%%%%%%%%%%%%%%%%%%%%%%%%%%%%%%%%%%%%%%%%%%%%%%%%%%%%%%%%%%

\printbibliography[heading=bibintoc]

\listoffigures
\listoftables
\listoflistings

\appendix
\chapter{Fehlerberichte und Korrekturen}
\label{anhang:fehler}

\section{Fehlerberichte und Korrekturen für die Xilinx-OpenCL-Laufzeitumgebung}
\label{anhang:fehler:xrt}

\section{Fehlerberichte und Korrekturen für den Xilinx-SYCL-Compiler}
\label{anhang:fehler:xilinx}

\section{Fehlerberichte und Korrekturen für den Intel-SYCL-Compiler}
\label{anhang:fehler:intel}

\section{Fehlerberichte und Korrekturen für den ComputeCpp-SYCL-Compiler}
\label{anhang:fehler:computecpp}

\chapter{GitHub-Diskussionen}
\label{anhang:github}

\section{Diskussionen mit dem SYCL-Spezifikationskomitee}
\label{anhang:github:syclspec}

\subsection{Why is there no way to allocate local memory inside a ND-kernel
            (parallel\_for)?}
\label{anhang:github:syclspec:staticshared}

Original: \url{https://github.com/KhronosGroup/SYCL-Docs/issues/20}, zuletzt
abgerufen am 07. August 2019.

\begin{otherlanguage}{english}
    \paragraph{Jan Stephan} I know this is possible using the hierarchical
                            \texttt{parallel\_for} invoke, but we can't do it
                            with the \texttt{nd\_item} version despite being
                            able to specify the group size. OpenCL allows this
                            (AFAIK), SYCL's competitors do, too, so why not
                            allow it in SYCL? I suppose there must be a reason
                            for leaving it out.
    \paragraph{Victor Lomuller (Codeplay)} The main reason is the host device.
                            If your compiler is not SYCL aware, you need to be
                            able to preallocate this memory before calling the
                            functor, which is not trivial without compiler
                            support.
    \paragraph{Jan Stephan} Wouldn't this be solvable by doing the inverse of
                            the hierarchical case? I.e. everything is
                            \texttt{private} by default if declared inside the
                            kernel, unless embedded with something like
                            \texttt{cl::sycl::local\_memory}.
    \paragraph{Victor Lomuller} It does not address the compiler support
                                problem. You still need to preallocate memory
                                before calling the functor, but without compiler
                                support, you cannot know the amount of memory
                                you need nor where to place the pointer to that
                                memory (the stack frame does not exist yet).
    \paragraph{Ronan Keryell (Xilinx)} which SYCL competitor can run on CPU
                              without a specific compiler? This allows for
                              example to use HellGrind \& ThreadSanitizer with
                              plain GCC or Clang to debug a SYCL program just by
                              running it on my laptop. I find this an amazing
                              feature of SYCL...
\end{otherlanguage}

\subsection{How to extract address space from raw pointers?}

Original: \url{https://github.com/KhronosGroup/SYCL-Docs/issues/21}, zuletzt
abgerufen am 07. August 2019.


\confirmation[language=ngerman]

\end{document}
