\RequirePackage[ngerman=ngerman-x-latest]{hyphsubst}
\documentclass[ngerman]{tudscrreprt}

\usepackage[ngerman]{babel}
\usepackage[math]{blindtext}
\usepackage[T1]{fontenc}
\usepackage[symbol]{footmisc}
\usepackage{isodate}
\usepackage{selinput}
\usepackage{tudscrsupervisor}

\begin{document}

\faculty{Fakultät Informatik}
\department{}
\institute{Institut für Technische Informatik}
\chair{Seniorprofessor Dr.-Ing. habil. Rainer G. Spallek}

\date{16.12.2019}
\title{Entwicklung eines SYCL-Backends für die Alpaka-Bibliothek und dessen
       Evaluation mit Schwerpunkt auf FPGAs}
\subject{diploma}
\graduation[Dipl.-Inf.]{Diplom-Informatiker}

\author{%
        Jan Stephan%
        \matriculationnumber{3755136}%
        \dateofbirth{8.5.1991}%
        \placeofbirth{Wilhelmshaven}}
\matriculationyear{2012}
\course{Informatik}

\issuedate{15.4.2019}
\duedate{16.12.2019}

\supervisor{Dr.-Ing. Oliver Knodel (HZDR) \and
            Matthias Werner, M.Sc. (HZDR)}
\referee{Prof. Dr.-Ing. habil. Rainer G. Spallek \and
         Prof. Dr. rer. nat. Ulrich Schramm (HZDR)}
\professor{Prof. Dr.-Ing. habil. Rainer G. Spallek}

\taskform{
    Alpaka ist eine plattformabstrahierende C++11-""Bibliothek zur parallelen
    Programmierung von Multicore- und Manycore-Architekturen. SYCL liefert ein
    modernes C++11-""\-Pro\-gram\-mier\-mo\-dell für OpenCL und ist aufgrund der
    zunehmenden Plattformunterstützung eine wünschenswerte Erweiterung für
    Alpaka. SYCL-Compiler erlauben u.a. Zugriff auf NVIDIA-GPUs (experimentell)
    und FPGAs (triSYCL).\\
    \\
    \noindent
    Ziel dieser Arbeit ist es, SYCL als Backend-Variante für Alpaka zu
    implementieren und den Einsatz gängiger SYCL-Compiler hinsichtlich CPUs,
    GPUs und FPGAs an einer realen Alpaka-Anwendung zu evaluieren.
}{%
    \item Literaturrecherche zu SYCL/OpenCL und FPGAs.
    \item Einarbeitung in die FPGA-Programmierung mittels des triSYCL-
          und Xilinx-SDAccel-Ökosystems.
    \item Untersuchung der Performance-Analysemöglichkeiten hinsichtlich der
          Nutzbarkeit und erreichten Leistung im Vergleich zu anderen Konzepten
          und Architekturen.
    \item Evaluierung des SYCL-Backends anhand eines in Alpaka entwickelten
          Verarbeitungsalgorithmus für Röntgenstrahlen-Pixeldetektordaten, u.a.
          hinsichtlich der erreichbaren Datenraten sowie der Nutzbarkeit von
          FPGAs.
    \item Zusammenstellung, Auswertung und Dokumentation der Ergebnisse.
}
\end{document}
