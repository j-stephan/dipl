\chapter{Der SYCL-Standard}\label{sycl}

Durch die Xilinx"=Implementierung der vor einigen Jahren veröffentlichten
SYCL"=Spezifikation gibt es eine weitere Möglichkeit, ein Problem auf
algorithmischer Ebene zu beschreiben und über die HLS in eine
\gls{fpga}"=Schaltung zu synthetisieren. Eine einfache SYCL"=Einführung sowie
die für \gls{fpga}s wichtigen Besonderheiten sind daher das Thema dieses
Kapitels.

\section{Überblick}\label{sycl:ueberblick}

Mit dem SYCL"=Standard\footnote{Entgegen des optischen Anscheins ist
\glqq SYCL\grqq\ kein Akronym, sondern ein Eigenname.}\cite[vgl.][]{sycl2019}
verfolgt die herausgebende Khronos"=Gruppe das Ziel eines abstrakten
C++"=Programmiermodells für \gls{opencl}, das die Flexibilität und Einfachheit
moderner C++"=Standards bieten soll, während gleichzeitig Konzeption und
Portabilität des \gls{opencl}"=1.2"=Standards \cite[vgl.][]{opencl2012}
beibehalten werden.

Von \gls{opencl} erbt SYCL damit den Anspruch, eine einheitliche
Programmierschnittstelle für verschiedene Beschleunigertypen unterschiedlicher
Hersteller zu bieten. Das heißt, dass ein einmal geschriebener Quelltext, der
auf einem Beschleuniger ausgeführt werden kann, möglichst ohne große Änderungen
sowohl auf einer \gls{cpu}, einer \gls{gpu}, einem \gls{dsp} oder einem
\gls{fpga} ausführbar sein soll.

SYCL unterscheidet sich von \gls{opencl} im Hinblick auf die Struktur des
Quelltextes: Bei \gls{opencl} sind die Quelltexte für das steuernde Programm
(\textit{Host}) und den Programmteil, der vom Beschleuniger (\textit{Device})
ausgeführt wird, voneinander getrennt. Diese Design"=Entscheidung des
\gls{opencl}"=Standards ist durch das Ziel der Plattformunabhängigkeit
begründet: Ein Entwickler kennt während der Kompilierung des Hauptprogramms
nicht notwendigerweise die vorhandenen Beschleuniger der Zielplattform. Dadurch
wird der \textit{Device}"=spezifische Quelltext häufig erst zur Laufzeit des
Programms kompiliert, da in diesem Moment der konkrete Beschleuniger bekannt
ist. Dieser Ansatz hat jedoch den Nachteil, dass der Compiler des
\textit{Device}"=Quelltexts (im Folgenden als \textit{Kernel} bezeichnet) keine
Annahmen über das \textit{Host}"=Programm bzw. den den \textit{Kernel}
umgebenden Quelltext mehr treffen kann, was zu einem geringeren
Optimierungspotential führt. \gls{opencl}"=\textit{Kernel} lassen sich zwar auch
vor der Laufzeit des Programms für eine konkrete Zielplattform kompilieren (dies
geschieht aufgrund der langen Kompilierungszeiten bei \gls{fpga}s), büßen
dadurch aber ihre Plattformunabhängigkeit ein.

Ein SYCL-Quelltext kennt dagegen keine strikte Trennung zwischen \textit{Host}-
und \textit{Device}"=Anweisungen, was zu einer besseren Analyse des den
\textit{Kernel} umgebenden Kontexts führt. Außerdem ergibt sich der weitere
Vorteil, dass \textit{Host} und \textit{Device} Quelltext teilen können, wie
z.B.\ beidseitig verwendete Hilfsfunktionen. Der \textit{Kernel} wird dabei vom
\textit{Device}"=Compiler extrahiert und in eine Form umgewandelt, die von der
Ziel"=Hardware zur Laufzeit kompiliert oder ausgeführt werden kann.
\cite[vgl][35]{sycl2019}

Ein weiterer wichtiger Unterschied zu \gls{opencl} besteht darin, dass
jedes SYCL"=Programm mit einem Standard"=C++"=Compiler übersetzt werden kann,
sofern keine direkten Interaktionen mit \gls{opencl} selbst erfolgen. Damit
lässt sich ein SYCL"=Programm auf jeder \gls{cpu} ausführen, für die ein
(moderner) C++"=Compiler existiert, wenngleich dies Einschränkungen bei der
erreichbaren Leistung bedeuten kann. So schreibt die SYCL"=Spezifikation für
diesen Fall nur die Ausführbarkeit selbst vor, aber nicht die Nutzung aller
\gls{cpu}"=Kerne oder Vektorregister. \cite[vgl.][15]{sycl2019}

Seit der ersten Veröffentlichung im März 2014 \cite[vgl.][]{khronos2014} mit der
Versionsnummer 1.2 wurde die SYCL"=Spezifikation stetig weiterentwickelt; die
zur Zeit aktuelle Spezifikation vom April 2019 trägt die Revisionsnummer 1.2.1
Revision 5. \cite[vgl.][1]{sycl2019}

Teil der Khronos"=Gruppe sind (unter anderen) die \gls{fpga}"=Hersteller Xilinx
und Intel. Xilinx unterstützt den SYCL"=Standard in Form einer Erweiterung der
bestehenden HLS"=Werkzeuge bereits, während Intel dies für die eigenen FPGAs
mittelfristig plant; für Intel"=\gls{cpu}s und "=\gls{gpu}s ist bereits eine
SYCL"=Implementierung verfügbar (der Abschnitt~\ref{sycl:implementierungen}
befasst sich mit allen verfügbaren Implementierungen).

\subsection{Der AXPY-Algorithmus als Beispiel}
\label{sycl:ueberblick:axpy}

Ein im Bereich der parallelen Programmierung häufig verwendetes einführendes
Beispiel ist der sogenannte AXPY"=Algorithmus, der ursprünglich aus der
Bibliothek \textit{Basic Linear Algebra Subprograms} (\gls{blas}) stammt
\cite[vgl.][]{lawson1979}. Dieser führt die Berechnung

\begin{equation}\label{sycl:ueberblick:axpy:formel}
    \vec{y} = a \cdot \vec{x} + \vec{y}
\end{equation}

aus und ist aufgrund seiner Einfachheit und hohen erreichbaren Parallelität
(sofern $\vec{x}$ und $\vec{y}$ viele Elemente enthalten) sehr beliebt.

AXPY lässt sich für eine Einführung in SYCL gut verwenden und wird daher in den
folgenden Abschnitten als illustrierendes Beispiel genutzt.

\subsection{Struktur eines SYCL-Programms}

Ein SYCL"=Programm lässt sich in mehrere aufeinander aufbauende Stufen
unterteilen, wie in Quelltext~\ref{sycl:ueberblick:axpy:struktur} zu sehen ist.
Die einzelnen Platzhalter im Quelltext werden in den nächsten Abschnitten mit
Inhalt gefüllt.

\begin{code}
    \begin{minted}[fontsize=\small]{c++}
#include <cstdlib>
#include <CL/sycl.hpp>

auto main() -> int
{
    // Beschleunigerwahl und Befehlswarteschlange

    // Speicherreservierung und -initialisierung

    // Kernel-Definition und -ausführung

    // Synchronisierung

    return EXIT_SUCCESS;
}
    \end{minted}
    \caption{Struktur eines SYCL-Programms}
    \label{sycl:ueberblick:axpy:struktur}
\end{code}

\subsection{Beschleunigerwahl und Befehlswarteschlange}
\label{sycl:ueberblick:axpy:queue}

Von \gls{opencl} erbt SYCL die Plattformunabhängigkeit. Es wird das
Vorhandensein mindestens einer \gls{opencl}"=Plattform auf dem System
angenommen\footnote{Diese wird aber nicht vorausgesetzt! Jede
SYCL"=Implementierung muss auch ohne eine OpenCL"=Plattform auf dem
\textit{Host} lauffähig sein.}, was im Umkehrschluss bedeutet, dass unter
Umständen zwischen mehreren verschiedenen Plattformen konkurrierender Hersteller
gewählt werden muss. 

Die SYCL"=Spezifikation bietet dem Programmierer mehrere Möglichkeiten, die
gewünschte Plattform für sein Programm auszuwählen. Der einfachste Ansatz
besteht darin, eine Befehlswarteschlange (die \textit{Queue} genannt wird) für
einen Beschleuniger zu konstruieren. Über die \textit{Queue} erfolgt die
Kommunikation zwischen dem \textit{Host} und einem \textit{Device}, also
Kopieroperationen, das Starten eines \textit{Kernels} sowie die
Synchronisierung. Letztere ist notwendig, da es sich bei dem \textit{Device}
um eine vom \textit{Host} weitestgehend unabhängige Hardware handelt, die
Operationen also (aus Sicht des \textit{Hosts}) asynchron ablaufen.

Jedes genutzte \textit{Device} erhält in SYCL mindestens eine eigene
\textit{Queue}, so dass auch die Nutzung mehrerer Beschleuniger möglich ist.

Eine \textit{Queue} kann durch das Übergeben eines Auswahlkriteriums für das
gewünschte \textit{Device} konstruiert werden. Die Auswahlkriterien werden in
der SYCL"=Spezifikation \texttt{device\_selector} genannt. Neben den in der
Spezifikation vorhandenen Kriterien (beispielsweise \texttt{cpu\_selector},
\texttt{gpu\_selector} oder \texttt{host\_selector}) ist es Herstellern oder
Programmierern selbst möglich, durch das Erben von der Elternklasse
\texttt{device\_selector} eigene Kriterien zu definieren. Beispielsweise findet
sich in den Testfällen der von Xilinx entwickelten SYCL"=Implementierung ein
\texttt{device\_selector} für die eigenen Geräte, der die \gls{fpga}s über den
Firmennamen findet. Mit dessen Hilfe lässt sich die \textit{Queue} für ein
Xilinx"=FPGA wie in Quelltext~\ref{sycl:ueberblick:axpy:queue:src} dargestellt
erzeugen.
%
\begin{code}
    \begin{minted}[fontsize=\small]{c++}
class XOCLDeviceSelector : public cl::sycl::device_selector {
    public:
        int operator()(const cl::sycl::device &Device) const override {
            const std::string DeviceVendor =
                            Device.get_info<cl::sycl::info::device::vendor>();
            return (DeviceVendor.find("Xilinx") != std::string::npos) ? 1 : -1;
        }
};

/* ... */

auto queue = cl::sycl::queue{XOCLDeviceSelector{}};
    \end{minted}
    \caption{Auswahl eines Xilinx"=FPGAs und Erzeugung einer zugehörigen
             \textit{Queue}}
    \label{sycl:ueberblick:axpy:queue:src}
\end{code}
%
\vspace{5mm}
Programmierern, die mehr Kontrolle über die Initialisierung des Beschleunigers
oder der gesamten SYCL"=Laufzeitumgebung wünschen, stellt die
SYCL"=Spezifikation das aus \gls{opencl} bekannte Schema zur Verfügung. Der
Programmierer kann zunächst eine Liste aller zur Verfügung stehenden
\gls{opencl}"=Plattformen anfordern, aus denen er frei wählen kann. Auf dem
Fundament der gewählten Plattform erzeugt der Programmierer im nächsten Schritt
einen SYCL"=Kontext (der einen \gls{opencl}"=Kontext kapselt). Der Kontext
stellt wiederum eine Liste aller \textit{Devices} der Plattform bereit, aus der
ein oder mehrere Beschleuniger ausgesucht werden können. Die Auswahl dient dann
als Parameter für die Erzeugung einer SYCL"=\textit{Queue}. In jedem der
genannten Schritte stehen dem Programmierer zahlreiche Informationen über die
jeweilige Klasse zur Verfügung (Hersteller der Plattform oder des
Beschleunigers, Hardware"=Eigenschaften, verfügbare Erweiterungen, usw.), die
eine Verfeinerung der Auswahl erlauben.

Der Quelltext~\ref{sycl:ueberblick:axpy:queue:ausfuehrlich} zeigt das
ausführliche Schema. In den folgenden Abschnitten wird jedoch die oben gezeigte,
einfachere und kürzere Variante verwendet.

\begin{code}
    \begin{minted}[fontsize=\small]{c++}
auto platforms = cl::sycl::platform::get_platforms();
auto my_platform = /* ... */;

auto context = cl::sycl::context{my_platform};

auto devices = context.get_devices();
auto my_device = /* ... */;

auto queue = cl::sycl::queue{my_device};
    \end{minted}
    \caption{Ausführliche Beschleunigerwahl und \textit{Queue}"=Konstruktion}
    \label{sycl:ueberblick:axpy:queue:ausfuehrlich}
\end{code}

\subsection{Speicherreservierung und -initialisierung}
\label{sycl:ueberblick:axpy:buffer}

Für die Vektoren $\vec{x}$ und $\vec{y}$ der
Formel~\ref{sycl:ueberblick:axpy:formel} ist eine Speicherreservierung auf dem
\textit{Device} sowie die Initialisierung des reservierten Speichers notwendig
(die Konstante $a$ kann als einfacher Parameter übergeben werden). SYCL stellt
dafür zwei Klassen bereit:
%
\begin{itemize}
    \item Ein \texttt{buffer} kapselt die auf dem \textit{Device} reservierten
          Speicherbereiche. Dabei ist zu beachten, dass ein \texttt{buffer}
          keinem \textit{Device} direkt zugeordnet ist, sondern dem gesamten
          Kontext zur Verfügung steht. Ein \texttt{buffer} kann so auch auf
          mehreren \textit{Devices} verwendet werden. Die notwendige
          Synchronisierung wird von der SYCL"=Laufzeitumgebung vorgenommen.
    \item Ein \texttt{accessor} sorgt für den Zugriff auf den von einem
          \texttt{buffer} verwalteten Speicher. Es existieren verschiedene
          \texttt{accessor}"=Typen, darunter auch einer für den Speicherzugriff
          auf der \textit{Host}"=Seite. Mit diesem lässt sich ein
          \texttt{buffer} direkt initialisieren, ohne eine explizite Kopie
          anstoßen zu müssen. Aus Sicht des Programmierers lässt sich ein
          \texttt{accessor} wie ein Zeiger oder Feld verwenden, d.h. über den
          \texttt{[]}"=Operator.
\end{itemize}
%
\noindent
Ein \texttt{buffer} (im Folgenden als Puffer bezeichnet) besteht aus einer
endlichen Anzahl von Elementen desselben Typs und kann ein-, zwei- oder
dreidimensional sein. Der Elemente"=Typ sowie die Dimension des Puffers sind als
Template"=Parameter zur Compile"=Zeit anzugeben, während die Anzahl als
Laufzeit"=Parameter übergeben wird. Ein \texttt{accessor} lässt sich über die
Methode \texttt{get\_access} der \texttt{buffer}"=Klasse erzeugen. Dabei wird
als Template"=Parameter der gewünschte Zugriffstyp angegeben. Diese Information
wird von der SYCL"=Laufzeitumgebung zur Sortierung der Abhängigkeiten zwischen
Operationen auf dem \textit{Device} genutzt (siehe auch
Abschnitt~\ref{sycl:konzepte:abhaengigkeiten}).

SYCL unterscheidet sechs verschiedene Zugriffstypen:
\begin{itemize}
    \item \texttt{read} gestattet ausschließlich lesenden Zugriff auf den
          Puffer.
    \item \texttt{write} ermöglicht ausschließlich schreibenden Zugriff.
    \item Durch \texttt{read\_write} kann sowohl lesend als auch schreibend
          auf den Puffer zugegriffen werden.
    \item \texttt{discard\_write} ermöglicht ausschließlich schreibenden Zugriff
          und verwirft alle vorher im Puffer enthaltenen Elemente (also auch bei
          partiellem Zugriff).
    \item \texttt{discard\_read\_write} ist die Kombination aus
          \texttt{read\_write} und \texttt{discard\_write}.
    \item \texttt{atomic} ermöglicht atomaren Zugriff bei paralleler Nutzung des
          Puffers.
\end{itemize}

Für das AXPY"=Beispiel lassen sich die benötigten Felder -- wie in
Quelltext~\ref{sycl:ueberblick:axpy:buffer:src} dargestellt -- anlegen und
initialisieren.
%
\begin{code}
    \begin{minted}[fontsize=\small]{c++}
// Puffer enthalten 1024 Elemente
const auto buf_range = cl::sycl::range<1>{1024};

// erzeuge eindimensionale Puffer für int-Elemente
auto buf_x = cl::sycl::buffer<int, 1>{buf_range};
auto buf_y = cl::sycl::buffer<int, 1>{buf_range};

// greife auf x und y zu, verwirf vorherige Elemente
auto h_acc_x = buf_x.get_access<cl::sycl::access::mode::discard_write>();
auto h_acc_y = buf_x.get_access<cl::sycl::access::mode::discard_write>();

// initialisiere x und y
for(auto i = 0; i < 1024; ++i)
{
    h_acc_x[i] = /* ... */;
    h_acc_y[i] = /* ... */;
}
    \end{minted}
    \caption{Speicherreservierung und -initialisierung in SYCL}
    \label{sycl:ueberblick:axpy:buffer:src}
\end{code}

\subsection{\textit{Kernel}-Definition und -ausführung}
\label{sycl:ueberblick:axpy:kernel}

Im nächsten Schritt wird der eigentliche \textit{Kernel} definiert und
ausgeführt. Ein SYCL"=\textit{Kernel} besteht aus zwei Teilen: Der eigentlichen
\textit{Kernel}"=Funktion, also der Abbildung des Algorithmus auf
SYCL"=C++"=Quelltext, sowie den Abhängigkeiten (in Form von
\texttt{accessor}"=Variablen). \textit{Kernel}"=Funktion und Abhängigkeiten
bilden gemeinsam eine \textit{command group} und werden in dieser Form an die
\textit{Queue} zur Ausführung übergeben. Dabei kann jede \textit{command group}
nur genau eine \textit{Kernel}"=Funktion (oder explizite Kopieroperation)
enthalten. Es bildet sich damit für das AXPY"=Beispiel das in
Quelltext~\ref{sycl:ueberblick:axpy:kernel:cg} gezeigte Grundgerüst einer
\textit{command group}, welche in diesem Fall als C++"=Lambdafunktion notiert
wird.

\begin{code}
    \begin{minted}[fontsize=\small]{c++}
[&](cl::sycl::handler& cgh)
{
    auto d_acc_x = buf_x.get_access<cl::sycl::access::mode::read>(cgh);
    auto d_acc_y = buf_y.get_access<cl::sycl::access::mode::read_write>(cgh);

    // Kernel-Funktionsaufruf
}
    \end{minted}
    \caption{Struktur einer \textit{command group}}
    \label{sycl:ueberblick:axpy:kernel:cg}
\end{code}
\vspace{5mm}
SYCL bietet für verschiedene Anwendungsfälle unterschiedliche Methoden zum
Aufrufen der \textit{Kernel}"=Funktion. Für datenparallele Algorithmen bietet sich
vor allem der Aufruf mittels der Methode \texttt{parallel\_for} an. Diese
entspricht dem aus \gls{opencl} bekannten \textit{NDRange}"=\textit{Kernel} und
nutzt die SIMD\footnote{Das ist \gls{simd}, eine von Flynn definierte Klasse
innerhalb seiner Taxonomie \cite[vgl.][]{flynn1966}.}"=Eigenschaften der zur
Verfügung stehenden Beschleuniger. Auf CPUs können so durch einen Aufruf mehrere
Kerne und deren SIMD"=Register genutzt werden oder auf einer GPU die
Multiprozessoren und SIMD"=Einheiten. Durch \texttt{parallel\_for} kann auf
einem FPGA ebenfalls eine SIMD"=Schaltung synthetisiert werden. 

Für aufgabenparallele Algorithmen steht in SYCL der Aufruf \texttt{single\_task}
zur Verfügung, was einem \textit{Task}"=\textit{Kernel} in \gls{opencl}
entspricht. Dieser wird z.B. auf einer CPU nur auf einem einzelnen Kern
ausgeführt. Mehrere \textit{Kernel} dieses Typs lassen sich dann parallel auf
den vorhandenen Kernen ausführen.

Für das AXPY"=Beispiel bietet sich der datenparallele Fall an, weshalb die
\textit{Kernel}"=Funktion mittels \texttt{parallel\_for} aufgerufen wird (siehe
Quelltext~\ref{sycl:ueberblick:axpy:kernel:call}). Die \num{1024} Elemente der
Vektoren werden dabei als Arbeitsgröße mit übergeben. Die SYCL"=Laufzeitumgebung
generiert daraus einen Ausführungsraum mit \num{1024} \textit{work-items}, einer
Abstraktion der zugrundeliegenden Hardware"=Features (SIMD"=Register, Threads,
usw.). Das jeweilige \textit{work-item} wird als Parameter an die
\textit{Kernel}"=Funktion übergeben. Es kapselt unter anderem einen Index, der
für den Zugriff auf ein Element im Speicher verwendet werden kann. 

\begin{code}
    \begin{minted}[fontsize=\small]{c++}
[&](cl::sycl::handler& cgh)
{
    auto d_acc_x = buf_x.get_access<cl::sycl::access::mode::read>(cgh);
    auto d_acc_y = buf_y.get_access<cl::sycl::access::mode::read_write>(cgh);

    cgh.parallel_for<class axpy>(cl::sycl::range<1>{1024},
    [=](cl::sycl::item<1> work_item)
    {
        auto idx = work_item.get_id();
        d_acc_y[idx] = a * d_acc_x[idx] + d_acc_y[idx];
    });
}
    \end{minted}
    \caption{Struktur einer \textit{command group} mit \textit{Kernel}"=Aufruf}
    \label{sycl:ueberblick:axpy:kernel:call}
\end{code}

\subsection{Synchronisierung}
\label{sycl:ueberblick:axpy:sync}

Nachdem der \textit{Kernel} an die \textit{Queue} übergeben wurde, muss das
Ergebnis überprüft werden. Um darauf zugreifen zu können, ist zunächst die
Synchronisierung von \textit{Host} und \textit{Device} erforderlich, da beide
asynchron zueinander arbeiten. Die \textit{Queue} verfügt jedoch über die
Methode \texttt{wait}, die den \textit{Host} so lange warten lässt, bis alle
bereits eingereihten Befehle abgearbeitet wurden. Dies ist in
Quelltext~\ref{sycl:ueberblick:axpy:sync:wait} dargestellt. Anschließend
lassen sich die Elemente des Vektors $\vec{y}$ auf der \textit{Host}"=Seite
über den während der Initialisierung der Puffer angelegten \texttt{accessor}
überprüfen.

\begin{code}
    \begin{minted}[fontsize=\small]{c++}
queue.wait();
    \end{minted}
    \caption{Synchronisierung einer SYCL-\textit{Queue}}
    \label{sycl:ueberblick:axpy:sync:wait}
\end{code}

\subsection{Zusammenfassung}
\label{sycl:ueberblick:axpy:zusammenfassung}

Das gesamte SYCL"=AXPY"=Beispiel findet sich in
Quelltext~\ref{sycl:ueberblick:axpy:zusammenfassung:code}, einschließlich
einiger unwesentlicher in den vorigen Abschnitten ausgelassener Details.

\begin{code}
    \begin{minted}[fontsize=\small]{c++}
#include <cstdlib>
#include <CL/sycl.hpp>

class XOCLDeviceSelector : public cl::sycl::device_selector {
    public:
        int operator()(const cl::sycl::device &Device) const override {
            const std::string DeviceVendor =
                            Device.get_info<cl::sycl::info::device::vendor>();
            return (DeviceVendor.find("Xilinx") != std::string::npos) ? 1 : -1;
        }
};

auto main() -> int {
    constexpr auto a = 42;

    // Beschleunigerwahl und Befehlswarteschlange
    auto queue = cl::sycl::queue{XOCLDeviceSelector{}};

    // Speicherreservierung und -initialisierung
    const auto buf_range = cl::sycl::range<1>{1024};

    auto buf_x = cl::sycl::buffer<int, 1>{buf_range};
    auto buf_y = cl::sycl::buffer<int, 1>{buf_range};

    auto h_acc_x = buf_x.get_access<cl::sycl::access::mode::discard_write>();
    auto h_acc_y = buf_x.get_access<cl::sycl::access::mode::discard_write>();

    for(auto i = 0; i < 1024; ++i)
    {
        h_acc_x[i] = /* ... */;
        h_acc_y[i] = /* ... */;
    }

    // Kerneldefinition und -ausführung
    queue.submit([&](cl::sycl::handler& cgh)
    {
        auto d_acc_x = buf_x.get_access<cl::sycl::access::mode::read>(cgh);
        auto d_acc_y = buf_y.get_access<cl::sycl::access::mode::read_write>(cgh);

        cgh.parallel_for<class axpy>(cl::sycl::range<1>{1024},
        [=](cl::sycl::item<1> work_item)
        {
            auto idx = work_item.get_id();
            d_acc_y[idx] = a * d_acc_x[idx] + d_acc_y[idx];
        });
    });

    // Synchronisierung
    queue.wait();

    // Zugriff auf h_acc_x und h_acc_y ab hier wieder möglich

    return EXIT_SUCCESS;
}
    \end{minted}
    \caption{AXPY -- vollständiges SYCL"=Beispiel}
    \label{sycl:ueberblick:axpy:zusammenfassung:code}
\end{code}

\section{Weiterführende Konzepte}\label{sycl:konzepte}

Die Entwicklung komplexerer Programme mit SYCL erfordert die Kenntnis einiger
weiterer Konzepte, die in der obigen Einführung nicht berücksichtigt wurden.
Dazu zählen die in SYCL vorhandene Hardware"=Abstraktion sowie die
Abhängigkeiten zwischen \textit{Kerneln}. Für die Analyse des entwickelten
Programms sind außerdem SYCLs Fähigkeiten zur Fehlerbehandlung und zum Profiling
relevant.

\subsection{Hardware"=Abstraktion}

Um eine bessere Anpassung des Programms auf die genutzte Hardware zu
ermöglichen, ohne die Plattformunabhängigkeit aufzugeben, führte die
\gls{opencl}"=Spezifikation eine Reihe von Abstraktionen ein. Diese entsprechen
konzeptionell den in der Hardware vorhandenen Fähigkeiten und wurden ebenfalls
von SYCL übernommen.

Eine \textit{Plattform} ist in OpenCL und SYCL aus dem \textit{Host} und
mindestens einem \textit{Device} zusammengesetzt. Jedes \textit{Device} besteht
wiederum aus mindestens einer \textit{compute unit} (CU). Eine CU lässt sich auf
einen oder mehrere Teile des Beschleunigers abbilden und ist in der Lage, einen
\textit{Kernel} auszuführen. Bei einer CPU lässt sich eine CU also auf einen
Kern abbilden oder bei einer GPU auf einen Multiprozessor. Bei FPGAs ist die
Abbildung dynamischer: Hier hängt die Zahl der verfügbaren CUs davon ab, wie
viele Ressourcen der \textit{Kernel} verbraucht. Die Zahl der gleichzeitig
platzierbaren \textit{Kernel} entspricht damit der Zahl der möglichen CUs,
sofern die Implementierung keine Obergrenze für die CU"=Anzahl vorgibt. Eine CU
besteht aus mindestens einem \textit{processing element} (PE). Ein PE lässt sich
dabei als Abstraktion der SIMD"=Fähigkeiten einer CU verstehen, also z.B. als
ein Element innerhalb eines SIMD"=Vektorregisters.
Die Abbildung~\ref{sycl:konzepte:abstraktion:plattform} veranschaulicht dieses
Modell.

\begin{figure}
    \centering
    \begin{tikzpicture}
        % hinterstes Device
        \draw [fill = white] (2.499, 2.499) rectangle ++(5, 2.5);
        \draw [fill = white] (4.249, 3.416) rectangle ++(3, 1.333);
        \draw [fill = HKS41!20] (4.349, 3.516) rectangle ++(0.2, 1.133);
        \draw [fill = HKS41!20] (4.749, 3.516) rectangle ++(0.2, 1.133);
        \draw [fill = HKS41!20] (5.149, 3.516) rectangle ++(0.2, 1.133);
        \draw [fill = HKS41!20] (6.949, 3.516) rectangle ++(0.2, 1.133);
        \draw [fill = white] (3.499, 3.0825) rectangle ++(3, 1.333);
        \draw [fill = HKS41!20] (3.599, 3.1825) rectangle ++(0.2, 1.133);
        \draw [fill = HKS41!20] (3.999, 3.1825) rectangle ++(0.2, 1.133);
        \draw [fill = HKS41!20] (4.399, 3.1825) rectangle ++(0.2, 1.133);
        \draw [fill = HKS41!20] (6.199, 3.1825) rectangle ++(0.2, 1.133);
        \draw [fill = white] (2.749, 2.749) rectangle ++(3, 1.333);
        \draw [fill = HKS41!20] (2.849, 2.849) rectangle ++(0.2, 1.133);
        \draw [fill = HKS41!20] (3.249, 2.849) rectangle ++(0.2, 1.133);
        \draw [fill = HKS41!20] (3.649, 2.849) rectangle ++(0.2, 1.133);
        \draw [fill = HKS41!20] (5.449, 2.849) rectangle ++(0.2, 1.133);

        % erstes von hinten
        \draw [fill = white] (1.666, 1.666) rectangle ++(5, 2.5);
        \draw [fill = white] (3.416, 2.583) rectangle ++(3, 1.333);
        \draw [fill = HKS41!20] (3.516, 2.683) rectangle ++(0.2, 1.133);
        \draw [fill = HKS41!20] (3.916, 2.683) rectangle ++(0.2, 1.133);
        \draw [fill = HKS41!20] (4.316, 2.683) rectangle ++(0.2, 1.133);
        \draw [fill = HKS41!20] (6.116, 2.683) rectangle ++(0.2, 1.133);
        \draw [fill = white] (2.666, 2.2495) rectangle ++(3, 1.333);
        \draw [fill = HKS41!20] (2.766, 2.3495) rectangle ++(0.2, 1.133);
        \draw [fill = HKS41!20] (3.166, 2.3495) rectangle ++(0.2, 1.133);
        \draw [fill = HKS41!20] (3.566, 2.3495) rectangle ++(0.2, 1.133);
        \draw [fill = HKS41!20] (5.366, 2.3495) rectangle ++(0.2, 1.133);
        \draw [fill = white] (1.916, 1.916) rectangle ++(3, 1.333);
        \draw [fill = HKS41!20] (2.016, 2.016) rectangle ++(0.2, 1.133);
        \draw [fill = HKS41!20] (2.416, 2.016) rectangle ++(0.2, 1.133);
        \draw [fill = HKS41!20] (2.816, 2.016) rectangle ++(0.2, 1.133);
        \draw [fill = HKS41!20] (4.616, 2.016) rectangle ++(0.2, 1.133);

        % zweites von hinten
        \draw [fill = white] (0.833, 0.833) rectangle ++(5, 2.5);
        \draw [fill = white] (2.583, 1.75) rectangle ++(3, 1.333);
        \draw [fill = HKS41!20] (2.683, 1.85) rectangle ++(0.2, 1.133);
        \draw [fill = HKS41!20] (3.083, 1.85) rectangle ++(0.2, 1.133);
        \draw [fill = HKS41!20] (3.483, 1.85) rectangle ++(0.2, 1.133);
        \draw [fill = HKS41!20] (5.283, 1.85) rectangle ++(0.2, 1.133);
        \draw [fill = white] (1.833, 1.4165) rectangle ++(3, 1.333);
        \draw [fill = HKS41!20] (1.933, 1.5165) rectangle ++(0.2, 1.133);
        \draw [fill = HKS41!20] (2.333, 1.5165) rectangle ++(0.2, 1.133);
        \draw [fill = HKS41!20] (2.733, 1.5165) rectangle ++(0.2, 1.133);
        \draw [fill = HKS41!20] (4.533, 1.5165) rectangle ++(0.2, 1.133);
        \draw [fill = white] (1.083, 1.083) rectangle ++(3, 1.333);
        \draw [fill = HKS41!20] (1.183, 1.183) rectangle ++(0.2, 1.133);
        \draw [fill = HKS41!20] (1.583, 1.183) rectangle ++(0.2, 1.133);
        \draw [fill = HKS41!20] (1.983, 1.183) rectangle ++(0.2, 1.133);
        \draw [fill = HKS41!20] (3.783, 1.183) rectangle ++(0.2, 1.133);

        % vorderstes Device
        \draw [fill = white] (0.0, 0.0) rectangle ++(5, 2.5);
        \draw [fill = white] (1.75, 0.917) rectangle ++(3, 1.333);
        \draw [fill = HKS41!20] (1.85, 1.017) rectangle ++(0.2, 1.133);
        \draw [fill = HKS41!20] (2.25, 1.017) rectangle ++(0.2, 1.133);
        \draw [fill = HKS41!20] (2.65, 1.017) rectangle ++(0.2, 1.133);
        \draw [fill = HKS41!20] (4.45, 1.017) rectangle ++(0.2, 1.133);
        \draw [fill = white] (1.0, 0.5835) rectangle ++(3, 1.333);
        \draw [fill = HKS41!20] (1.1, 0.6835) rectangle ++(0.2, 1.133);
        \draw [fill = HKS41!20] (1.5, 0.6835) rectangle ++(0.2, 1.133);
        \draw [fill = HKS41!20] (1.9, 0.6835) rectangle ++(0.2, 1.133);
        \draw [fill = HKS41!20] (3.7, 0.6835) rectangle ++(0.2, 1.133);
        \draw [fill = white] (0.25, 0.25) rectangle ++(3, 1.333);
        \draw [fill = HKS41!20] (0.35, 0.35) rectangle ++(0.2, 1.133);
        \draw [fill = HKS41!20] (0.75, 0.35) rectangle ++(0.2, 1.133);
        \draw [fill = HKS41!20] (1.15, 0.35) rectangle ++(0.2, 1.133);
        \draw [fill = HKS41!20] (2.95, 0.35) rectangle ++(0.2, 1.133);
        % Punkte sind nur hier notwendig, Rest wird überdeckt
        \draw [fill = black] (1.825, 0.9165) circle [radius = 0.5mm];
        \draw [fill = black] (2.125, 0.9165) circle [radius = 0.5mm];
        \draw [fill = black] (2.425, 0.9165) circle [radius = 0.5mm];

        % Host
        \draw (10.0, 1.2497) rectangle (13.0, 3.7493)
                      node[pos = 0.5, align = center] {Host};

        % Verbindungen
        \draw [line width = 1mm] (5, 1.25) -- (6, 1.25) -- (8.499, 3.749) -- (7.499, 3.749);
        \draw [line width = 1mm] (5.833, 2.083) -- (6.833, 2.083);
        \draw [line width = 1mm] (6.666, 2.916) -- (7.666, 2.916);
        \draw [line width = 1mm] (10.0, 2.4995) -- (7.25, 2.4995);

        % Beschriftungen
        \draw (5, 0.25) -- (6.666, -0.3)
                      node[pos = 1.0, align = center, xshift = 6mm] {Device};
        \draw (2.125, 0.25) -- (0.5, -0.75)
                      node[pos = 1.0, align = center, yshift = -2mm] {Compute Unit};
        \draw (0.45, 0.75) -- (-1, 2.35)
                      node[pos = 1.0, align = left, xshift = -6mm] {Processing\\Element};
    \end{tikzpicture}
    \caption[SYCLs Plattform-Modell]
            {SYCLs Plattform-Modell \cite[nach][23]{opencl2012}}
    \label{sycl:konzepte:abstraktion:plattform}
\end{figure}

Um die Parallelität mehrerer CUs nutzen zu können, ist es erforderlich, die
Arbeit des \textit{Kernels} aufzuteilen. Bei acht verfügbaren CUs wäre es daher
wünschenswert, die Berechnungen in mindestens acht Blöcken (oder einem
Vielfachen davon) parallel durchzuführen. Diese Aufteilung wird in \gls{opencl}
und SYCL \textit{work"=group} genannt. \textit{Work"=groups} werden durch ihre
Zuordnung zu unterschiedlichen CUs asynchron zueinander ausgeführt und eine
Synchronisierung der Gruppen ist nicht ohne weiteres möglich.

Eine \textit{work"=group} besteht aus mindestens einem \textit{work"=item},
wobei die Implementierung auch eine maximale Anzahl von \textit{work"=items}
festlegen kann. Ein \textit{work"=item} wird während der Ausführung einem PE
zugeteilt. \textit{Work"=items} werden zueinander asynchron ausgeführt, lassen
sich jedoch über Funktionen der gemeinsamen \textit{work"=group}
synchronisieren. Es ist jedoch nicht möglich, \textit{work"=items} verschiedener
\textit{work"=groups} direkt über das SYCL"=Interface zu synchronisieren. Diese
können nur über den globalen Speicher oder über atomare Funktionen miteinander
kommunizieren.

Durch diese Hardware"=Abstraktion wird aus Sicht des Programmierers ein
Indexraum aufgespannt -- in OpenCL und SYCL \textit{NDRange} genannt --, in dem
jedem \textit{work"=item} ein eindeutiger Index innerhalb der
\textit{work"=group} sowie der Menge aller \textit{work"=items} zugewiesen wird.
Diese Indizes werden als lokale bzw. globale Indizes bezeichnet. Die
Abbildung~\ref{sycl:konzepte:abstraktion:ndrange} zeigt diese Aufteilung am
Beispiel einer zweidimensionalen \textit{NDRange}. Dabei stehen $G_x$ und 
$G_y$ für die Gesamtzahl der \textit{work"=items} sowie $S_x$ und $S_y$ für die
Zahl der \textit{work"=items} pro \textit{work"=group}, jeweils in x- und
y-Richtung. $w_x$ und $w_y$ bezeichnen die Position der \textit{work"=group}
innerhalb der \textit{NDRange}, während $s_x$ und $s_y$ die Position eines
\textit{work"=items} in der \textit{work"=group} -- also den lokalen Index --
darstellen. Der globale Index $(g_x, g_y)$ eines \textit{work"=items} lässt sich
demnach wie folgt berechnen \cite[vgl.][24]{opencl2012}:
\[
    (g_x, g_y) = (w_x \cdot S_x + s_x, w_y \cdot S_y + s_y)
\]
Die Zahl $(W_x, W_y)$ der \textit{work"=groups} innerhalb der \textit{NDRange}
lässt sich ebenfalls bestimmen \cite[vgl.][25]{opencl2012}:
\[
    (W_x, W_y) = \left(\frac{G_x}{S_x}, \frac{G_y}{S_y}\right)
\]

\begin{figure}
    \centering
    \begin{tikzpicture}
        % NDRange
        \draw (0, 0) rectangle ++(4, 4);

        \draw (0.4, 0.4) rectangle ++(1, 1);
        \draw (0.65, 0.4) -- ++(0, 1);
        \draw (0.9, 0.4) -- ++(0, 1);
        \draw (1.15, 0.4) -- ++(0, 1);
        \draw (0.4, 0.65) -- ++(1, 0);
        \draw (0.4, 0.9) -- ++(1, 0);
        \draw (0.4, 1.15) -- ++(1, 0);

        \draw (1.5, 0.4) rectangle ++(1, 1);
        \draw (1.75, 0.4) -- ++(0, 1);
        \draw (2, 0.4) -- ++(0, 1);
        \draw (2.25, 0.4) -- ++(0, 1);
        \draw (1.5, 0.65) -- ++(1, 0);
        \draw (1.5, 0.9) -- ++(1, 0);
        \draw (1.5, 1.15) -- ++(1, 0);

        \draw (2.6, 0.4) rectangle ++(1, 1);
        \draw (2.85, 0.4) -- ++(0, 1);
        \draw (3.1, 0.4) -- ++(0, 1);
        \draw (3.35, 0.4) -- ++(0, 1);
        \draw (2.6, 0.65) -- ++(1, 0);
        \draw (2.6, 0.9) -- ++(1, 0);
        \draw (2.6, 1.15) -- ++(1, 0);
        
        \draw (0.4, 1.5) rectangle ++(1, 1);
        \draw (0.65, 1.5) -- ++(0, 1);
        \draw (0.9, 1.5) -- ++(0, 1);
        \draw (1.15, 1.5) -- ++(0, 1);
        \draw (0.4, 1.75) -- ++(1, 0);
        \draw (0.4, 2) -- ++(1, 0);
        \draw (0.4, 2.25) -- ++(1, 0);

        \draw (1.5, 1.5) rectangle ++(1, 1);
        \draw (1.75, 1.5) -- ++(0, 1);
        \draw (2, 1.5) -- ++(0, 1);
        \draw (2.25, 1.5) -- ++(0, 1);
        \draw (1.5, 1.75) -- ++(1, 0);
        \draw (1.5, 2) -- ++(1, 0);
        \draw (1.5, 2.25) -- ++(1, 0);

        \draw (2.6, 1.5) rectangle ++(1, 1);
        \draw (2.85, 1.5) -- ++(0, 1);
        \draw (3.1, 1.5) -- ++(0, 1);
        \draw (3.35, 1.5) -- ++(0, 1);
        \draw (2.6, 1.75) -- ++(1, 0);
        \draw (2.6, 2) -- ++(1, 0);
        \draw (2.6, 2.25) -- ++(1, 0);

        \draw (0.4, 2.6) rectangle ++(1, 1);
        \draw (0.65, 2.6) -- ++(0, 1);
        \draw (0.9, 2.6) -- ++(0, 1);
        \draw (1.15, 2.6) -- ++(0, 1);
        \draw (0.4, 2.85) -- ++(1, 0);
        \draw (0.4, 3.1) -- ++(1, 0);
        \draw (0.4, 3.35) -- ++(1, 0);

        \draw (1.5, 2.6) rectangle ++(1, 1);
        \draw (1.75, 2.6) -- ++(0, 1);
        \draw (2, 2.6) -- ++(0, 1);
        \draw (2.25, 2.6) -- ++(0, 1);
        \draw (1.5, 2.85) -- ++(1, 0);
        \draw (1.5, 3.1) -- ++(1, 0);
        \draw (1.5, 3.35) -- ++(1, 0);

        \draw (2.6, 2.6) rectangle ++(1, 1);
        \draw (2.85, 2.6) -- ++(0, 1);
        \draw (3.1, 2.6) -- ++(0, 1);
        \draw (3.35, 2.6) -- ++(0, 1);
        \draw (2.6, 2.85) -- ++(1, 0);
        \draw (2.6, 3.1) -- ++(1, 0);
        \draw (2.6, 3.35) -- ++(1, 0);

        \draw [Latex-Latex] (-0.4, 0) -- ++(0, 4)
            node [pos = 0.5, left] {$G_y$};
        \draw [Latex-Latex] (0, -0.4) -- ++(4, 0)
            node [pos = 0.5, below] {$G_x$};

        % work-group
        \draw [dashed] (1.5, 1.5) -- (5, -0.2);
        \draw [dashed] (2.5, 1.5) -- (13, -0.2);
        \draw [dashed] (1.5, 2.5) -- (5, 8.3);
        \draw [dashed] (2.5, 2.5) -- (13, 8.3);
        \draw [fill = white] (5, -0.2) rectangle ++(8, 8.5);

        \node [align=center] (wgtext) at (9, 7.9) {\textbf{work-group}};

        \draw [Latex-Latex] (5, 8.7) -- ++(8, 0)
            node [pos = 0.5, above] {$S_x$};
        \draw [Latex-Latex] (13.4, -0.2) -- ++(0, 8.5)
            node [pos = 0.5, right] {$S_y$};

        % work-items
        \draw (5.25, 0.05) rectangle ++(3.25, 3.25)
            node [pos = 0.5, align=center] {\textbf{work-item}\\~\\
                                            {\small globaler Index:}\\
                                            ${\scriptstyle (w_x \cdot S_x + s_x, w_y \cdot S_y + s_y)}$\\
                                            {\small lokaler Index:}\\
                                            ${\scriptstyle (s_x, s_y) = (0, S_y - 1)}$};
        \draw (9.5, 0.05) rectangle ++(3.25, 3.25)
            node [pos = 0.5, align=center] {\textbf{work-item}\\~\\
                                            {\small globaler Index:}\\
                                            ${\scriptstyle (w_x \cdot S_x + s_x, w_y \cdot S_y + s_y)}$\\
                                            {\small lokaler Index:}\\
                                            ${\scriptstyle (s_x, s_y) = (S_x - 1, S_y - 1)}$};
        \draw (5.25, 4.3) rectangle ++(3.25, 3.25)
            node [pos = 0.5, align=center] {\textbf{work-item}\\~\\
                                            {\small globaler Index:}\\
                                            ${\scriptstyle (w_x \cdot S_x + s_x, w_y \cdot S_y + s_y)}$\\
                                            {\small lokaler Index:}\\
                                            ${\scriptstyle (s_x, s_y) = (0, 0)}$};
        \draw (9.5, 4.3) rectangle ++(3.25, 3.25)
            node [pos = 0.5, align=center] {\textbf{work-item}\\~\\
                                            {\small globaler Index:}\\
                                            ${\scriptstyle (w_x \cdot S_x + s_x, w_y \cdot S_y + s_y)}$\\
                                            {\small lokaler Index:}\\
                                            ${\scriptstyle (s_x, s_y) = (S_x - 1, 0)}$};

        % Punkte
        \draw [fill = black] (8.8, 1.675) circle [radius = 0.5mm];
        \draw [fill = black] (9, 1.675) circle [radius = 0.5mm];
        \draw [fill = black] (9.2, 1.675) circle [radius = 0.5mm];

        \draw [fill = black] (8.8, 5.925) circle [radius = 0.5mm];
        \draw [fill = black] (9, 5.925) circle [radius = 0.5mm];
        \draw [fill = black] (9.2, 5.925) circle [radius = 0.5mm];

        \draw [fill = black] (6.875, 3.6) circle [radius = 0.5mm];
        \draw [fill = black] (6.875, 3.8) circle [radius = 0.5mm];
        \draw [fill = black] (6.875, 4.0) circle [radius = 0.5mm];

        \draw [fill = black] (11.125, 3.6) circle [radius = 0.5mm];
        \draw [fill = black] (11.125, 3.8) circle [radius = 0.5mm];
        \draw [fill = black] (11.125, 4.0) circle [radius = 0.5mm];

        \draw [fill = black] (8.8, 4.0) circle [radius = 0.5mm];
        \draw [fill = black] (9, 3.8) circle [radius = 0.5mm];
        \draw [fill = black] (9.2, 3.6) circle [radius = 0.5mm];
    \end{tikzpicture}
    \caption[SYCLs Indexraum]{SYCLs Indexraum \cite[nach][25]{opencl2012}}
    \label{sycl:konzepte:abstraktion:ndrange}
\end{figure}

Da eine \textit{NDRange} bis zu drei Dimensionen umfassen kann, lassen sich
mehrdimensionale Algorithmen auf diese Weise einfach implementieren.

\subsection{\textit{Kernel}-Start}
\label{sycl:konzepte:kernelstart}

SYCL unterstützt verschiedene Möglichkeiten, den \textit{Kernel} unter
Einbeziehung der im vorigen Abschnitt geschilderten Hierarchie aus
\textit{NDRange}, \textit{work"=groups} und \textit{work"=items} zu starten.

Der Aufruf eines \textit{Kernels} über die Funktion \texttt{parallel\_for()}
erzeugt eine \textit{NDRange} nach dem oben beschriebenen Muster. Der
Programmierer muss immer die Anzahl der \textit{work"=items} innerhalb der
\textit{NDRange} angeben, während die Angabe der Zahl der \textit{work"=items}
pro \textit{work"=group} optional ist. Sofern der Programmierer darauf
verzichtet die Größe der \textit{work"=groups} selbst festzulegen, ist es
Aufgabe der SYCL"=Implementierung, die Zahl der nötigen \textit{work"=groups} zu
bestimmen und die Zuordnung der \textit{work"=items} durchzuführen. Mit dieser
Entscheidung verliert der Programmierer jedoch die Möglichkeit gruppenweite
Operationen durchzuführen, wie z.B. die Synchronisierung innerhalb der
\textit{work"=group}.

Durch den Aufruf über die Funktion \texttt{parallel\_for\_work\_group()} hat der
Programmierer die Möglichkeit, seinen \textit{Kernel} hierarchisch zu gliedern. Der
\textit{Kernel} führt seine Anweisungen zunächst auf der Ebene der \textit{work"=group}
aus, das heißt, dass jede Anweisung von einem einzigen \textit{work"=item}
durchgeführt und das Ergebnis der gesamten \textit{work"=group} bekannt gemacht
wird. Innerhalb des \textit{Kernels} kann durch die Funktion
\texttt{parallel\_for\_work\_item()} die Parallelität auf der
\textit{work"=item}"=Ebene ausgedrückt werden: Anweisungen innerhalb dieses
\textit{Kernel}"=Teils werden individuell von einer Menge von \textit{work"=items}
ausgeführt. Diese kann alle \textit{work"=items} der \textit{work"=group}
umfassen oder nur einen Teil der \textit{work"=group}. Am Ende des
\textit{work"=item}"=Teils erfolgt eine implizite Synchronisierung der
\textit{work"=items}.

Sequentielle \textit{Kernel} lassen sich über die Funktion \texttt{single\_task()}
starten. Die aufgespannte \textit{NDRange} umfasst dann genau eine
\textit{work"=group} mit genau einem \textit{work"=item}. Durch den wiederholten
Aufruf von \texttt{single\_task()} lassen sich komplexe Ketten aufeinander
folgender sequentieller Aufgaben einfach modellieren.

\subsection{Abhängigkeiten zwischen \textit{Kerneln}}
\label{sycl:konzepte:abhaengigkeiten}

Alle \textit{command groups}, die der Programmierer in eine
SYCL"=\textit{Queue} einreiht, werden grundsätzlich asynchron ausgeführt, sofern
keine Abhängigkeiten zueinander bestehen. Vorhandene Abhängigkeiten werden über
die von den \textit{Kerneln} verwendeten Puffer durch die SYCL"=Laufzeitumgebung
automatisch erkannt und die \textit{Kernel} in der richtigen Reihenfolge ausgeführt
\cite[vgl.][21--23]{sycl2019}. In diesem Aspekt unterscheidet sich SYCL von
\gls{opencl}, das neben vollständig seriellen \textit{Queues} nur asynchrone
\textit{Queues} kennt, bei denen die richtige Sortierung der \textit{Kernel} Aufgabe des
Programmierers ist. Wichtig ist außerdem, dass die Abhängigkeiten über die
Puffer"=Verfügbarkeit ermittelt werden -- und nicht über das Ende der \textit{Kernel}
\cite[vgl.][166]{sycl2019}.

Graphisch lässt sich dies in Form eines gerichteten Graphen veranschaulichen,
wie für einen einfachen Fall in
Abbildung~\ref{sycl:konzepte:abhaengigkeiten:graph} gezeigt. Im Beispiel hängen
die \textit{command groups} B und C von der \textit{command group} A ab, sind
jedoch voneinander unabhängig. Dadurch müssen sie beide auf das Ende von A
warten, können danach jedoch parallel ausgeführt werden. Die
\textit{command group} D benötigt wiederum die in B und C berechneten
Ergebnisse und wartet daher auf deren Ende.

Quelltext~\ref{sycl:konzepte:abhaengigkeiten:src} zeigt die dem Graphen
entsprechende Verwendung einer SYCL"=\textit{Queue}. Wie man leicht sieht, ist
der Aufwand hinsichtlich der Abhängigkeitsverwaltung für den Programmierer sehr
gering.

\begin{figure}
    \centering
    \begin{tikzpicture}
        \draw (0, 0) circle [radius = 5mm] node {A};
        \draw (-1.5, -2.5) circle [radius = 5mm] node {B};
        \draw (1.5, -2.5) circle [radius = 5mm] node {C};
        \draw (0, -5) circle [radius = 5mm] node {D};

        \draw [-Latex] (0, -0.5) -- (-1.5, -2);
        \draw [-Latex] (0, -0.5) -- (1.5, -2);
        \draw [-Latex] (-1.5, -3) -- (0, -4.5);
        \draw [-Latex] (1.5, -3) -- (0, -4.5);
    \end{tikzpicture}
    \caption{Einfacher Aufgabengraph}
    \label{sycl:konzepte:abhaengigkeiten:graph}
\end{figure}

\begin{code}
    \begin{minted}[fontsize=\small]{c++}
queue.submit(/* A */);
queue.submit(/* B */);
queue.submit(/* C */);
queue.submit(/* D */);
    \end{minted}
    \caption{Einfacher SYCL-Aufgabengraph}
    \label{sycl:konzepte:abhaengigkeiten:src}
\end{code}

\subsection{Fehlerbehandlung}

SYCL übernimmt das System der Ausnahmefehler (engl. \textit{exceptions}) aus
der C++"=Standardbibliothek. Das Fehlersystem ist in SYCL asynchron:
grundsätzlich werden nur (synchrone) Fehler der \textit{Host}"=Seite ausgegeben,
während auf der \textit{Device}"=Seite aufgetretene Fehler ignoriert werden. Das
Abfangen der \textit{Device}"=Fehler erfordert einen weiteren Parameter für die
\textit{Queue}. Dieser ist eine Datenstruktur, welche die asynchronen Fehler der
\textit{Device}"=Seite abfängt und in synchrone \textit{Host}"=Fehler umwandt,
die dann vom Programmierer weiter verarbeitet werden können
(siehe Anhang~\ref{anhang:source:cpp:syclexceptions}).

\subsection{Profiling}

SYCL ermöglicht über die \textit{Queue} ein rudimentäres Profiling. Dieses
bietet dem Programmierer die Möglichkeit, über von der \textit{Queue} generierte
\textit{Events} Informationen über Start- und Endzeitpunkt der \textit{Kernel} sowie den
gegenwärtigen Ausführungsstand eines \textit{Kernels} zu erhalten. Dazu muss der
\textit{Queue} ein besonderer Parameter während der Konstruktion übergeben
werden (siehe Anhang~\ref{anhang:source:cpp:syclprofiling}).

\subsection{Referenz-Semantik}

Ein wichtiger Unterschied zur üblichen C++"=Programmierung sind SYCLs
Referenz"=Semantiken. Die Spezifikation schreibt vor
\cite[siehe][Abschnitt 4.3.2]{sycl2019}:
\begin{otherlanguage}{english}
    \begin{quote}
        Each of the following SYCL runtime classes: \texttt{device},
        \texttt{context}, \texttt{queue}, \texttt{program}, \texttt{kernel},
        \texttt{event}, \texttt{buffer}, \texttt{image}, \texttt{sampler},
        \texttt{accessor} and \texttt{stream} must obey the following
        statements, where \texttt{T} is the runtime class: [...]
        \\
        Any instance of \texttt{T} that is constructed as a copy of another
        instance, via either the copy constructor or copy assignment operator,
        must behave as-if it were the original instance and as-if any action
        performed on it were also performed on the original instance and if said
        instance is not a host object must represent and continue to represent
        the same underlying OpenCL objects as the original instance where
        applicable.
    \end{quote}
\end{otherlanguage}
Bemerkenswert ist, dass diese Semantik ebenfalls für die Typen \texttt{buffer}
und \texttt{image} gilt, das heißt Datentypen, die größere Speicherbereiche
kapseln. In der C++-Standardbibliothek werden die internen Felder vergleichbarer
Typen (wie \texttt{vector}) ebenfalls kopiert. Nach dem Kopiervorgang existieren
damit zwei voneinander verschiedene Objekte, die getrennte Speicherbereiche
verwalten. Die SYCL-Objekte beziehen sich jedoch nach dem Kopiervorgang auf den
selben Speicherbereich, es wird also bei der Objektkopie kein neuer Speicher
angelegt. De facto handelt es sich bei der Kopie eines SYCL-Objekts daher
lediglich um eine Referenz auf das ursprüngliche Objekt.

\section{Implementierungen}\label{sycl:implementierungen}

Der ersten Veröffentlichung der SYCL-Spezifikation im Mai 2015 folgten im Laufe
der Zeit einige konkrete Implementierungen verschiedener Anbieter. Diese werden
in den folgenden Abschnitten vorgestellt.

Darüber hinaus existiert eine von der Firma Codeplay betreute Internet-Seite,
die sich dem gesamten \gls{sycl}"=Ökosystem widmet. \cite[vgl.][]{sycltech}

\subsection{ComputeCpp}

Die schottische Firma Codeplay ist der zur Zeit einzige Anbieter einer
kommerziellen SYCL-Implementierung, die unter dem Namen \textit{ComputeCpp}
vermarktet wird. Sie richtet sich in erster Linie an Hardware für die Bereiche
Automotive und Embedded, unterstützt jedoch (bei einer bereits vorhandenen
OpenCL"=Implementierung) auch \gls{cpu}s und \gls{gpu}s der Firma Intel sowie
(experimentell) NVIDIA-\gls{gpu}s. Nach vorheriger Registrierung ist für
nichtkommerzielle Zwecke auch eine kostenlose \textit{community edition}
verfügbar. \cite[vgl.][]{computecpp}

\subsection{Intel}

Eine wichtige quelloffene Implementierung kommt von der Firma Intel. Strategisch
soll diese Implementierung mit dem Compiler \textit{clang} des LLVM"=Projekts
vereinigt werden. Zur Zeit handelt es sich jedoch noch um eine eigenständige
Implementierung, die vor allem auf die Intel"=\gls{opencl}"=Implementierungen
für \gls{cpu}s und \gls{gpu}s abzielt. Aktivitäten innerhalb des öffentlich
einsehbaren Quelltext"=Repositoriums deuten jedoch darauf hin, dass auch die
eigenen \gls{fpga}s unterstützt werden sollen. \cite[vgl.][]{intelsycl}

\subsection{triSYCL}

Das Projekt triSYCL ist eine quelloffene Implementierung des
\gls{sycl}"=Standards, die früher von der Firma AMD und jetzt von Xilinx
entwickelt wird. Nach eigener Aussage dient es vornehmlich experimentellen
Zwecken, um dem \gls{sycl}"=Komitee und dem \gls{opencl}"=C++"=Komitee des
Khronos"=Konsortiums sowie dem C++"=Standardisierungskomitee der \gls{iso}
Feedback liefern zu können. Das Hauptprojekt unterstützt \gls{cpu}s (über \gls{openmp}
oder \gls{tbb}) sowie \gls{opencl}"=Implementierungen, die die Verarbeitung des
\gls{spir}"=Zwischencodes unterstützen. Die meisten SYCL"=Features werden
unterstützt, jedoch sind einige SYCL"=Klassen noch nicht vollständig
implementiert.\cite[vgl.][]{trisycl}

Daneben existiert unter dem Dach des triSYCL"=Projekts ein von der
Intel"=Implementierung abgeleitetes Compiler"=Projekt, das sich vornehmlich der
besseren Unterstützung von Xilinx"=\gls{fpga}s anzunehmen scheint. Dieser
Compiler wird im Folgenden der Einfachheit halber als Xilinx"=Implementierung
bezeichnet. \cite[vgl.][]{trisyclclang}

\subsection{hipSYCL}

Der Heidelberger Doktorand Aksel Alpay ist der Autor einer weiteren
SYCL"=Implementierung. Diese setzt auf dem CUDA"=Klon der Firma AMD auf, der
\gls{gpgpu}"=Sprache \gls{hip}. \gls{hip} ist sowohl auf AMD- als auch auf
NVIDIA"=\gls{gpu}s ausführbar. Dadurch können auch mit hipSYCL entwickelte
Programme auf diesen \gls{gpu}s ausgeführt werden. hipSYCL war über weite
Strecken ein Ein-Mann-Projekt, erst seit Februar 2019 ist die regelmäßige
Mitarbeit eines weiteren Entwicklers zu verzeichnen. Aus diesem Grund ist
hipSYCL unvollständig implementiert. Es fehlen unter anderem atomare Funktionen
oder die Möglichkeit, Ausnahmefehler zu werfen und abzufangen.
\cite[vgl.][]{hipsycl}

\subsection{sycl-gtx}

Eine weitere Open-Source-Implementierung ist das in der Einleitung erwähnte
\textit{sycl"=gtx}. Ursprünglich ist diese Implementierung im Rahmen einer
Masterarbeit entstanden \cite[vgl.][]{zuzek2016} und wird bis heute vom
ursprünglichen Autoren weiterentwickelt. Aufgrund der begrenzten
Entwicklerkapazitäten ist diese Variante aber immer noch sehr rudimentär und
unterstützt nur eine Teilmenge der SYCL"=Spezifikation.

Im Gegensatz zu den anderen Implementierungen wird der SYCL"=\gls{kernel} erst
zur Laufzeit des kompilierten Programms in einen \gls{opencl}"=\gls{kernel}
umgewandelt und anschließend an die zugrundeliegende
\gls{opencl}"=Laufzeitumgebung weitergereicht. Dadurch ist \textit{sycl"=gtx}
sehr portabel, da es nicht auf eine bestimmte Hardware beschränkt ist;
grundsätzlich soll es mit jeder \gls{opencl}"=Umgebung kompatibel sein, die
mindestens den Standard in Version 1.2 unterstützt. 
\cite[vgl.][47\psqq]{zuzek2016}

\subsection{Zusammenfassung}

In der Tabelle~\ref{sycl:implementierungen:zusammenfassung:tabelle} sind die
einzelnen Implementierungen mit ihren unterstützten Hardware"=Plattformen und
dem jeweiligen Implementierungsstand zusammengefasst.

\begin{table}[htb]
    \centering
    \begin{tabulary}{\textwidth}{@{}LLL@{}}
        \toprule
        \textbf{Implementierung} & \textbf{Hardwareunterstützung} &
        \textbf{Featurestatus} \tabularnewline\midrule
        ComputeCpp & \makecell[cl]{Automotive, Embedded,\\ Intel (CPU, GPU),\\NVIDIA-GPU (experimentell)} & vollständig\tabularnewline\midrule
        Intel & \makecell[cl]{Intel-CPU, Intel-GPU,\\ Intel-FPGA (langfristig)} & vollständig\tabularnewline\midrule
        Xilinx & wie Intel, dazu Xilinx-FPGA & vollständig\tabularnewline\midrule
        triSYCL & CPU, SPIR-fähige OpenCL-Hardware & unvollständig\tabularnewline\midrule
        hipSYCL & AMD-GPU, NVIDIA-GPU & unvollständig\tabularnewline\midrule
        sycl-gtx & OpenCL-1.2-fähige Hardware & unvollständig\tabularnewline\bottomrule
    \end{tabulary}
    \caption{Übersicht der verfügbaren SYCL"=Implementierungen}
    \label{sycl:implementierungen:zusammenfassung:tabelle}
\end{table}

\section{Erweiterungen für FPGAs}\label{sycl:erweiterungen}

Für die \gls{fpga}s des Herstellers Xilinx steht bereits eine experimentelle
\gls{sycl}"=Implementierung zur Verfügung. Um die speziellen Eigenschaften
dieses Hardware"=Typs besser nutzen zu können, gibt es Erweiterungen, die
\gls{sycl}s Funktionsumfang um \gls{fpga}"=spezifische Funktionalität ergänzen.
Diese sind in der Header-Datei \texttt{CL/sycl/xilinx/fpga.hpp} definiert und
werden in den folgenden Abschnitten vorgestellt.

\subsubsection{Datenflussorientierte Ausführung}
\label{sycl:erweiterungen:xilinx:dataflow}

Aus Xilinx' OpenCL-Implementierung übernimmt der triSYCL-Compiler eine
datenflussbasierte Erweiterung. Diese Erweiterung ermöglicht die
aufgabenparallele Ausführung aufeinanderfolgender Funktionen und Schleifen. Mit
ihr wird der Compiler angewiesen, die Abhängigkeiten zwischen den einzelnen
Schritten zu analysieren und für diese Schritte das
\textit{Pro\-du\-cer}/\textit{Con\-sumer}"=Prin\-zip durch eine Zwischenschaltung von
Puffern durchzusetzen. \cite[siehe][70\psqq]{sdaccelopt2019}

In OpenCL ist diese Erweiterung als \texttt{xcl\_dataflow} verfügbar und wird
im OpenCL"=C"=Dialekt einem \textit{Kernel}, einer Funktion oder einer Schleife als
Attribut zugewiesen. Der SYCL"=Implementierung steht diese Erweiterung unter dem
Namen \texttt{dataflow} zur Verfügung. Mit ihr werden Funktionen markiert, auf
deren innere Funktionen und Schleifen die entsprechenden Optimierungen angewandt
werden (siehe Quelltext~\ref{sycl:erweiterungen:xilinx:dataflow:sycl}).

\begin{code}
    \begin{minted}[fontsize=\small]{c++}
auto body(/* ... */)
{
    /* Funktionskörper */
}

struct kernel
{
    auto operator()()
    {
        cl::sycl::xilinx::dataflow(body(/* ... */));
    }
};
    \end{minted}
    \caption{Datenfluss-Erweiterung in SYCL}
    \label{sycl:erweiterungen:xilinx:dataflow:sycl}
\end{code}

\subsubsection{Pipeline-basierte Ausführung}
\label{sycl:erweiterungen:xilinx:pipeline}

Die triSYCL"=Implementierung übernimmt aus Xilinx' OpenCL-Umgebung eine
pipeline"=basierte Erweiterung. Mit dieser kann der Compiler angewiesen werden,
die Iterationen einer Schleife zu überlappen. Dadurch können die Iterationen
bestimmte Ressourcen zeitgleich nutzen, wodurch sich der Ressourcenverbrauch
insgesamt sowie die Latenz verringern können.
\cite[siehe][67\psqq]{sdaccelopt2019}

In der von Xilinx ausgelieferten OpenCL"=Implementierung handelt es sich bei
dieser Erweiterung um das Attribut \texttt{xcl\_pipeline\_loop}, mit dem
Schleifen markiert werden. In SYCL ist sie unter dem Namen \texttt{pipeline}
verfügbar und wird auf Funktionen angewendet, deren innere Schleifen dieser
Optimierung unterzogen werden (siehe
Quelltext~\ref{sycl:erweiterungen:xilinx:pipeline:sycl}).

\begin{code}
    \begin{minted}[fontsize=\small]{c++}
auto body(/* ... */)
{
    for(int i = 0; i < 32; ++i)
    {
        /* Schleifenkörper */
    }
}

struct kernel
{
    auto operator()()
    {
        cl::sycl::xilinx::pipeline(body(/* ... */));
    }
};
    \end{minted}
    \caption{Pipeline-Erweiterung in SYCL}
    \label{sycl:erweiterungen:xilinx:pipeline:sycl}
\end{code}

\subsubsection{Feldpartitionierung}
\label{sycl:erweiterungen:xilinx:partitioning}

Durch die Verteilung eines Datenfeldes auf mehrere physische Speichersegmente
lässt sich für manche Anwendungen eine höhere Speicherbandbreite erzielen.
Mit Xilinx' High-Level-Synthese lässt sich ein logisches Datenfeld auf drei
verschiedene Weisen zerlegen: \textit{cyclic}, \textit{block} und
\textit{complete}. Diese Strategien sind in
Abbildung~\ref{sycl:erweiterungen:xilinx:partitioning:img} grafisch dargestellt.
\cite[vgl.][16]{sdxpragma2019}

Der Typ \textit{cyclic} führt eine zyklische Zerlegung des Feldes durch. Geht
man von einem acht-elementigen Feld aus und hat vier physische Speicher zur
Verfügung, so werden die Elemente einzeln in aufsteigender Reihenfolge auf die
Speicher aufgeteilt: Element 0 wird dem Speicher 0 zugeordnet, Element 1 dem
Speicher 1, und so weiter. Ist Speicher 3 erreicht, beginnt die Zuteilung wieder
von vorne, Element 4 wird dem Speicher 0 zugeordnet, Element dem Speicher 1,
und so weiter. \cite[vgl.][17]{sdxpragma2019}

Der Typ \textit{block} zerlegt das Feld blockweise. Das heißt, dass zuerst der
Speicher 0 mit den ersten Elementen des Feldes befüllt wird, dann der Speicher
1, und so weiter. \cite[vgl.][17]{sdxpragma2019}

Beim Typ \textit{complete} wird das Feld in einzelne Elemente zerlegt. Dies
entspricht einer Verteilung des Feldes auf einzelne Register.
\cite[vgl.][17]{sdxpragma2019}

Das Attribut \texttt{xcl\_array\_partition(<Typ>, <Faktor>, <Dimension>)} steht
als Erweiterung in Xilinx' OpenCL-Implementierung zur Verfügung, um die
Partitionierung durchzuführen. Dabei bezeichnet \texttt{<Typ>} einen der drei
oben genannten Typen. \cite[vgl.][17]{sdxpragma2019}

\texttt{<Faktor>} gibt für \textit{cyclic} die Anzahl der Speicher an, auf die
das Feld verteilt werden soll, und für \textit{block} die Anzahl der Elemente
pro Speicher. Für den Typ \textit{complete} ist dieser Parameter nicht
definiert. \cite[vgl.][17]{sdxpragma2019}

\texttt{<Dimension>} gibt an, welche Dimension des Feldes auf die beschriebene
Weise partitioniert werden soll. \cite[vgl.][17]{sdxpragma2019}

In SYCL steht diese Erweiterung unter dem Namen \texttt{partition\_array} zur
Verfügung, wobei die Zuweisung der oben aufgeführten Parameter hier über
Templates erfolgt. Der
Quelltext~\ref{sycl:erweiterungen:xilinx:partitioning:sycl} zeigt die Anwendung
dieser Erweiterung.

\begin{code}
    \begin{minted}[fontsize=\small]{c++}
struct kernel
{
    auto operator()()
    {
        // zyklische Verteilung von a auf 4 physische Speicher
        auto a = cl::sycl::xilinx::partition_array<int, 16,
                    cl::sycl::xilinx::partition::cyclic<4, 1>>{};

        // blockweise Verteilung von b mit 4 Elementen pro physischem Speicher
        auto b = cl::sycl::xilinx::partition_array<int, 16,
                    cl::sycl::xilinx::partition::block<4, 1>>{};

        // Zerlegung von c in 16 Register
        auto c = cl::sycl::xilinx::partition_array<int, 16,
                    cl::sycl::xilinx::partition::complete<1>>{};
    }
};
    \end{minted}
    \caption{Feldpartitionierung in SYCL}
    \label{sycl:erweiterungen:xilinx:partitioning:sycl}
\end{code}

\begin{figure}
    \centering
    \begin{tikzpicture}
        % cyclic
        \draw (0.0, 0.0) rectangle (0.5, 0.5) node [pos = 0.5, align = center] {15};
        \draw (0.0, 0.5) rectangle (0.5, 1.0) node [pos = 0.5, align = center] {14};
        \draw (0.0, 1.0) rectangle (0.5, 1.5) node [pos = 0.5, align = center] {13};
        \draw (0.0, 1.5) rectangle (0.5, 2.0) node [pos = 0.5, align = center] {12};
        \draw (0.0, 2.0) rectangle (0.5, 2.5) node [pos = 0.5, align = center] {11};
        \draw (0.0, 2.5) rectangle (0.5, 3.0) node [pos = 0.5, align = center] {10};
        \draw (0.0, 3.0) rectangle (0.5, 3.5) node [pos = 0.5, align = center] {9};
        \draw (0.0, 3.5) rectangle (0.5, 4.0) node [pos = 0.5, align = center] {8};
        \draw (0.0, 4.0) rectangle (0.5, 4.5) node [pos = 0.5, align = center] {7};
        \draw (0.0, 4.5) rectangle (0.5, 5.0) node [pos = 0.5, align = center] {6};
        \draw (0.0, 5.0) rectangle (0.5, 5.5) node [pos = 0.5, align = center] {5};
        \draw (0.0, 5.5) rectangle (0.5, 6.0) node [pos = 0.5, align = center] {4};
        \draw (0.0, 6.0) rectangle (0.5, 6.5) node [pos = 0.5, align = center] {3};
        \draw (0.0, 6.5) rectangle (0.5, 7.0) node [pos = 0.5, align = center] {2};
        \draw (0.0, 7.0) rectangle (0.5, 7.5) node [pos = 0.5, align = center] {1};
        \draw (0.0, 7.5) rectangle (0.5, 8.0) node [pos = 0.5, align = center] {0};

        \draw (2.5, -0.75) rectangle (3.0, -0.25) node [pos = 0.5, align = center] {15};
        \draw (2.5, -0.25) rectangle (3.0, 0.25) node [pos = 0.5, align = center] {11};
        \draw (2.5, 0.25) rectangle (3.0, 0.75) node [pos = 0.5, align = center] {7};
        \draw (2.5, 0.75) rectangle (3.0, 1.25) node [pos = 0.5, align = center] {3};

        \draw (2.5, 1.75) rectangle (3.0, 2.25) node [pos = 0.5, align = center] {14};
        \draw (2.5, 2.25) rectangle (3.0, 2.75) node [pos = 0.5, align = center] {10};
        \draw (2.5, 2.75) rectangle (3.0, 3.25) node [pos = 0.5, align = center] {6};
        \draw (2.5, 3.25) rectangle (3.0, 3.75) node [pos = 0.5, align = center] {2};

        \draw (2.5, 4.25) rectangle (3.0, 4.75) node [pos = 0.5, align = center] {13};
        \draw (2.5, 4.75) rectangle (3.0, 5.25) node [pos = 0.5, align = center] {9};
        \draw (2.5, 5.25) rectangle (3.0, 5.75) node [pos = 0.5, align = center] {5};
        \draw (2.5, 5.75) rectangle (3.0, 6.25) node [pos = 0.5, align = center] {1};

        \draw (2.5, 6.75) rectangle (3.0, 7.25) node [pos = 0.5, align = center] {12};
        \draw (2.5, 7.25) rectangle (3.0, 7.75) node [pos = 0.5, align = center] {8};
        \draw (2.5, 7.75) rectangle (3.0, 8.25) node [pos = 0.5, align = center] {4};
        \draw (2.5, 8.25) rectangle (3.0, 8.75) node [pos = 0.5, align = center] {0};

        \node [align = center] (cyclictext) at (1.5, 9.75) {\textbf{cyclic}};

        \draw [-Latex] (0.5, 7.75) -- (2.5, 8.5); % 0
        \draw [-Latex] (0.5, 7.25) -- (2.5, 6.0); % 1
        \draw [-Latex] (0.5, 6.75) -- (2.5, 3.5); % 2
        \draw [-Latex] (0.5, 6.25) -- (2.5, 1.0); % 3

        \draw [line width = 0.75mm] (3.5, -1.0) -- (3.5, 9.0); 

        % block
        \draw (4.0, 0.0) rectangle (4.5, 0.5) node [pos = 0.5, align = center] {15};
        \draw (4.0, 0.5) rectangle (4.5, 1.0) node [pos = 0.5, align = center] {14};
        \draw (4.0, 1.0) rectangle (4.5, 1.5) node [pos = 0.5, align = center] {13};
        \draw (4.0, 1.5) rectangle (4.5, 2.0) node [pos = 0.5, align = center] {12};
        \draw (4.0, 2.0) rectangle (4.5, 2.5) node [pos = 0.5, align = center] {11};
        \draw (4.0, 2.5) rectangle (4.5, 3.0) node [pos = 0.5, align = center] {10};
        \draw (4.0, 3.0) rectangle (4.5, 3.5) node [pos = 0.5, align = center] {9};
        \draw (4.0, 3.5) rectangle (4.5, 4.0) node [pos = 0.5, align = center] {8};
        \draw (4.0, 4.0) rectangle (4.5, 4.5) node [pos = 0.5, align = center] {7};
        \draw (4.0, 4.5) rectangle (4.5, 5.0) node [pos = 0.5, align = center] {6};
        \draw (4.0, 5.0) rectangle (4.5, 5.5) node [pos = 0.5, align = center] {5};
        \draw (4.0, 5.5) rectangle (4.5, 6.0) node [pos = 0.5, align = center] {4};
        \draw (4.0, 6.0) rectangle (4.5, 6.5) node [pos = 0.5, align = center] {3};
        \draw (4.0, 6.5) rectangle (4.5, 7.0) node [pos = 0.5, align = center] {2};
        \draw (4.0, 7.0) rectangle (4.5, 7.5) node [pos = 0.5, align = center] {1};
        \draw (4.0, 7.5) rectangle (4.5, 8.0) node [pos = 0.5, align = center] {0};

        \draw (6.5, -0.75) rectangle (7.0, -0.25) node [pos = 0.5, align = center] {15};
        \draw (6.5, -0.25) rectangle (7.0, 0.25) node [pos = 0.5, align = center] {14};
        \draw (6.5, 0.25) rectangle (7.0, 0.75) node [pos = 0.5, align = center] {13};
        \draw (6.5, 0.75) rectangle (7.0, 1.25) node [pos = 0.5, align = center] {12};

        \draw (6.5, 1.75) rectangle (7.0, 2.25) node [pos = 0.5, align = center] {11};
        \draw (6.5, 2.25) rectangle (7.0, 2.75) node [pos = 0.5, align = center] {10};
        \draw (6.5, 2.75) rectangle (7.0, 3.25) node [pos = 0.5, align = center] {9};
        \draw (6.5, 3.25) rectangle (7.0, 3.75) node [pos = 0.5, align = center] {8};

        \draw (6.5, 4.25) rectangle (7.0, 4.75) node [pos = 0.5, align = center] {7};
        \draw (6.5, 4.75) rectangle (7.0, 5.25) node [pos = 0.5, align = center] {6};
        \draw (6.5, 5.25) rectangle (7.0, 5.75) node [pos = 0.5, align = center] {5};
        \draw (6.5, 5.75) rectangle (7.0, 6.25) node [pos = 0.5, align = center] {4};

        \draw (6.5, 6.75) rectangle (7.0, 7.25) node [pos = 0.5, align = center] {3};
        \draw (6.5, 7.25) rectangle (7.0, 7.75) node [pos = 0.5, align = center] {2};
        \draw (6.5, 7.75) rectangle (7.0, 8.25) node [pos = 0.5, align = center] {1};
        \draw (6.5, 8.25) rectangle (7.0, 8.75) node [pos = 0.5, align = center] {0};

        \node [align = center] (cyclictext) at (5.5, 9.75) {\textbf{block}};

        \draw [-Latex] (4.5, 7.75) -- (6.5, 8.5); % 0
        \draw [-Latex] (4.5, 7.25) -- (6.5, 8.0); % 1
        \draw [-Latex] (4.5, 6.75) -- (6.5, 7.5); % 2
        \draw [-Latex] (4.5, 6.25) -- (6.5, 7.0); % 3

        \draw [line width = 0.75mm] (7.5, -1.0) -- (7.5, 9.0); 

        % complete
        \draw (8.0, 0.0) rectangle (8.5, 0.5) node [pos = 0.5, align = center] {15};
        \draw (8.0, 0.5) rectangle (8.5, 1.0) node [pos = 0.5, align = center] {14};
        \draw (8.0, 1.0) rectangle (8.5, 1.5) node [pos = 0.5, align = center] {13};
        \draw (8.0, 1.5) rectangle (8.5, 2.0) node [pos = 0.5, align = center] {12};
        \draw (8.0, 2.0) rectangle (8.5, 2.5) node [pos = 0.5, align = center] {11};
        \draw (8.0, 2.5) rectangle (8.5, 3.0) node [pos = 0.5, align = center] {10};
        \draw (8.0, 3.0) rectangle (8.5, 3.5) node [pos = 0.5, align = center] {9};
        \draw (8.0, 3.5) rectangle (8.5, 4.0) node [pos = 0.5, align = center] {8};
        \draw (8.0, 4.0) rectangle (8.5, 4.5) node [pos = 0.5, align = center] {7};
        \draw (8.0, 4.5) rectangle (8.5, 5.0) node [pos = 0.5, align = center] {6};
        \draw (8.0, 5.0) rectangle (8.5, 5.5) node [pos = 0.5, align = center] {5};
        \draw (8.0, 5.5) rectangle (8.5, 6.0) node [pos = 0.5, align = center] {4};
        \draw (8.0, 6.0) rectangle (8.5, 6.5) node [pos = 0.5, align = center] {3};
        \draw (8.0, 6.5) rectangle (8.5, 7.0) node [pos = 0.5, align = center] {2};
        \draw (8.0, 7.0) rectangle (8.5, 7.5) node [pos = 0.5, align = center] {1};
        \draw (8.0, 7.5) rectangle (8.5, 8.0) node [pos = 0.5, align = center] {0};

        \draw (11.5, 0.0) rectangle (12.0, 0.5) node [pos = 0.5, align = center] {15};
        \draw (11.5, 1.0) rectangle (12.0, 1.5) node [pos = 0.5, align = center] {14};
        \draw (11.5, 2.0) rectangle (12.0, 2.5) node [pos = 0.5, align = center] {13};
        \draw (11.5, 3.0) rectangle (12.0, 3.5) node [pos = 0.5, align = center] {12};

        \draw (11.5, 4.0) rectangle (12.0, 4.5) node [pos = 0.5, align = center] {11};
        \draw (11.5, 5.0) rectangle (12.0, 5.5) node [pos = 0.5, align = center] {10};
        \draw (11.5, 6.0) rectangle (12.0, 6.5) node [pos = 0.5, align = center] {9};
        \draw (11.5, 7.0) rectangle (12.0, 7.5) node [pos = 0.5, align = center] {8};

        \draw (10.5, 0.5) rectangle (11.0, 1.0) node [pos = 0.5, align = center] {7};
        \draw (10.5, 1.5) rectangle (11.0, 2.0) node [pos = 0.5, align = center] {6};
        \draw (10.5, 2.5) rectangle (11.0, 3.0) node [pos = 0.5, align = center] {5};
        \draw (10.5, 3.5) rectangle (11.0, 4.0) node [pos = 0.5, align = center] {4};

        \draw (10.5, 4.5) rectangle (11.0, 5.0) node [pos = 0.5, align = center] {3};
        \draw (10.5, 5.5) rectangle (11.0, 6.0) node [pos = 0.5, align = center] {2};
        \draw (10.5, 6.5) rectangle (11.0, 7.0) node [pos = 0.5, align = center] {1};
        \draw (10.5, 7.5) rectangle (11.0, 8.0) node [pos = 0.5, align = center] {0};

        \node [align = center] (cyclictext) at (9.5, 10.0) {\textbf{complete}};

        \draw [-Latex] (8.5, 7.75) -- (10.5, 7.75); % 0
        \draw [-Latex] (8.5, 7.25) -- (10.5, 6.75); % 1
        \draw [-Latex] (8.5, 6.75) -- (10.5, 5.75); % 2
        \draw [-Latex] (8.5, 6.25) -- (10.5, 4.75); % 3
    \end{tikzpicture}
    \caption{Darstellung der Feldpartitionierungsstrategien}
    \label{sycl:erweiterungen:xilinx:partitioning:img}
\end{figure}
