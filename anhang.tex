\chapter{Fehlerberichte und Korrekturen}
\label{anhang:fehler}

\section{Fehlerberichte und Korrekturen für die Xilinx-OpenCL-Laufzeitumgebung}
\label{anhang:fehler:xrt}

\section{Fehlerberichte und Korrekturen für den Xilinx-SYCL-Compiler}
\label{anhang:fehler:xilinx}

\section{Fehlerberichte und Korrekturen für den Intel-SYCL-Compiler}
\label{anhang:fehler:intel}

\section{Fehlerberichte und Korrekturen für den ComputeCpp-SYCL-Compiler}
\label{anhang:fehler:computecpp}

\chapter{GitHub-Diskussionen}
\label{anhang:github}

\section{Diskussionen mit dem SYCL-Spezifikationskomitee}
\label{anhang:github:syclspec}

\subsection{Why is there no way to allocate local memory inside a ND-kernel
            (parallel\_for)?}
\label{anhang:github:syclspec:staticshared}

Original: \url{https://github.com/KhronosGroup/SYCL-Docs/issues/20}, zuletzt
abgerufen am 07. August 2019.

\begin{otherlanguage}{english}
    \paragraph{Jan Stephan} I know this is possible using the hierarchical
                            \texttt{parallel\_for} invoke, but we can't do it
                            with the \texttt{nd\_item} version despite being
                            able to specify the group size. OpenCL allows this
                            (AFAIK), SYCL's competitors do, too, so why not
                            allow it in SYCL? I suppose there must be a reason
                            for leaving it out.
    \paragraph{Victor Lomuller (Codeplay)} The main reason is the host device.
                            If your compiler is not SYCL aware, you need to be
                            able to preallocate this memory before calling the
                            functor, which is not trivial without compiler
                            support.
    \paragraph{Jan Stephan} Wouldn't this be solvable by doing the inverse of
                            the hierarchical case? I.e. everything is
                            \texttt{private} by default if declared inside the
                            kernel, unless embedded with something like
                            \texttt{cl::sycl::local\_memory}.
    \paragraph{Victor Lomuller} It does not address the compiler support
                                problem. You still need to preallocate memory
                                before calling the functor, but without compiler
                                support, you cannot know the amount of memory
                                you need nor where to place the pointer to that
                                memory (the stack frame does not exist yet).
    \paragraph{Ronan Keryell (Xilinx)} which SYCL competitor can run on CPU
                              without a specific compiler? This allows for
                              example to use HellGrind \& ThreadSanitizer with
                              plain GCC or Clang to debug a SYCL program just by
                              running it on my laptop. I find this an amazing
                              feature of SYCL...
\end{otherlanguage}

\subsection{How to extract address space from raw pointers?}

Original: \url{https://github.com/KhronosGroup/SYCL-Docs/issues/21}, zuletzt
abgerufen am 07. August 2019.
