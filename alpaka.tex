\chapter{Die Alpaka-Bibliothek}\label{alpaka}

Dieses Kapitel führt in die Alpaka"=Bibliothek ein. Wie im vorherigen
SYCL"=Kapitel wird der grundlegende Aufbau eines Alpaka"=Programms anhand des
AXPY"=Beispiels dargestellt. Ein weiterer Abschnitt ist den darauf aufbauenden,
erweiterten Konzepten, wie etwa der Hardware"=Abstraktion, gewidmet.

\section{Überblick}\label{alpaka:ueberblick}

Alpaka (Eigenschreibweise: \textit{alpaka}) steht für
\textit{Abstraction Library for Parallel Kernel Acceleration} und wurde
ursprünglich von Benjamin Worpitz im Rahmen seiner Masterarbeit entwickelt
\cite[vgl.][]{worpitz2015}. Mittlerweile wird die Entwicklung durch
die Gruppe \textit{Computergestützte Strahlenphysik} des
\textit{Instituts für Strahlenphysik} am
\textit{Helmholtz"=Zentrum Dresden"=Rossendorf} fortgeführt.

Die Alpaka"=Bibliothek definiert eine abstrakte C++"=Schnittstelle, mit deren
Hilfe parallele Programme geschrieben werden können. Im Hintergrund wird Alpaka
auf hersteller- oder hardware"=spezifische Schnittstellen, wie CUDA oder OpenMP,
-- im Folgenden als \textit{Backend} bezeichnet -- abgebildet. Alpaka ist somit
ein einheitliches Paket, das die abstrakte Schnittstelle nach außen und die
konkrete Implementierung vereinigt. Damit unterscheidet sich die Bibliothek von
ähnlichen Ansätzen wie OpenCL oder SYCL, die ebenfalls eine abstrakte
Schnittstelle definieren, die Implementierung jedoch den Hardware- und
Software"=Herstellern überlassen.

Wie bei SYCL sind die Quelltexte für \textit{Host} und \textit{Device} nicht
voneinander getrennt. Die Abbildung auf ein oder mehrere Backends erfolgt zur
Compile"=Zeit durch Template"=Metaprogrammierung, wodurch ein
Abstraktions"=Overhead zur Laufzeit vermieden wird.

\subsection{AXPY und Alpaka}\label{alpaka:ueberblick:axpy}

\subsubsection{Beschleunigerwahl und Befehlswarteschlange}
\label{alpaka:ueberblick:axpy:queue}

\subsubsection{Speicherreservierung und -initialisierung}
\label{alpaka:ueberblick:axpy:buffer}

\subsubsection{Kerneldefinition und -ausführung}
\label{alpaka:ueberblick:axpy:kernel}

\subsubsection{Synchronisierung}
\label{alpaka:ueberblick:axpy:sync}

\subsubsection{Zusammenfassung}
\label{alpaka:ueberblick:axpy:zusammenfassung}

\section{Weiterführende Konzepte}\label{alpaka:konzepte}

\subsection{Hardware"=Abstraktion}

\subsection{Abhängigkeiten zwischen Kerneln}
\label{alpaka:konzepte:abhaengigkeiten}

\subsection{Fehlerbehandlung}

\subsection{Profiling}
